%% -*- coding:utf-8 -*-
\section{Описание}
Курс посвящен современным алгоритмам классического шифрования и новым
методам вскрытия данных алгоритмов посредством квантовых вычислений. 

Курс состоит из 10 лекций, продолжительностью 1 час каждая.

1 лекция содержит краткое описание курса и введение в квантовую
механику. 

2 лекция продолжает введение в квантовую механику

3 лекция содержит описание базовых принципов квантовых вычислений

4 лекция посвящена описанию классического алгоритма RSA и его связи с
задачей поиска периода функции.

5 лекция посвящена дискретному (классическому) преобразованию Фурье и
как с его помощью может быть найден период функций. Предлагается
реализация дискретного преобразования Фурье на квантовых элементах.

6 лекция содержит описание алгоритма Шора для взлома RSA

7 лекция описывает симметричные алгоритмы шифрования и алгоритм
Гровера.

8 лекция посвящена классическим алгоритмам шифрования, основанным на
сложности дискретного логарифмирования. Подробно рассматриваются
алгоритмы эллиптической криптографии.

9 лекция описывает модификацию алгоритма Шора для решения задачи
дискретного логарифмирования.

10 лекция подводит итог и описывает замену классической криптографии с
помощью квантовой. Рассматривается эксперимент Белла и то каким образом
квантовая теория вероятности отличается от классической
(Колмогоровской). Предлагается схема квантовой криптографии,
основанная на эксперименте Белла.

По ходу лекций будут даваться необходимые математические пояснения, как
то
- Линейная алгебра и операции с матрицами: перемножение матриц,
линейные операторы, собственные числа и собственные функции линейных
операторов
- Дискретная математика: малая теорема Ферма, алгоритм
Евклида и т.п.
- Классическая теория вероятности: события, случайные величины,
среднее случайной величины
