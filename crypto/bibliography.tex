%% -*- coding:utf-8 -*- 
% \begin{thebibliography}{99}
\bibliographystyle{fixedgost2008}  %% стилевой файл для оформления по
%ГОСТу
%\bibdata{main}
\bibliography{main}     %% имя библиографической базы (bib-файла)
%\end{thebibliography} 

%% \begin{thebibliography}{99}
%% \bibitem{bMandel} Л.Мандель, Э.Вольф. Оптическая когерентность и
%%   квантовая оптика. ? Пер. с англ./Под ред. В.В.Самарцева - М.:
%%   Наука. ФИЗМАТЛИТ, 2000.- 896с. 
%% \bibitem{bBelok} В.В.Белокуров, О.Д.Тимофеевская,
%%   О.А.Хрусталев. Квантовая телепортация - обыкновенное чудо. - Ижевск:
%%   НИЦ ``Регулярная и хаотическая динамика''. 2000. - 256с. 
%% \bibitem{bKilin} С.Я.Килин. Квантовая информация. - УФН, 1999, т.169,
%%   - N 5, с.507-526. 
%% \bibitem{bLlishko_1994} Д.Н.Клышко. Квантовая оптика: квантовые,
%%   классические и метафизические аспекты. - УФН, 1994, т.164, N 11,
%%   с.1187-1214. 
%% \bibitem{bLlishko_1996} Д.Н.Клышко. Неклассический свет. - УФН, 1996,
%%   т.166, N 6, с.613-638. 
%% \bibitem{bVoron} Ю.И.Воронцов. Фаза осциллятора в квантовой
%%   теории. Что это такое на самом деле? - УФН, 2002, т.172, N 8,
%%   с.907-929. 
%% \bibitem{bScully} M.O.Scully, M.S.Zubuiry. Quantum Optics. 1997,
%%   Cambridge University Press, UK, 635p. 
%% \bibitem{bLoudon} R.Loudon. The Quantum Theory of Light. Third
%%   Edition. Oxford University Press, 2000, 438. 
%% \bibitem{bYamamoto} Y.Yamamoto, A.Imamoglu.  Mesoscopic quantum
%%   optics. 1999, USA, J.Wiley \& Sons, 301p. 

%% \end{thebibliography}
