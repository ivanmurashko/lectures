%% -*- coding:utf-8 -*- 
\chapter*{Введение}
Целью данного курса является введение в современную криптографию и то
каким образом квантовая механика может быть использована для решения
сложных криптографических задач.

Курс состоит из 10 лекций, продолжительностью 1 час каждая.

Охватываются следующие вопросы
\begin{itemize}
\item Введение в квантовую механику (1, 2 лекции)
\item Описание базовых принципов квантовых вычислений (3 лекция)
\item Симметричные алгоритмы шифрования и алгоритм
Гровера. (4 лекция)
\item Классический алгоритм RSA и его связь с задачей поиска периода
  функции. (5 лекция) 
\item Дискретное (классическое) преобразование Фурье и
его применения для поиска периода период функций. Предлагается
реализация дискретного преобразования Фурье на квантовых элементах 
(6,7 лекция)
\item Алгоритм Шора для взлома RSA (8 лекция)
\item Классические алгоритмы шифрования, основанные на
сложности дискретного логарифмирования. Модификация алгоритма Шора для
решения задачи дискретного логарифмирования (9 лекция) 
\item Последняя лекция 10 лекция посвящена алгоритмам шифрования
  построенным на базе эллиптических кривых. Рассматривается алгоритм
  ECDH. Описывается модификация алгоритма Шора для решения задачи
  дискретного логарифмирования на эллиптических кривых.
\end{itemize}

По ходу лекций будут даваться необходимые математические пояснения, как
то
\begin{itemize}
\item Дискретная математика: малая теорема Ферма, алгоритм
  Евклида и т.п.
\item Общая алгебра: понятие группы, теорема Лагранжа, циклическая
  группа, понятие поля. Поля Галуа.
\item Линейная алгебра и операции с матрицами: перемножение матриц,
  линейные операторы, собственные числа и собственные функции линейных
  операторов
\item Классическая теория вероятности: события, случайные величины,
среднее случайной величины
\end{itemize}


%% Целью данного курса является введение в современную криптографию и то
%% каким образом квантовая механика может быть использована для решения
%% сложных криптографических задач.

%% В первой и второй части курса дается введение в современную классическую
%% криптографию. Основной упор делается на несимметричные методы
%% шифрования.

%% Во третьей части дается краткое введение в квантовую механику.

%% В последней, четвертой, части рассматриваются основы квантовых вычислений и их
%% применения для решения задач классической криптографии. Описаны
%% наиболее известные алгоритмы, такие как алгоритм Шора, позволяющий
%% проводить факторизацию целых чисел за линейное время, и алгоритм
%% Гровера, производящий поиск в не отсортированном массиве данных за
%% время $O\left(\sqrt{N}\right)$.
