%% -*- coding:utf-8 -*-
\subsection{Прямая сумма линейных пространств и тензорное произведение}

Допустим у нас имеется два линейных пространства $\mathbb{A}$ и 
$\mathbb{B}$ с базисами $\{a_i\} \subset \mathbb{A}$ и
$\{b_j\} \subset \mathbb{B}$ соответственно. 

\begin{definition}[Тензорное произведение линейных пространств]
Линейное пространство $\mathbb{L} = \mathbb{A} \otimes \mathbb{B}$ с
базисом $l_{ij} = a_i \otimes b_j$ называется тензорным произведением
линейных пространства $\mathbb{A}$ и $\mathbb{B}$. 
\end{definition}

\begin{definition}[Тензорное (внешнее) произведение линейных операторов]
\label{def:tensorprod}
\rindex{Тензорное произведение!определение}
Рассмотрим два линейных оператора $\hat{A}: \mathbb{A} \to \mathbb{A}$
и $\hat{B} : \mathbb{B} \to \mathbb{B}$ тогда оператор 
$\hat{L} = \hat{A} \otimes \hat{B} : \mathbb{L} \to \mathbb{L}$
действующий на базис по правилу
\[
\hat{L} l_{ij} = 
\hat{L} a_i \otimes b_j = 
\hat{A} a_i \otimes \hat{B} b_j
\]
называется тензорным произведением операторов.
\end{definition}
