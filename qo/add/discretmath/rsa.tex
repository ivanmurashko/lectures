%% -*- coding:utf-8 -*- 
\section{Алгоритм RSA}
\label{AddRSA}
\index{алгоритм RSA}
Алгоритм RSA (аббревиатура от фамилий Rivest, Shamir и Adleman) - 
несимметричный алгоритм шифрования
\footnote{Несимметричным (с открытым ключом) называется такой алгоритм
  шифрования, в 
  котором используются два различных ключа: один для шифрования, а
  второй для де-шифрования},  
основанный на сложности разложения числа на простые множители.  

\subsection{Генерация ключей}
\index{алгоритм RSA!генерация ключей}
Состоит из нескольких шагов
\input ./add/discretmath/rsagenalgo.tex

Исходные числа $p$ и $q$ держатся в секрете, потому что с их помощью
можно вычислить закрытый ключ.

\input ./add/discretmath/rsagenex.tex

\subsection{Шифрование}
\index{алгоритм RSA!шифрование}
Допустим надо зашифровать некоторое сообщение $M$. Вначале оно
переводится в целое число(числа) $m$ такое, что 
$0 < m < \phi\left(n\right)$. Далее вычисляется за зашифрованный текст
$c$:
\begin{equation}
c \equiv m^e \mod{n}
\label{eqAddRSACode}
\end{equation}

\begin{example}
\emph{(RSA. Шифрование)}
Допустим у нас есть открытый ключ $\left(n=12, e=5\right)$ (см. прим. \ref{exAddRSAKeyGen}) 
и мы хотим зашифровать следующее сообщение $m = 1101_2 = 11_{10}$. 
%(%i1) power_mod(11,5,21);         
%(%o1)                                  2
Шифротекст вычисляется по формуле \eqref{eqAddRSACode} $c \equiv 11^5 \mod{21} = 2$.
\label{exAddRSACode}
\end{example}

\subsection{Де-шифрование}
\index{алгоритм RSA!де шифрование}
$m$ может быть восстановлено из $c$ по следующей формуле:
\begin{equation}
m \equiv c^d \mod{n}.
\label{eqAddRSADeCode}
\end{equation}
Имея $m$ можно восстановить исходное сообщение $M$.
\begin{example}
\emph{(RSA. Де-шифрование)}
Допустим у нас есть закрытый ключ $\left(n=12, d=5\right)$ (см. прим. \ref{exAddRSAKeyGen}) и шифротекст $c = 2$ из примера \ref{exAddRSACode}.

%(%i2) power_mod(2,5,21);
%(%o2)                                 11
Исходный текст вычисляется по формуле \eqref{eqAddRSADeCode} $m \equiv 2^5 \mod{21} = 11 = 1101_2$.
\label{exAddRSADeCode}
\end{example}

\subsection{Доказательство}
Хотим доказать что 
\[
\left(m^e\right)^d \equiv m \mod{p \cdot q}
\]
для любых положительных числ $m$ когда $p$ и $q$ простые числа, а $e$
и $d$ удовлетворяют выражению
\[
d \cdot e \equiv 1 \mod{\phi\left(p \cdot q\right)},
\]
которое мы можем переписать в виде
\[
d \cdot e - 1 = h \left(p - 1\right)\left(q - 1\right).
\]

Таким образом
\[
m^{e\cdot d} =m m^{h \left(p - 1\right)\left(q - 1\right)}.
\]
Далее возможны два случая: когда $m$ делится на $p$ и когда $m$ и $p$
взаимно просты.

В первом случае 
\[
m^{e\cdot d} \equiv m \equiv 0 \mod{p}
\]
Во втором случае используем
\myref{addDiscretSmallFerma}{малую теорему Ферма}:
\[
m m^{h \left(p - 1\right)\left(q - 1\right)} 
= m \left(m^{p - 1}\right)^{h \left(q - 1\right)} \equiv m \cdot 1^{h
  \left(q - 1\right)} \equiv m \mod{p}.
\]
Аналогично имеем либо
\[
m^{e\cdot d} \equiv m \equiv 0 \mod{q}
\]
либо в силу малой теоремы Ферма
\[
m m^{h \left(p - 1\right)\left(q - 1\right)} 
= m \left(m^{q - 1}\right)^{h \left(p - 1\right)} \equiv m \cdot 1^{h
  \left(p - 1\right)} \equiv m \mod{q}
\]
откуда
\[
m^{e\cdot d} \equiv m \mod{p \cdot q}
\]
