%% -*- coding:utf-8 -*-
\begin{example}
\emph{(RSA. Генерация ключей)}
Выбираем два простых числа  $p = 3$ и $q = 7$. Произведение этих чисел $n = 21$. Функция Эйлера 
\(
\phi\left(n\right)=\left(p - 1 \right)\left(q - 1 \right) = 2 \cdot 6 = 12
\). 

Выбираем число $e$  (открытая экспонента), таким образом, что $1 < e < 12$ и  
$\mbox{НОД}\left( e, 12 \right) = 1$. Очевидно $e = 5$ удовлетворяет заявленным условиям. 

Вычисляем закрытую экспоненту $d \equiv 5^{-1} \mod{12}$, т. е. $d = 5$. 
Действительно $5 \cdot 5 = 25 = 2 \cdot 12 + 1$, т. е. $5 \cdot 5 \equiv 1 \mod{12}$.

Т. о. получаем
\begin{itemize}
\item Открытый ключ $\left(n=12, e=5\right)$
\item Закрытый ключ $\left(n=12, d=5\right)$
\end{itemize}
\label{exAddRSAKeyGen}
\end{example}
