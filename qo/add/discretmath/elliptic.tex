%% -*- coding:utf-8 -*-
\section{Эллиптическая криптография}

В эллиптической криптографии рассматривают определенные наборы
объектов которые образуют группу (см. \autoref{sec:add:group}). В
качестве такого набора мы будем рассматривать точки принадлежащие
некоторой кривой: 
\[
E: y^2 = x^3 +a x + b,
\]
где коэффициенты $A,B$ должны удовлетворять следующему соотношению
\[
4 a^3 + 27 b^2 \ne 0.
\]
На \autoref{fig:add:ellipticR} изображена такая кривая.

\input ./add/discretmath/figelliptic.tex

То
есть для которых в соответствии с определением \ref{def:add:group}
заданы 
\begin{enumerate}
\item Определена бинарная операция, которую мы будем называть сложением
\item Элемент $e$ будет называться нулем
\item Для каждого элемента есть соответствующий обратный элемент
\end{enumerate}
TBD
