%% -*- coding:utf-8 -*-
\section{Алгебра множеств и базовые понятия теории вероятностей}

\begin{definition}[$\sigma$- алгебра]
\label{def:sigma_algebra}
Допустим, что $\Omega$ некоторое множество, тогда семейство
$\mathcal{F}$ подмножеств множества $\Omega$ называется
$\sigma$-алгеброй если выполнены следующие условия
\begin{itemize}
\item $\mathcal{F}$ содержит $\Omega$: $\Omega \in \mathcal{F}$
\item если $\Upsilon  \in \mathcal{F}$ тогда $\Omega \setminus \Upsilon
  \in \mathcal{F}$
\item объединение или пересечение счетного подсемейства $\mathcal{F}$
  принадлежит $\mathcal{F}$
\end{itemize}
\end{definition}

\begin{definition}[Мера]
\label{def:measure} 
Если $\mathcal{F}$ некоторая $\sigma$-алгебра тогда
следующая функция 
\[
\mu: \mathcal{F} \to \left[0, \infty\right]
\]
называется мерой если 
\begin{itemize}
\item $\mu\left(\emptyset\right) = 0$
\item $\forall A, B \in \mathcal{F}, A \cap B = \emptyset$ мы имеем
$\mu\left(A \cup B\right) = \mu\left(A\right) +
  \mu\left(B\right)$ 
\end{itemize}
\end{definition}

