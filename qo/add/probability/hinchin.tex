%% -*- coding:utf-8 -*- 
\subsection{Случайные процессы}
Для полного описания некоторого случайного процесса $x\left(t\right)$
необходимо знать все функции плотности совместной вероятности
\[
p_n\left(\left\{x\right\}_n,\left\{t\right\}_n\right) = 
p_n\left(x_1,x_2,\dots,x_n,t_1,t_2,\dots,t_n\right).
\] 
Например при помощи $p_1 = p_1\left(x,t\right)$ можно подсчитать среднее 
значение случайной величины
\begin{equation}
\left<x\left(t\right)\right> = \int x p_1\left(x,t\right) dx,
\label{eqAddHinchin1}
\end{equation}
а при помощи $p_2 = p_2\left(x_1,x_2,t_1,t_2\right)$ - так называемую
двухвременную автокореляционную функцию  
\begin{equation}
r\left(t_1, t_2\right) = \left<x^{*}\left(t_1\right) x\left(t_2\right)\right> = \int
\int x_1 x_2 p_2\left(x_2,t_2,x_1,t_1\right)dx_1 dx_2.
\label{eqAddHinchin2}
\end{equation}

В том случае, когда среднее случайной величины \eqref{eqAddHinchin1} не
зависит от $t$:
\begin{equation}
\left<x\left(t\right)\right> = const_t,
\label{eqAddHinchinStatWide1}
\end{equation}
а автокорреляционная функция \eqref{eqAddHinchin2} зависит только от
разности $t_1 - t_2 = \tau$:
\begin{equation}
r\left(t_1, t_2\right) = r\left(t_1 - t_2\right) = r\left(\tau\right),
\label{eqAddHinchinStatWide2}
\end{equation}
то случайный процесс называется стационарным в широком
смысле. Очевидно, что для таких процессов справедливо следующее
тождество
\begin{eqnarray}
r\left(- \tau\right) = r^{*}\left(\tau\right),
\label{eqAddHinchinStatWide3}
\end{eqnarray}
действительно, из \eqref{eqAddHinchinStatWide2} имеем
\begin{eqnarray}
r\left(- \tau\right) = r\left(t_2 - t_1\right) = r\left(t_2,
t_1\right) = r^{*}\left(t_1, t_2\right) = r^{*}\left(\tau\right).
\nonumber
\end{eqnarray}


Для стационарных в широком смысле случайных процессов справедлива
следующая теорема,

\begin{theorem}[Хинчин-Винер]
\label{thm:khinchin_wiener}
Автокореляционная функция стационарных в широком смысле случайных процессов
и их спектральная плотность связаны прямым и обратным
преобразованием Фурье.

\begin{proof}
Доказательство будет не строгим и является попыткой объяснить смысл
этой теоремы, для чего рассмотрим следующий интеграл
\begin{equation}
\tilde{x}\left(\omega\right) = \frac{1}{2 \pi}
\int_{-\infty}^{\infty}x\left(t\right)e^{i \omega t}dt,
\label{eqAddHinchinFourier1}
\end{equation}
который можно трактовать как Фурье образ случайного процесса
$x\left(t\right)$. Для стационарных случайных процессов очевидно, что
интеграл \eqref{eqAddHinchinFourier1} в общем случае не существует.
Действительно в случае существования этого интеграла было бы
справедливо соотношение Парсеваля
\begin{equation}
\int_{-\infty}^\infty \left|\tilde{x}\left(\omega\right)\right|^2 d
\omega = \int_{-\infty}^\infty \left|x\left(t\right)\right|^2 dt,
\label{eq:add:hinchin:parseval}
\end{equation}
которое очевидным образом расходится в случае стационарного случайного
процесса. При этом в силу \eqref{eq:add:hinchin:parseval} 
$S\left(\omega\right) =\left|\tilde{x}\left(\omega\right)\right|^2$ представляет собой
распределение энергии по оси частот для конкретной реализации
процесса, а усредненное значение 
$\left<S\left(\omega\right)\right>$ представляет некоторую
характеристику случайного процесса, которое описывает распределение
энергии по частотам.

Не смотря на то что интеграл \eqref{eq:add:hinchin:parseval}
расходится для
рассматриваемых случайных процессов, как показали Хинчин и Винер,
существует интеграл
\begin{equation}
\left<\tilde{x}^{*}\left(\omega\right)\tilde{x}\left(\omega'\right)\right>
= 
\frac{1}{\left(2 \pi\right)^2}
\int_{-\infty}^{\infty}\int_{-\infty}^{\infty}\left<x^{*}\left(t\right)x\left(t'\right)\right>e^{i
  \left(\omega' t' - \omega t\right)}dtdt',
\label{eqAddHinchinFourier2}
\end{equation}
который можно трактовать как спектральную плотность мощности $S\left(\omega,\omega'\right)$
рассматриваемого случайного процесса. Таким образом, с учетом стационарности
\[
\left<x^{*}\left(t\right)x\left(t'\right)\right> = r\left(t' - t\right) = r\left(\tau\right),
\]
получим
\begin{eqnarray}
S\left(\omega, \omega'\right) =
\left<\tilde{x}^{*}\left(\omega\right)\tilde{x}\left(\omega'\right)\right>
= 
\nonumber \\
=
\frac{1}{\left(2 \pi\right)^2}
\int_{-\infty}^{\infty}\int_{-\infty}^{\infty}\left<x^{*}\left(t\right)x\left(t'\right)\right>e^{i
  \left(\omega' t' - \omega t\right)}dtdt' =
\nonumber \\
=
\frac{1}{\left(2 \pi\right)^2}
\int_{-\infty}^{\infty}\int_{-\infty}^{\infty}
r\left(\tau\right)
e^{i\left(\omega' - \omega\right) t}
e^{i\omega'\left(t' - t\right) }
dtd\tau =
\nonumber \\
=
\frac{1}{\left(2 \pi\right)^2}
\int_{-\infty}^{\infty}
e^{i \left(\omega' - \omega\right) t}dt 
\int_{-\infty}^{\infty}r\left(\tau\right)
e^{i \omega' \tau}d\tau =
\nonumber \\
= \tilde{r}\left(\omega'\right)\delta\left(\omega' - \omega\right). 
\label{eqAddHinchinFourier3}
\end{eqnarray}
Таким образом из \eqref{eqAddHinchinFourier3} следует, что Фурье образ
автокореляционной функции $\tilde{r}\left(\omega\right)$ может
рассматриваться как спектральная плотность случайного процесса
\eqref{eqAddHinchinFourier2}. 
\end{proof}
\end{theorem}
