%% -*- coding:utf-8 -*-
\subsection{Случайная величина. Функции распределения}

\begin{definition}[Случайная величина]
\label{def:random_variable}
Если $\Omega$ - пространство событий (см. определение \autoref{def:events_set}), то 
отображение $X: \Omega \to \mathbb{R}$ называется случайной
величиной. 
\end{definition}

Стоит отметить что если случайную величину записывают с помощью
большой буквы $X$, то конкретную реализацию с помощью строчной буквы
$x$, т.е. если $\omega \in \Omega$, то $x = X(\omega)$. 
В частности для дискретных случайных величин каждой реализации
$x = X(\omega)$ можно поставить в соответствие некоторую вероятность
$P\left(x\right) = P\left(\omega\right)$. 
В противном случае используют функции распределения.

\begin{definition}[Функция распределения случайной величины]
Функция распределения $F_X$ случайной величины $X$ называется функция
$F_X : \mathbb{R} \to [0,1]$, которая задается формулой
\[
F_X \left(x\right) = P\left(X \le x\right),
\]
т.о. функция распределения от аргумента $x$ равна вероятности тех
событий $\omega \in \Omega$ для которых
$X(\omega) \le x$.
\end{definition}

\begin{definition}[Плотность распределения случайной величины]
С случае непрерывных случайных величин имеет смысл говорить о
плотности распределения $f_X\left(x\right)$, 
которая определяется следующим выражением 
\[
f_X\left(x\right) = 
\frac{d F_X\left(x\right)}{d x}
\]
\end{definition}

\begin{definition}[Среднее случайной величины]
Для дискретной случайной величины, т.е. если множество элементарных
событий счетно, то среднее $E\left(X\right)$ случайной величины $X$
определяется следующим соотношением
\[
E\left(X\right) = \sum_\omega p_\omega \cdot X(\omega) = 
\sum_x p_x \cdot x,
\] 
где суммирование производится по всем возможным элементарным исходам
$\omega$, $x = X(\omega)$, $p_\omega = p_{X(\omega)} = p_x$.
Для непрерывных случайных величин это соотношение трансформируется в 
\[
E\left(X\right) = \int_{-\infty}^\infty x \cdot f_X\left(x\right) dx.
\]
\end{definition}


TBD
