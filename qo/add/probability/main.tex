%% -*- coding:utf-8 -*-
\chapter{Теория вероятностей}
Когда говорят о квантовой механике, то очень часто ее считают
прикладной частью классической теории вероятности. Такое допущение
верно лишь отчасти, вероятности лежащие в основе таких эффектов как
перепутанные состояния и неклассический свет имеют принципиально
неклассическую природу, т.е. не описываются классической
Колмогоровской теорией вероятности. Материал в этом разделе в основном
опирается на монографии \cite{bHolevo2003} и \cite{bHolevo2003add} в
которых делается 
сравнение между классической и квантовой теорией вероятности.

\input ./add/probability/kolmogorov.tex
\input ./add/probability/quantum.tex



