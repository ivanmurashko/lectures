%% -*- coding:utf-8 -*- 
\section{Процедура квантования}
Как мы уже выяснили физическим величинам в квантовой механике
соответствуют самосопряженные операторы. Если нам известна волновая
функция (например с помощью уравнения Шредингера), то зная выражение
для оператора интересующей нас физической величины можно производить
практические расчеты, например средних значений.

Процедура вывода соотношений которым подчиняются операторы называется
квантованием. 

Ниже приведена информация о связи классической механики с процедурой
квантования, а так же приведен пример квантования оператора углового
момента.

\subsection{Обобщенные координаты и импульсы}

Для полного описания механической системы необходимо задать ее
координаты и импульсы. Не всегда координата представляет собой
классическую координату, например позиции в декартовой системе
координат $(x,y,z)$. Иногда для описания рассматриваемой системы
удобно пользоваться особыми координатами, такими как например угол в
полярной системе координат. В этом случае встает вопрос что же считать
импульсом соответствующим такой координате. Для ответа на этот вопрос
следует рассмотреть лагранжиан который определяется как разность
кинетической и потенциальной энергий:
\[
\mathcal{L} = T - U.
\]
Если мы имеем совокупность обобщенных координат ${q_n}$, то лагранжиан
$\mathcal{L}$ может быть представлен как функция этих переменных и их
производных по времени:
\[
\mathcal{L} = \mathcal{L}\left(q_1, q_2, \dots, \dot{q}_1, \dot{q}_2,
\dots \right).
\]

Мы можем определить обобщенный импульс $p_i$ соответствующий обобщенной
координате $q_i$ как
\[
p_i = \frac{\partial \mathcal{L}}{\partial \dot{q}_i}
\]

Для квантовой механики особую роль играют уравнения движения в
Гамильтоновой форме:
\begin{eqnarray}
\dot{q}_i = \frac{\partial \mathcal{H}}{\partial p_i},
\nonumber \\
\dot{p}_i = - \frac{\partial \mathcal{H}}{\partial q_i},
\label{eq:add:quantel:hamilton}
\end{eqnarray}
где $\mathcal{H}$ - гамильтониан (полная энергия) системы как функция
обобщенных координат и импульсов.


\begin{example}[Прямолинейное движение материальной точки]
Допустим у нас имеется материальная точка которая движется вдоль оси
$x$ под действием силы $F = - \frac{d U}{d x}$, где $U$ -
потенциальная энергия. В этом случае
\[
\mathcal{L} = T - U = \frac{m \dot{x}^2}{2} - U(x).
\] 
Обобщенный импульс $p$, соответствующий координате $x$
\[
p = \frac{\partial \mathcal{L}}{\partial \dot{x}} = m \dot{x},
\]
что совпадает с классическим определением импульса. 

Гамильтониан 
\[
\mathcal{H}\left(x, p\right) = T + U(x) = \frac{p^2}{2 m} + U(x)
\]

Уравнения движения:
\begin{eqnarray}
\dot{x} = \frac{\partial \mathcal{H}}{\partial p} = \frac{p}{m},
\nonumber \\
\dot{p} = - \frac{\partial \mathcal{H}}{\partial x} = - \frac{d U}{d
  x} = F
\nonumber
\end{eqnarray}
где первое уравнение совпадает с определением импульса, а второе
представляет собой известный закон Ньютона
\[
F = m \ddot{x}.
\]
\end{example}

\subsection{Квантование углового момента}

\input ./add/quant/figquant.tex
 
\subsubsection{Классика}
Рассматриваемая система состоит из материальной частицы движущейся по
кругу. Обобщенными координатами, описывающими частицу являются угол
$\theta$ и радиус окружности $r$, который считается постоянным: $r =
\left. const \right|_t$ ( см. \autoref{figAddQuantAngleMoment}).

\rindex{Гамильтониан}
Гамильтониан системы имеет следующий вид 
\begin{eqnarray}
\mathcal{H} = T + U = \frac{m v^2}{2} + U\left( r \right) = 
\nonumber \\
= \frac{m r^2 \dot{\theta}^2 }{2} + U\left( r \right),
\nonumber
\end{eqnarray}
где $T = \frac{m v^2}{2}$ - кинетическая энергия частицы, 
а $U$ - потенциальная энергия, которая в силу симметрии задачи, не
зависит от угла $\theta$ и зависит только от растояния $r$.

Лангранжин системы
\[
\mathcal{L} = T - U = \frac{m r^2 \dot{\theta}^2 }{2} - U\left( r \right)
\]
Угловой момент (обобщенный импульс соотвествующий обобщенной
координате $\theta$) определяется как 
\begin{equation}
l = \frac{d \mathcal{L}}{d \dot{\theta}} = 
m r^2 \dot{\theta} = I \dot{\theta},
\label{eqAngualrMomentumClass}
\end{equation}
где через $I$ обозначено $I = m r^2$ - момент инерции (moment of
inertia).

Уравнения движения для координаты r:
\[
\frac{ \partial \mathcal{L} }{\partial r} = 
\frac{d}{d t} \frac{\partial \mathcal{L}}{\partial \dot{r}}
\]
 откуда
\begin{eqnarray}
\frac{ \partial \mathcal{L} }{\partial r} = 0,
\nonumber \\
\frac{ \partial T }{\partial r} - \frac{ \partial U }{\partial r} = 0,
\nonumber \\
\frac{\partial U}{\partial r} = m r \dot{\theta}^2,
\nonumber
\end{eqnarray}
 или же
\[
 U = \frac{m r^2 \dot{\theta}^2}{2} = \frac{l^2}{2 I}
\]
 
Таким образом гамильтониан \rindex{Гамильтониан} рассматриваемой системы
\begin{eqnarray}
\mathcal{H} = \frac{m r^2 \dot{\theta}^2 }{2} + \frac{l^2}{2 I} = 
\nonumber \\
= \frac{l^2}{2 I} + \frac{l^2}{2 I} = \frac{l^2}{I}
\label{eqHClassical}
\end{eqnarray}

\subsubsection{Квантование}

Пусть $\left|\psi\right>$ - собственная функция оператора углового
момента $\hat{L}$ отвечающая собственному числу $l$:
\begin{equation}
\hat{L} \left|\psi\right> = l \left|\psi\right>.
\label{eqLPsi}
\end{equation}
Эта же волновая функция должна удовлетворять уравнению Шредингера
\[
i \hbar \frac{\partial \left|\psi\right>}{ \partial t} = 
\hat { \mathcal{H} } \left|\psi\right>
\]
Из \eqref{eqHClassical} имеем
\[
\hat { \mathcal{H} } = \frac{\hat{L}^2}{I}
\]
откуда с учетом \eqref{eqLPsi}
\[
i \hbar \frac{\partial \left|\psi\right>}{ \partial t} = 
\frac{1}{I} \hat{L} \hat{L} \left|\psi\right> = 
\frac{l^2}{I} \left|\psi\right>,
\]
или же
\[
\frac{\partial \left|\psi\right>}{ \partial t} = 
\frac{-i l^2}{\hbar I} \left|\psi\right>
\]
Таким образом
\[
\left|\psi\left(t \right)\right> = C \cdot exp\left\{\frac{-i l^2}{\hbar
    I} t
\right\}
\]
Волновая функция должна удовлетворять условию периодичности
\[
\left|\psi\left(t \right) \right>= \left|\psi\left(t + T \right)\right>
\]
где $T$ - период колебаний. Его можно найти из
\eqref{eqAngualrMomentumClass}:
\begin{eqnarray}
\dot{\theta} = \frac{l}{I},
\nonumber \\
\theta = \theta_0 + \frac{l \cdot t}{I},
\nonumber \\
2 \pi = \frac{l \cdot T}{I},
\nonumber \\
T = \frac{2 \pi I}{l}
\nonumber 
\end{eqnarray}
таким образом
\[
2 i \pi n = \frac{-i l^2}{\hbar I} T = 
\frac{2 \pi I}{l} \frac{-i l^2}{\hbar I}
\]
откуда
\[
l = -\hbar n
\]
