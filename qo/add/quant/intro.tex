%% -*- coding:utf-8 -*-
Необходимость этого раздела вызвана в первую очередь особенностями
преподавания квантовой механики в технических вузах. Основной упор
делается на практическое применение и при этому часто упускаются
теоретические и философские основы квантовой механики.

Например подробно описываются практические приемы для обработки
результатов множества однотипных измерений, т. е. описываются правила
по которым производится расчет средних значений физических величин в
заданном квантовом состоянии. При этом упускается ответы на вопросы
возникающие при анализе единичных измерений, как то как соотносятся
показания прибора с изменение волновой функции при измерении, что
такое декогеренция и многие другие. Актуальность этих вопросов
повысилась в последнее время благодаря возможности анализировать
результаты единичных измерений и проектировать новые приборы которые
опираются на эти свойства.
