%% -*- coding:utf-8 -*-
Необходимость этого раздела вызвана в первую очередь особенностями
преподавания квантовой механики в технических вузах. Основной упор
делается на практическое применение и при этом часто упускаются
теоретические и философские основы квантовой механики.

Например подробно описываются практические приемы для обработки
результатов множества однотипных измерений, т. е. описываются правила
по которым производится расчет средних значений физических величин в
заданном квантовом состоянии. При этом упускается ответы на вопросы
возникающие при анализе единичных экспериментов, такие как соотношение
показаний прибора с изменением волновой функции при измерении, что
такое декогеренция \rindex{Декогеренция}
и многие другие. Актуальность этих вопросов
повысилась в последнее время благодаря необходимости анализировать
результаты единичных опытов, что привело к возможности
проектировать новые приборы, которые опираются на эти свойства чистых
квантовых состояний. В частности нобелевская премия по физике 2012
г. была присуждена  Сержу Арошу (Serge Haroche) и
Дэвиду Вайнленду (David J. Wineland) за ``прорывные
экспериментальные методы, которые сделали возможными измерение
отдельных квантовых систем и управление ими''. 
