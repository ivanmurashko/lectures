%% -*- coding:utf-8 -*-
\section{Золотое правило Ферми}
Золотое правило Ферми позволяет, используя временную теорию возмущений
вычислить вероятность перехода между двумя состояниями квантовой
системы.

\begin{theorem}[Золотое правило Ферми]
  \label{addQuantGoldenRuleFermi}
  Допустим, что квантовая система находится изначально в состоянии
  $\ket{i}$ и мы хотим оценить вероятность перехода в состояние 
  $\ket{f}$. Гамильтониан рассматриваемой системы состоит из
  \rindex{Гамильтониан}
  двух частей: стационарного гамильтониана $\hat{H_0}$ и слабого
  возмущения $\hat{V}$:
  \begin{equation}
    \hat{H} = \hat{H_0} + \hat{V}
    \nonumber
  \end{equation}
  В этом случае искомая вероятность задается
  \begin{equation}
    W_{i \rightarrow f} = \frac{2 \pi}{\hbar}
    \left|
    \left<
    f
    \right|
    \hat{V}
    \left|
    i
    \right>
    \right|^2 \rho
    \nonumber,
  \end{equation}
  где $\rho$ является плотностью конечных состояний.
\end{theorem}

\begin{proof}
  Допустим, что существует состояния
  $\left\{\ket{n}\right\}$ образуют полный набор собственных
  векторов гамильтониана $\hat{H_0}$, т.е.
  \begin{equation}
    \hat{H_0} \ket{ n } = E_n \ket{ n },
    \nonumber
  \end{equation}
  при этом, рассматриваемое начальное состояние  $\ket{i}$ и
  конечное состояние $\ket{f}$ принадлежат этой системе
  векторов, т. е.
  \begin{equation}
    \ket{i}, \ket{f} \in \left\{\ket{n}\right\}.
    \nonumber
  \end{equation}
  В некоторый момент времени система оказывается в состоянии
  $\left|\psi\right>$, которое удовлетворяет уравнению Шредингера:
  \begin{equation}
    i \hbar \frac{\partial \left|\psi\right>}{\partial t} =
    \hat{H} \left|\psi\right>.
    \label{eqAddQuantGoldenRuleFermiShred}
  \end{equation}
  Наша задача заключается в вычислении вероятности того что при
  наблюдении система окажется в состоянии $\ket{f}$, при том
  что начальное условие для \eqref{eqAddQuantGoldenRuleFermiShred}
  имеет следующий вид
  \begin{equation}
    \left.\left|\psi\right>\right|_{t=0} = \ket{i}.
    \label{eqAddQuantGoldenRuleFermiInitialCond}
  \end{equation}

  В силу полноты системы $\left\{\ket{n}\right\}$ состояние
  $\left|\psi\right>$ может быть представлено в виде
  \begin{equation}
    \left|\psi\right> = \sum_n a_n\left(t\right) \ket{n}
    e^{\frac{-i E_n t}{\hbar}}
    \label{eqAddQuantGoldenRuleFermiExp}
  \end{equation}
  Подставив \eqref{eqAddQuantGoldenRuleFermiExp} в уравнение Шредингера
  \eqref{eqAddQuantGoldenRuleFermiShred} получим
  \begin{eqnarray}
    i \hbar \sum_n \frac{ \partial a_n\left(t\right)}{\partial t }
    \ket{n} e^{\frac{-i E_n t}{\hbar}} +
    \sum_n a_n\left(t\right) \ket{n}
    i \hbar \frac{\partial e^{\frac{-i E_n t}{\hbar}}}{ \partial t} =
    \\ \nonumber
    = i \hbar \sum_n  e^{\frac{-i E_n t}{\hbar}} \ket{n} \left(
    \frac{ \partial a_n\left(t\right)}{\partial t } - \frac{i E_n}{\hbar} 
    \right) =
    \\ \nonumber
    =  i \hbar \sum_n  e^{\frac{-i E_n t}{\hbar}}
    \frac{ \partial a_n\left(t\right)}{\partial t } \ket{n}+
    \sum_n E_n a_n\left(t\right) \ket{n}
    e^{\frac{-i E_n t}{\hbar}} =
    \nonumber \\
    =
     i \hbar \sum_n  e^{\frac{-i E_n t}{\hbar}}
     \frac{ \partial a_n\left(t\right)}{\partial t } \ket{n}+
      \sum_n E_n a_n\left(t\right) \ket{n}
    e^{\frac{-i E_n t}{\hbar}}.
    \label{eqAddQuantGoldenRuleFermiExp2}
  \end{eqnarray}
  С другой стороны выражение \eqref{eqAddQuantGoldenRuleFermiExp2}
  должно равняться
  \begin{equation}
    \hat{H} \left|\psi\right> =
    \sum_n E_n a_n\left(t\right) \ket{n}
    e^{\frac{-i E_n t}{\hbar}} +
    \sum_n  a_n\left(t\right) 
    e^{\frac{-i E_n t}{\hbar}} \hat{V} \ket{n},
    \nonumber
  \end{equation}
  т. е.
  \begin{eqnarray}
    i \hbar \sum_n  e^{\frac{-i E_n t}{\hbar}}
    \frac{ \partial a_n\left(t\right)}{\partial t } \ket{n} =
     \sum_n  a_n\left(t\right) 
    e^{\frac{-i E_n t}{\hbar}} \hat{V} \ket{n}.
    \label{eqAddQuantGoldenRuleFermiExp3}
  \end{eqnarray}
  Выражение \eqref{eqAddQuantGoldenRuleFermiExp3} будем решать
  методом теории возмущений, т.е.
  \begin{equation}
    a_n\left(t\right) = a_n^{(0)}\left(t\right) +
    a_n^{(1)}\left(t\right) + \dots,    
    \nonumber
  \end{equation}
  при этом будем полагать, что
  \[
  a_n^{(0)}\left(t\right) = \left.const\right|_t = a_n^{(0)}\left(0\right),
  \]
  откуда с учетом начального условия
  \eqref{eqAddQuantGoldenRuleFermiInitialCond}
  \[
  a_n^{(0)}\left(t\right) = \delta_{ni}.
  \]
  Таким образом для первого порядка теории возмущений из
  \eqref{eqAddQuantGoldenRuleFermiExp3} получим
  \begin{eqnarray}
    i \hbar \sum_n  e^{\frac{-i E_n t}{\hbar}}
    \frac{ \partial a_n^{(1)}\left(t\right)}{\partial t } \ket{n} =
     \sum_n  \delta_{ni} 
     e^{\frac{-i E_n t}{\hbar}} \hat{V} \ket{n} =
     e^{\frac{-i E_i t}{\hbar}} \hat{V} \ket{i}.
    \label{eqAddQuantGoldenRuleFermiExp4}
  \end{eqnarray}
  Следовательно из \eqref{eqAddQuantGoldenRuleFermiExp4} имеем
  \begin{equation}
    \frac{ \partial a_n^{(1)}\left(t\right)}{\partial t }  =
    - \frac{i}{\hbar} e^{\frac{-i \left(E_i - E_n\right) t}{\hbar}}
    \bra{n}\hat{V}\ket{i},
    \nonumber
  \end{equation}
  обозначив $E_i - E_n = \hbar \omega_{in}$, где $\omega_{in}$ -
  частота перехода между состояниями $\ket{i}$ и
  $\ket{n}$, после интегрирования получим
  \begin{equation}
  a_n^{(1)}\left(t\right)  =
    - \frac{i}{\hbar} \int_0^t e^{-i \omega_{in} t'}
    \bra{n}\hat{V}\ket{i} dt',
    \nonumber
  \end{equation}
  т.е.
  \begin{eqnarray}
  a_n^{(1)}\left(t\right)  =
  \frac{1}{\hbar \omega_{in}} \bra{n}\hat{V}\ket{i}
  \left(e^{-i \omega_{in} t} -  1\right) =
  \\ \nonumber =
  \frac{1}{\hbar \omega_{in}} \bra{n}\hat{V}\ket{i}
  e^{\frac{-i \omega_{in} t}{2}}
  \left(e^{\frac{-i \omega_{in} t}{2}} -  e^{\frac{i \omega_{in}
      t}{2}}\right) =
  \\ \nonumber
  =
  - \frac{2 i}{\hbar \omega_{in}} \bra{n}\hat{V}\ket{i}
  e^{\frac{-i \omega_{in} t}{2}}
  \sin\left(\frac{ \omega_{in} t}{2}\right)
  \nonumber
  \end{eqnarray}
  
  
  Нас интересует скорость $W_{i \rightarrow f}$ перехода в
  состояние $\ket{f}$, которая задается выражением
  \[
  W_{i \rightarrow f} = \frac{\left|a_f^{(1)}\left(t\right)\right|^2}{t}.
  \]
  Или же
  \begin{eqnarray}
    W_{i \rightarrow f} =
    \frac{t}{\hbar^2}
    \left|\bra{f}\hat{V}\ket{i}\right|^2
    \frac{\sin^2\left(\frac{ \omega_{if} t}{2}\right)}
         {\left(\frac{ \omega_{if} t}{2}\right)^2}.
    \label{eqAddQuantGoldenRuleFermiExp5}
  \end{eqnarray}

  В случае если финальное состояние $\ket{f}$ представляет
  собой некоторый непрерывный спектр вблизи энергии $E_f$, то
  выражение \eqref{eqAddQuantGoldenRuleFermiExp5} должно быть
  просуммировано для всех финальных состояний состояний у которых
  $\omega_{nf} \approx 0$. Введем понятие плотности состояний
  $\rho(E) = \frac{dE}{dn}$, так что число состояний $dn$ приходящееся
  на бесконечно малый интервал энергий $dE$ может быть представлено в
  виде $dn = \rho dE$. Следовательно скорость перехода может быть
  переставлена в виде:
  \begin{equation}
    W_{i \rightarrow f} = \sum_{n: E_n \approx E_f}
    \rho\left(E_n\right) W_{i \rightarrow f}\left(E_n\right), 
    \nonumber
  \end{equation}
  т.е. предполагая для рассматриваемого интервала энергий
  $\rho\left(E_n\right) \approx \rho\left(E_f\right)$,
  \[
  \left|\bra{n}\hat{V}\ket{i}\right|^2 \approx
  \left|\bra{f}\hat{V}\ket{i}\right|^2
  \]
  и заменяя суммирование интегрированием, получим
  \begin{eqnarray}
    W_{i \rightarrow f} \approx
    \int_{E_n \approx E_f} \rho\left(E_n\right) d E_n      \frac{t}{\hbar^2}
    \left|\bra{n}\hat{V}\ket{i}\right|^2
    \frac{\sin^2\left(\frac{ \omega_{in} t}{2}\right)}
         {\left(\frac{ \omega_{in} t}{2}\right)^2} =
         \\ \nonumber
         = \int_{\omega_{in} \approx \omega_{if}} \rho\left(E_n\right)
         d E_n      \frac{t}{\hbar^2}
    \left|\bra{n}\hat{V}\ket{i}\right|^2
    \frac{\sin^2\left(\frac{ \omega_{in} t}{2}\right)}
         {\left(\frac{ \omega_{in} t}{2}\right)^2} \approx
         \\ \nonumber
    \approx
    \frac{2}{\hbar}
    \left|\bra{f}\hat{V}\ket{i}\right|^2 \rho
    \int_{-\infty}^{\infty}
     d \left(\frac{ \omega t}{2}\right)
    \frac{\sin^2\left(\frac{ \omega t}{2}\right)}
         {\left(\frac{ \omega t}{2}\right)^2} =
    \frac{2 \pi}{\hbar} 
    \left|\bra{f}\hat{V}\ket{i}\right|^2 \rho,
    \label{eqAddQuantGoldenRuleFermiExp6}
  \end{eqnarray}
  что совпадает с доказываемым выражением.
  
  
\end{proof}
