%% -*- coding:utf-8 -*-
\section{Золотое правило Ферми}
Золотое правило Ферми позволяет, используя временную теорию возмущений
вычислить вероятность перехода между двумя состояниями квантовой
системы.

\begin{theorem}
\emph{(Золотое правило Ферми)}
\label{addQuantGoldenRuleFermi}
Допустим, что квантовая система находится изначально в состоянии
$\left|i\right>$ и мы хотим оценить вероятность перехода в состояние 
$\left|f\right>$. Гамильтониан рассматриваемой системы состоит из
двух частей: стационарного гамильтониана $\hat{H_0}$ и слабого
возмущения $\hat{V}$:
\begin{equation}
  \hat{H} = \hat{H_0} + \hat{V}
  \nonumber
\end{equation}
В этом случае искомая вероятность задается
\begin{equation}
  W_{i \rightarrow f} = \frac{2 \pi}{\hbar}
  \left|
  \left<
  f
  \right|
  \hat{V}
  \left.
  i
  \right>
  \right|^2 \rho
  \nonumber,
\end{equation}
где $\rho$ является плотностью конечных состояний.
\end{theorem}

\begin{proof}
Допустим, что существует конечное число собственных состояний
гамильтониана $\hat{H_0}$ - $\left|n\right>$, т.е.
\begin{equation}
  \hat{H_0} \left| n \right> = E_n \left| n \right>,
  \nonumber
\end{equation}
кроме того система этих состояний является полной,
таким образом произвольное решение
$\left|\psi\right>$
уравнения Шредингера
\begin{equation}
  i \hbar \frac{\partial \left|\psi\right>}{\partial t} =
  \hat{H} \left|\psi\right>
  \label{eqAddQuantGoldenRuleFermiShred}
\end{equation}
может быть представлено в виде
\begin{equation}
  \left|\psi\right> = \sum_n a_n\left(t\right) \left|n\right>
  e^{\frac{-i E_n t}{\hbar}}
  \label{eqAddQuantGoldenRuleFermiExp}
\end{equation}
Подставив (\ref{eqAddQuantGoldenRuleFermiExp}) в уравнение Шредингера
(\ref{eqAddQuantGoldenRuleFermiShred}) получим
\begin{eqnarray}
  i \hbar \sum_n \frac{ \partial a_n\left(t\right)}{\partial t }
  \left|n\right> e^{\frac{-i E_n t}{\hbar}} +
  \sum_n a_n\left(t\right) \left|n\right>
  i \hbar \frac{\partial e^{\frac{-i E_n t}{\hbar}}}{ \partial t} =
  i \hbar \left( \right)
  \nonumber
\end{eqnarray}

\end{proof}
