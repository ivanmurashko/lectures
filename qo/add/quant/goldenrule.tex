%% -*- coding:utf-8 -*-
\section{Золотое правило Ферми}
\label{addQuantGoldenRuleFermi}
Золотое правило Ферми позволяет, используя временную теорию возмущений
вычислить вероятность перехода между двумя состояниями квантовой
системы.

Допустим, что квантовая система находится изначально в состоянии
$\left|i\right>$ и мы хотим оценить вероятность перехода в состояние 
$\left|f\right>$. Гамильтониан рассматриваемой системы состоит из
двух частей: стационарного гамильтониана $\hat{H_0}$ и слабого
возмущения $\hat{V}$. В этом случае искомая вероятность задается
\begin{equation}
  W_{i \rightarrow f} = \frac{2 \pi}{\hbar}
  \left|
  \left<
  f
  \right|
  \hat{V}
  \left.
  i
  \right>
  \right|^2 \rho
  \nonumber,
\end{equation}
где $\rho$ является плотностью конечных состояний.
TBD
