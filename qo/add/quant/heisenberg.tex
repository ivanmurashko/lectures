%% -*- coding:utf-8 -*- 
\section{Измеримость физических величин}
\subsection{Принцип неопределенности Гейзенберга}
\label{AddHeisenbergUncertaintyPrinciple}
Допустим два оператора $\hat{A}$ и $\hat{B}$ не коммутируют друг с
другом, т.е. 
\begin{equation}
\left[
\hat{A}\hat{B}
\right] = 
\hat{A}\hat{B} - \hat{B}\hat{A} = i \hat{C} \ne 0,
\nonumber
\end{equation}
где $\hat{C}$ некоторый эрмитовый оператор.

Допустим система находится в состоянии $\left|\psi\right>$. Тогда
средние значения операторов выражаются следующими соотношениями:
\begin{eqnarray}
\left<\psi\right|\hat{A}\left|\psi\right> = \left<\hat{A}\right>,
\nonumber \\
\left<\psi\right|\hat{B}\left|\psi\right> = \left<\hat{B}\right>.
\nonumber
\end{eqnarray}

Определим неопределённость измерения величин $A$ и $B$ как их
дисперсии следующим образом
\begin{eqnarray}
\Delta A = \sqrt{\left<\psi\right|
\hat{\mathcal{A}}^2\left|\psi\right>}, 
\nonumber \\
\Delta B = \sqrt{\left<\psi\right|
\hat{\mathcal{B}}^2\left|\psi\right>}, 
\nonumber
\end{eqnarray}
где
\begin{eqnarray}
\hat{\mathcal{A}} = \hat{A}-\left<\hat{A}\right>, 
\nonumber \\
\hat{\mathcal{B}} = \hat{B}-\left<\hat{B}\right>.
\nonumber
\end{eqnarray}

Введем оператор $\hat{D}$ следующим образом
\begin{equation}
\hat{D} = \hat{\mathcal{A}} + i \lambda \hat{\mathcal{B}}.
\nonumber
\end{equation}
Рассмотрим оператор $\hat{D}^{\dag}\hat{D}$, который является
эрмитовым. Его среднее в состоянии $\left|\psi\right>$:
\begin{eqnarray}
\left<\psi\right|\hat{D}^{\dag}\hat{D}\left|\psi\right> = 
\left<\phi\right|\left.\phi\right> \ge 0,
\nonumber
\end{eqnarray}
где
$\left|\phi\right> = \hat{D}\left|\psi\right>$. С другой стороны 
\begin{eqnarray}
\left<\psi\right|\hat{D}^{\dag}\hat{D}\left|\psi\right> = 
\left<\psi\right|\left(\hat{\mathcal{A}} - i \lambda \hat{\mathcal{B}}\right)
\left(\hat{\mathcal{A}} + i \lambda \hat{\mathcal{B}}\right)\left|\psi\right> =
\nonumber \\
=
\left<\psi\right|\hat{\mathcal{A}}^2\left|\psi\right> +
\lambda^2\left<\psi\right|\hat{\mathcal{B}}^2\left|\psi\right> +
i \lambda 
\left<\psi\right|
\left[ \mathcal{\hat{A}}, \mathcal{\hat{B}}\right]
\left|\psi\right>
 = 
\nonumber \\
=
\left(\Delta A\right)^2 + \lambda^2 \left(\Delta B\right)^2 +
i \lambda 
\left<\psi\right|
\left[ \hat{A}, \hat{B}\right]
\left|\psi\right> = 
\nonumber \\
=
\lambda^2 \left(\Delta B\right)^2 - 
\lambda \left<C\right> + \left(\Delta A\right)^2 \ge 0.
\nonumber
\end{eqnarray}
Рассмотрим многочлен 
\[
f\left(\lambda\right) = \lambda^2 \left(\Delta B\right)^2 - 
\lambda \left<C\right> + \left(\Delta A\right)^2.
\]
Имеем $f\left( \pm \infty \right) > 0$ т.о. 
$f\left(\lambda\right) \ge 0$ если этот многочлен имеет не больше
одного вещественного корня, т.е.
\[
\left<C\right>^2 - 4 \left(\Delta A\right)^2 \left(\Delta B\right)^2
\le 0
\]
или
\begin{equation}
  \Delta A \Delta B \ge \frac{\left|\left< C \right>\right|}{2},
  \label{eqAddHeisenbergUncertaintyPrinciple}
\end{equation}
что представляет собой неравенство Гейзенберга. 

\subsection{Соотношение неопределенности энергия-время}
\label{AddHeisenbergUncertaintyPrincipleEnergyTime}
Время в квантовой механике не имеет соответствующего оператора и
поэтому для оценки времени следует использовать некоторую отдельную
наблюдаемую $\hat{O}$. С помощью
\eqref{eqAddHeisenbergUncertaintyPrinciple} можно получить следующее
соотношение
\begin{eqnarray}
  \Delta E \Delta O \ge \frac{\left|\left< C \right>\right|}{2},
  \nonumber
\end{eqnarray}
где, с учетом \eqref{eqAddWaveFunc_HeizenbergT},
\[
\frac{d \hat{O}}{d t} = \frac{i}{\hbar}
\left[\hat{\mathcal{H}}, \hat{O}\right]
\]
и следовательно
\[
C = \frac{1}{i}\left[\hat{\mathcal{H}}, \hat{O}\right] =
- \hbar \frac{d \hat{O}}{d t}.
\]
Таким образом
\begin{eqnarray}
  \Delta E \Delta O \ge \frac{\left|\left< C \right>\right|}{2} =
  \frac{\hbar}{2}\left|\left<\frac{d \hat{O}}{d t}\right>\right|.
  \nonumber
\end{eqnarray}
Обозначив $\Delta t = \frac{\Delta O}{\left|\left<\frac{d \hat{O}}{d
    t}\right>\right|}$
\footnote{
  Мы можем считать, что при малых временах наблюдаемая $\hat{O}$
  меняется линейно, т.е. $\left<\frac{d \hat{O}}{dt}\right> =
  \frac{\Delta O}{\Delta t}$. Т. о. $\Delta t$ это время значимого
  изменения наблюдаемой $O$. 
}
окончательно получим
\begin{eqnarray}
  \Delta E \Delta t \ge \frac{\hbar}{2}.
  \label{eqAddHeisenbergUncertaintyPrincipleET}
\end{eqnarray}

Стоит отметить, что состояния с определенной энергией не противоречат
соотношению \eqref{eqAddHeisenbergUncertaintyPrincipleET} поскольку
несмотря на то что $\delta E = 0$, мы имеем для любой наблюдаемой
$\left<\frac{d \hat{O}}{dt}\right> = 0$ и следовательно, $\Delta t =
\infty$. 

\subsection{Одновременная измеримость физических величин}
\label{AddHeisenbergUncertaintyPrincipleMesuranmet}
Вопрос об одновременной измеримости двух физических
величин имеет большой смысл, особенно при исследовании
квантово-механических парадоксов, таких как парадокс
ЭПР (см. \autoref{sec:part3:epr}).
\rindex{парадокс ЭПР}

Допустим у нас имеется две наблюдаемых $\hat{A}, \hat{B}$, каждой из
которых соответствует набор собственных функций $\ket{a_i},
\ket{b_i}$ и значений $\{a_i\}, \{b_i\}$, т. е.
\begin{eqnarray}
  \hat{A}\ket{a_i} = a_i \ket{a_i},
  \nonumber \\
  \hat{B}\ket{b_i} = b_i \ket{b_i}.
  \nonumber
\end{eqnarray}
Если в какой-то момент времени мы сможем измерить значения этих
величин одновременно,то полученные значения можно описать числами $a$
и $b$, при этом очевидно что $a \in \{a_i\}, b \in \{b_i\}$ и
\begin{eqnarray}
  \hat{A}\ket{a} = a \ket{a},
  \nonumber \\
  \hat{B}\ket{b} = b \ket{b}.
  \nonumber
\end{eqnarray}
С учетом того, что измерение величин проводилось одновременно, в силу
редукции волновой функции (см. \autoref{sec:add:reduction}) имеем
\[
\ket{a} = \ket{b},
\]
т. е. операторы $\hat{A}$ и $\hat{B}$ имеют общие собственные
функции. При этом необязательно чтобы все собственные вектора этих
операторов совпадали, достаточно совпадения единственного вектора.

Стоит отметить, что если операторы коммутируют, то наборы собственных
векторов операторов совпадают, что находится в соответствии с
неравенством Гейзенберга которое в данном случае запишется в виде
\[
\Delta a \Delta b \ge 0.
\]

Действительно если
\[
\hat{A} \ket{a} = a \ket{a}, 
\]
то, с учетом $\hat{A}\hat{B} = \hat{B}\hat{A}$
\[
\hat{A}\hat{B}\ket{a} = 
\hat{B}\hat{A}\ket{a} =
a \hat{B}\ket{a}.
\]
Обозначив $\ket{b} = \hat{B}\ket{a}$ имеем
\[
\hat{A}\ket{b} = 
a \ket{b},
\]
т. е.
$\ket{b} = \ket{a}$ и данный вектор является собственным
для обеих операторов $\hat{A}$ и  $\hat{B}$. Т. о. если операторы
коммутируют то они обладают общим базисом (набором собственных
векторов). Обратное тоже верно см. \cite{bHolevo2016}.
