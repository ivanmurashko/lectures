%% -*- coding:utf-8 -*-
\section{Поля}
\begin{definition}[Поле]
  Рассмотрим некоторое множество $\mathbb{F}$ с определенной на ней
  бинарной операцией $+$.
  The ring $\left(R, \oplus, \odot\right)$ is called as a field if
  $\left(R \setminus \{0\}, \odot\right)$ is an \autoref{def:abeliangroup}.

  The inverse element to $a$ in
  $\left(R \setminus\{0\}, \odot\right)$ is denoted as $a^{-1}$
  \label{def:field}
\end{definition}

\begin{example}[Поле $\mathbb{Q}$]
  Note that $\mathbb{Z}$ is not a field because not for every integer
  number an inverse exists. But if we consider a set of fractions
  $\mathbb{Q} = \left\{a/b \mid a \in \mathbb{Z}, b \in
  \mathbb{Z}\setminus\{0\}\right\}$ when it will be a field.

  The
  inverse element to $a/b$  in
  $\left(\mathbb{Q}\setminus\{0\}, \cdot\right)$  will be $b/a$.
  \label{ex:field_q}
\end{example}

\begin{example}[Поле $\mathbb{R}$]
TBD
  \label{ex:field_r}
\end{example}

\begin{example}[Поле $\mathbb{C}$]
TBD
  \label{ex:field_c}
\end{example}
