%% -*- coding:utf-8 -*-
\section{Поля}
\begin{definition}[Поле (алгебра)]
  Пусть задана Абелева группа (см. определение
  \ref{def:add:abeliangroup}) \rindex{Абелева группа} 
  $(\mathcal{F}, +)$. Единичный элемент этой группы $e_\mathcal{F}$ -
  $0$. Пусть также $(\mathcal{F} \setminus \{0\}, \cdot)$ - некоторая
  другая группа (не
  обязательно Абелева) с единичным элементом $1$. Кроме того операции
  $+,\cdot$ удовлетворяют свойству дистрибутивности, т.е. $\forall
  a,b,c \in \mathcal{F}$:
  \begin{eqnarray}
  c \cdot \left(a + b\right) = c \cdot a + c \cdot b,
  \nonumber \\
  \left(a + b\right) \cdot c = a \cdot c + b \cdot c.
  \nonumber
  \end{eqnarray}
  В этом случае $(\mathcal{F}, +, \cdot)$ называется полем.
  \label{def:field}
\end{definition}

\begin{example}[Поле $\mathbb{Q}$]
  Заметим, что $\mathbb{Z}$ не является полем, поскольку не для любого
  целого определен обратный элемент относительно операции умножения. 
  Вместе с тем следующее множество будет полем: $\mathbb{Q} =
  \left\{a/b \mid a \in \mathbb{Z}, b \in 
  \mathbb{Z}\setminus\{0\}\right\}$. При этом обратный по отношению к
  $a/b \in \left(\mathbb{Q}\setminus\{0\}, \cdot\right)$ будет $b/a$.
  \label{ex:field_q}
\end{example}

\begin{example}[Поле $\mathbb{R}$]
  Вещественные числа образуют поле.
  \label{ex:field_r}
\end{example}

\begin{example}[Поле $\mathbb{C}$]
  Комплексные числа образуют поле.
  \label{ex:field_c}
\end{example}
