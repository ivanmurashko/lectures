%% -*- coding:utf-8 -*- 
\section{Введение в теорию групп}
\label{sec:add:group}
\begin{definition}
\label{def:add:group}
Группой $\mathcal{G}$ называется множество элементов $g \in
\mathcal{G}$ для которой определена 
некоторая бинарная операция (часто ее называют умножением, либо сложением):
\begin{eqnarray}
\forall g_1,g_2 \in \mathcal{G},
\nonumber \\
g_1 \cdot g_2 \in \mathcal{G}.
\label{addGroupMulDef}
\end{eqnarray}
Данная операция \autoref{addGroupMulDef} обладает свойством
ассоциативности:
\begin{equation}
g_1 \cdot \left( g_2 \cdot g_3 \right ) = 
\left( g_1 \cdot  g_2 \right ) \cdot g_3.
\nonumber
\end{equation}
Рассматриваемое множество должно содержать элемент $e$ обладающий
следующим свойством, справедливым для любого элемента множества $g$:
\begin{equation}
g \cdot e = e \cdot g = g.
\nonumber
\end{equation}
Для каждого элемента группы $g$ должен существовать обратный
элемент $g^{-1} \in \mathcal{G}$, обладающий следующим свойством
\begin{equation}
g \cdot g^{-1} = g^{-1} \cdot g = e
\nonumber
\end{equation}
\end{definition}

\begin{definition}[Моноид]
\label{def:add:monoid}
Если для некоторого множества элементов $\mathcal{G}$ у нас не
выполнено последнее свойств группы (существование обратного элемента),
то данное множество называется моноидом (monoid) или полугруппой.
\end{definition}

\begin{definition}[Абелева группа]
\label{def:add:abeliangroup}
Группа $(\mathcal{G}, \cdot)$ называется абелевой, или коммутирующей
если $\forall a_1,a_2 \in \mathcal{A}$: $a_1 \cdot a_2 = a_2 \cdot a_1$.
\end{definition}

\begin{example}
\emph{Группа $\left(\mathbb{Z}, +\right)$}
Множество целых чисел $\mathbb{Z} = \left\{0, \pm1, \pm2,
\dots\right\}$ представляет собой группу относительно операции сложения.
\nonumber
\end{example}

\begin{definition}[Циклическая группа]
Циклическая группа $G$ - это группа которая порождена единственным
элементом $g: G = <g>$, т.е. все ее элементы являются степенями $g$.
Элемент $g$ называется порождающим элементом, или генератором, группы $G$.
\label{def:add:algebra:cyclic_group}
\end{definition}

\begin{definition}[Мультипликативная группа кольца вычетов]
Рассмотрим набор целых чисел взаимно простых с $n$ и меньших $n$,
который обозначим через $\left(\mathbb{Z}/n\mathbb{Z}\right)^\times$. В
качестве операции умножения двух элементов $a,b \in
\left(\mathbb{Z}/n\mathbb{Z}\right)^\times$ примем
\[
a \cdot b = ab \mod n.
\]
Единичным элементом является $1$. Кроме того можно показать, что для
каждого $a \in \left(\mathbb{Z}/n\mathbb{Z}\right)^\times$
существует $a^{-1} \in \left(\mathbb{Z}/n\mathbb{Z}\right)^\times$
такой что $a \cdot a^{-1} = 1$. Таким образом
$\left(\mathbb{Z}/n\mathbb{Z}\right)^\times$ является группой.
\label{def:add:algebra:mult_group}
\end{definition}

\begin{theorem}[О порядке $\left(\mathbb{Z}/n\mathbb{Z}\right)^\times$]
Порядок группы $\left(\mathbb{Z}/n\mathbb{Z}\right)^\times$
определяется следующим соотношением
\[
\left|\left(\mathbb{Z}/n\mathbb{Z}\right)^\times\right| = \phi(n),
\]
где $\phi(n)$ - \myref{def:add:discretmath:eulerfun}{функция Эйлера}.
При этом если 
$n=p$ - простое число, то
\[
\left|\left(\mathbb{Z}/p\mathbb{Z}\right)^\times\right| = \phi(p),
\]
и если $a \in \left(\mathbb{Z}/p\mathbb{Z}\right)^\times, a \ne 1$,
то $a$ - генератор рассматриваемой группы, т.е.
\[
a^{p-1} = 1.
\]
\begin{proof}
TBD
\end{proof}
\label{thm:add:algebra:cyclic_mult_group}
\end{theorem}

\begin{theorem}[Лагранж]
\label{thm:lagrange}
Для любой конечной группы $\mathcal{G}$ порядок (число элементов)
любой подгруппы $\mathcal{H}$ делит порядок $\mathcal{G}$:
\[
\left|\mathcal{G}\right| = h \left|\mathcal{H}\right|,
\] 
где целое число $h$ называется кофактором подгруппы.
\end{theorem}
