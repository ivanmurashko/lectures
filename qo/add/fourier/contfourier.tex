%% -*- coding:utf-8 -*-
\section{Преобразование Фурье}
\label{sec:cont_fourier}
В этом курсе мы будем использовать следующую форму записи для
преобразования Фурье
\footnote{
Чаще всего используется следующее определение Фурье образа
\[
\tilde{x}\left(\omega\right) = \frac{1}{\sqrt{2 \pi}}
\int_{-\infty}^{\infty}x\left(t\right)e^{-i \omega t}dt.
\]
Вместе с тем в соответствии с \cite{wiki:fourier_transform} существуют
альтернативные определения отличающиеся знаком перед частотой и
коэффициентом перед интегралом.}
\begin{equation}
\tilde{x}\left(\omega\right) = \frac{1}{2 \pi}
\int_{-\infty}^{\infty}x\left(t\right)e^{i \omega t}dt,
\label{eq:direct_fourier}
\end{equation}
при этом обратное преобразование Фурье выглядит следующим образом
\begin{equation}
x\left(t\right) =
\int_{-\infty}^{\infty}\tilde{x}\left(\omega\right)e^{-i \omega t}dt, 
\label{eq:reverse_fourier}
\end{equation}

Следующая таблица содержит важные формулы обращения
\begin{table}[H]
\centering
\begin{tabular}{|c|c|}
\hline
Функция $x\left(t\right)$ &  Образ $\tilde{x}\left(\omega\right)$\\ \hline
$1$  & $\delta\left(\omega\right)$ \\
$\delta\left(t\right)$  & $\frac{1}{2 \pi}$  \\
\hline
\end{tabular}
\caption{Преобразование Фурье}
\label{tbl:fourier}
\end{table}

Из \autoref{tbl:fourier} следует важная формула (интегральное
представление дельта-функции) \cite{wiki:deltafunction}
\begin{equation}
\delta\left(\omega\right) = \frac{1}{2 \pi} \int_{-\infty}^{\infty}
e^{i \omega t} dt
\label{eq:delta_from_integral}
\end{equation}

