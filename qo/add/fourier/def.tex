%% -*- coding:utf-8 -*-
\section{Определение}
\begin{definition}
Допустим имеется $M$ отсчетов $x_0, x_1, \dots, x_{M-1}$ тогда
дискретное преобразование Фурье задается следующим соотношением
\begin{equation}
\tilde{X}_k = \sum^{M - 1}_{m = 0} x_m e^{-\frac{2 \pi}{M} k\cdot m},
\label{eqAddFourierDiscretFourier}
\end{equation}
которое так же записывается в виде
\begin{equation}
\left\{x_m\right\} \longleftrightarrow \left\{\tilde{X}_k\right\}.
\nonumber
\end{equation}
\end{definition}

На рис. \ref{picAddFourierFourier} приведен график некоторой
периодической функции и ее преобразования Фурье. 

%draw2d(yrange=[0, 9], points_joined = true, point_type = 6, points(
%map(lambda([x, y], cons(x, [y])),  makelist(x, x, 1, 64),
%abs(fft(makelist(power_mod(2, x, 21), x, 1, 64)))) ),
%user_preamble="set output 'picshorclassfourier.tex'; set terminal
%latex; set xlabel '$y$'; set ylabel '$F(y)$'"); 
\input ./add/fourier/figfourier.tex

Выражение (\ref{eqAddFourierDiscretFourier}) может быть также
переписано в матричной форме
\begin{equation}
\vec{\tilde{X}} = \hat{F} \vec{x},
\nonumber
\end{equation}
где
\begin{equation}
\vec{x} = 
\left(
\begin{array}{c}
x_0 \\
x_1 \\
\vdots \\
x_{M-1}
\end{array}
\right)
,
\vec{\tilde{X}} = 
\left(
\begin{array}{c}
\tilde{X}_0 \\
\tilde{X}_1 \\
\vdots \\
\tilde{X}_{M-1}
\end{array}
\right)
,
\nonumber
\end{equation}
а матрица $\hat{F}$ имеет вид
\begin{equation}
\hat{F} = 
\begin{pmatrix}
1 & 1 & 1 & \cdots & 1 \\
1 & e^{-i \omega} & e^{-2 i \omega} & \cdots & 
e^{-\left( M - 1 \right) i \omega} \\
1 & e^{-2 i \omega} & e^{-4 i \omega} & \cdots & 
e^{-2 \left( M - 1 \right) i \omega} \\
1 & e^{-3 i \omega} & e^{-6 i \omega} & \cdots & 
e^{-3 \left( M - 1 \right) i \omega} \\
\vdots & \vdots & \vdots & \ddots & \vdots \\
1 & e^{-\left( M - 1 \right) i \omega} & e^{-2\left( M - 1 \right) i \omega} & \cdots & 
e^{- \left( M - 1 \right)\left( M - 1 \right) i \omega} \\
\end{pmatrix}
,
\label{eqAddFourierDiscretFourierMatrixElem}
\end{equation}
где
\[
\omega = \frac{2 \pi}{M}.
\]
Для матричного элемента матрицы
(\ref{eqAddFourierDiscretFourierMatrixElem}) можно записать
\begin{equation}
F_{n m} = e^{-i \omega n m},
\label{eqAddFourierDiscretFourierMatrixElem2}
\end{equation}
где $n, m \in \{ 0, 1, \dots, M - 1\}$.
