%% -*- coding:utf-8 -*-
\subsection{Свойства дискретного преобразования Фурье}
Следует отметить следующие свойства преобразования Фурье, которые
играют основную роль в алгоритме Шора:

\begin{lemma}
\emph{(Сдвиг)}
\label{lemmaAddFourierDiscretFourierShiftTime}
Если $\left\{x_n\right\} \longleftrightarrow \left\{\tilde{X}_k\right\}$ то
$\left\{x_{\left(n - m\right) \mod M}\right\} \longleftrightarrow
\left\{e^{-i \omega m k}\tilde{X}_k\right\}$ 
\end{lemma}

\begin{proof}
Для последовательности $\left\{x_{\left(n - m\right) \mod M}\right\}$
можно записать
\begin{eqnarray}
\left\{x_{\left(n - m\right) \mod M}\right\} = 
\begin{pmatrix}
x_{-m \mod M} \\
x_{-m+1 \mod M}\\ 
\vdots \\
x_{-1 \mod M}\\
x_0 \\
x_1 \\
\vdots \\
x_{M - m - 1}
\end{pmatrix} = 
\begin{pmatrix}
x_{M - m} \\
x_{M - m + 1}\\ 
\vdots \\
x_{M - 1}\\
x_0 \\
x_1 \\
\vdots \\
x_{M - m - 1}
\end{pmatrix},
\nonumber
\end{eqnarray}
при этом
\begin{eqnarray}
\hat{F}\left\{x_{\left(n - m\right) \mod M}\right\} = 
\nonumber \\
= \frac{1}{\sqrt{M}}
\begin{pmatrix}
1 & 1 & 1 & \cdots & 1 \\
1 & e^{-i \omega} & e^{-2 i \omega} & \cdots & 
e^{-\left( M - 1 \right) i \omega} \\
1 & e^{-2 i \omega} & e^{-4 i \omega} & \cdots & 
e^{-2 \left( M - 1 \right) i \omega} \\
1 & e^{-3 i \omega} & e^{-6 i \omega} & \cdots & 
e^{-3 \left( M - 1 \right) i \omega} \\
\vdots & \vdots & \vdots & \ddots & \vdots \\
1 & e^{-\left( M - 1 \right) i \omega} & e^{-2\left( M - 1 \right) i \omega} & \cdots & 
e^{- \left( M - 1 \right)\left( M - 1 \right) i \omega} \\
\end{pmatrix}
\begin{pmatrix}
x_{M - m} \\
x_{M - m + 1}\\ 
\vdots \\
x_0 \\
\vdots \\
x_{M - m - 1}
\end{pmatrix} = 
\nonumber \\
= 
\begin{pmatrix}
\frac{x_{M - m}}{\sqrt{M}} + \frac{x_{M - m + 1}}{\sqrt{M}} + \cdots + 
\frac{x_0}{\sqrt{M}} + \cdots + \frac{x_{M - m - 1}}{\sqrt{M}}\\
\frac{x_{M - m}}{\sqrt{M}} + 
\frac{e^{-i \omega} x_{M - m + 1}}{\sqrt{M}} + 
\cdots + 
\frac{e^{-i \omega m  } x_0}{\sqrt{M}} + 
\cdots +
\frac{e^{-i \omega \left( M - 1 \right) } x_{M - m - 1}}{\sqrt{M}}\\ 
\frac{x_{M - m}}{\sqrt{M}} + 
\frac{e^{-2 i \omega} x_{M - m + 1}}{\sqrt{M}} + 
\cdots + 
\frac{e^{-2 i \omega m  } x_0}{\sqrt{M}} + 
\cdots +
\frac{e^{-2 i \omega \left( M - 1 \right) } x_{M - m - 1}}{\sqrt{M}}\\ 
\vdots \\
\frac{x_{M - m}}{\sqrt{M}} + 
\frac{e^{-m i \omega} x_{M - m + 1}}{\sqrt{M}} + 
\cdots + 
\frac{e^{-m i \omega m  } x_0}{\sqrt{M}} + 
\cdots +
\frac{e^{-m i \omega \left( M - 1 \right) } x_{M - m - 1}}{\sqrt{M}}\\ 
\vdots 
\end{pmatrix}.
\label{eqlemmaAddFourierDiscretFourierShiftTimeMatrix}
\end{eqnarray}
С учетом соотоношения
\[
e^{-i \omega k M} = 1, k \in \left\{0, 1, \dots \right\},
\]
выражение
\eqref{eqlemmaAddFourierDiscretFourierShiftTimeMatrix} может
быть переписано в следующем виде
\begin{eqnarray}
\hat{F}\left\{x_{\left(n - m\right) \mod M}\right\} = 
\nonumber \\
=
\frac{1}{\sqrt{M}}
\begin{pmatrix}
x_{M - m} + \cdots +
x_{M - m - 1}\\
e^{-i \omega m } e^{-i \omega \left( M - m \right) } x_{M - m} + 
\cdots + 
e^{-i \omega 2 m } e^{-i 2 \omega \left( M - m - 1\right) }\\ 
e^{-i \omega 2 m } e^{-i 2 \omega \left( M - m \right) } x_{M - m} + 
\cdots + 
e^{-i \omega 2 m } e^{-i 2 \omega \left( M - m - 1\right) }\\ 
\vdots 
\end{pmatrix} = 
\nonumber \\
= 
\begin{pmatrix}
\tilde{X}_0 \\
e^{-i \omega m} \tilde{X}_1 \\
e^{-i \omega 2 m} \tilde{X}_2 \\
\vdots  
\end{pmatrix}.
\nonumber
\end{eqnarray}
\end{proof}


\begin{lemma}
\emph{(Периодичность)}
\label{lemmaAddFourierDiscretFourierPeriod}
Если последовательность $\left\{x_n\right\}$ имеет период $r$: $x_n =
x_{n + r}$, а число отсчетов $M$ кратно $r$, то не нулевые члены
преобразования Фурье следуют с периодом $\frac{M}{r}$.
\begin{proof}
Действительно если $M \mod r = 0$ и $k r \mod M \ne 0$,
то
\[
1 - e^{-i \omega k r} \ne 0,
\]
откуда
\begin{eqnarray}
\tilde{X}_k = \frac{1}{\sqrt{M}}\sum_{n = 0}^{M - 1}e^{-i \omega k n}x_n = 
\nonumber \\
= \frac{1}{\sqrt{M}} \left(
\sum_{n = 0}^{r - 1}e^{-i \omega k n} x_n + 
\sum_{n = 0}^{r - 1}e^{-i \omega k \left(n + r \right) } x_{n+r} +
\right.
\nonumber \\
\left. +
\sum_{n = 0}^{r - 1}e^{-i \omega k \left(n + 2r \right) } x_{n+2r} + 
\dots  \right)=
\nonumber \\
= \frac{1}{\sqrt{M}} \left(
\sum_{n = 0}^{r - 1}e^{-i \omega k n} x_n + 
\sum_{n = 0}^{r - 1}e^{-i \omega k \left(n + r \right) } x_n + 
\sum_{n = 0}^{r - 1}e^{-i \omega k \left(n + 2r \right) } x_n + 
\dots \right) =
\nonumber \\
= \frac{1}{\sqrt{M}} \left( \sum_{n = 0}^{r - 1} x_n e^{-i \omega k n} 
\sum_{l = 0}^{\frac{M}{r}- 1} e^{-i \omega k l r } = 
\sum_{n = 0}^{r - 1} x_n e^{-i \omega k n} 
\frac{1 - e^{-i \omega k \frac{M}{r} r }}{1 - e^{-i \omega k r }}
\right) = 
\nonumber \\
=
\frac{1}{\sqrt{M}}
\frac{1 - e^{-i \omega k M }}{1 - e^{-i \omega k r}}
\sum_{n = 0}^{r - 1} x_n e^{-i \omega k n} = 0.
\label{eqAddFourierDiscretFourierPeriodCase1}
\end{eqnarray}
Если $M \mod r = 0$ и $k r \mod M = 0$, то
\[
e^{-i \omega k r } = e^{-i \frac{2 \pi }{M} k r } = 1,
\]
откуда
\begin{eqnarray}
\tilde{X}_k = \frac{1}{\sqrt{M}} \sum_{n = 0}^{M - 1}e^{-i \omega k n}x_n = 
\nonumber \\
= \frac{1}{\sqrt{M}} \left(
\sum_{n = 0}^{r - 1}e^{-i \omega k n} x_n + 
\sum_{n = 0}^{r - 1}e^{-i \omega k n } x_{n+r} + 
\sum_{n = 0}^{r - 1}e^{-i \omega k n } x_{n+2r} + 
\dots \right)=
\nonumber \\
= \frac{1}{\sqrt{M}} \left(
\sum_{n = 0}^{r - 1}e^{-i \omega k n} x_n + 
\sum_{n = 0}^{r - 1}e^{-i \omega k n  } x_n + 
\sum_{n = 0}^{r - 1}e^{-i \omega k n  } x_n + 
\dots \right) = 
\nonumber \\ 
= \frac{1}{\sqrt{M}} \frac{M}{r} \sum_{n = 0}^{r - 1}e^{-i \omega k n  } x_n \ne 0.
\label{eqAddFourierDiscretFourierPeriodCase2}
\end{eqnarray}
Таким образом из выражений 
\eqref{eqAddFourierDiscretFourierPeriodCase1} и 
\eqref{eqAddFourierDiscretFourierPeriodCase2} следует что
ненулевые члены следуют с периодом $T = \frac{M}{r}$.
\end{proof}
\end{lemma}

\begin{remark}
\emph{(Периодичность максимумов)}
\label{rem:dsp:fourier:periodprop}
Стоит отметить, что выражение 
\eqref{eqAddFourierDiscretFourierPeriodCase1} (в случае когда
период не кратен числу отсчетов: $M \mod r \ne 0$) будет приближенно
равно 0 для тех значений $k$ которые сильно отличаются от значений
кратных $\frac{M}{r}$, т. е. локальные максимумы перобразования Фурье
будут повторяться с периодом $\frac{M}{r}$.
\end{remark}

