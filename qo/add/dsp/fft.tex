%% -*- coding:utf-8 -*-
\subsection{Быстрое преобразование Фурье}
Вычисления по формуле \eqref{eqAddFourierDiscretFourier} имеют
сложность порядка $O\left(M^2\right)$, где $M$ - число элементов (отсчетов)
\footnote{Действительно необходимо получить $M$ элементов, для
  подсчета каждого из которых требуется $M$ операций умножения}. 

Существует алгоритм быстрого расчета по формуле
\eqref{eqAddFourierDiscretFourier} который имеет сложность
$O\left(M \log{M}\right)$.

Воспользовавшись парадигмой ``разделяй и властвуй''
(см. разд. \ref{addDivideAndConquer}) можно 
обратить внимание на форму записи
\eqref{eqAddFourierDiscretFourierMatrixElem2} и 
заметить, что выражение
\eqref{eqAddFourierDiscretFourier} может быть переписано в
в виде 
\begin{equation}
\tilde{X}_k = \sum^{M - 1}_{m = 0} F_{km}^{M} x_m,
\nonumber
\end{equation}
где запись $F_{km}^{M}$ обозначает, что используется матрица
\eqref{eqAddFourierDiscretFourierMatrixElem2} размера $M\times M$.
Если $M$ четно, то 
\begin{equation}
\tilde{X}_k = \sum^{M - 1}_{m = 0} F_{k,m}^{M} x_m = 
\sum^{\frac{M}{2} - 1}_{m = 0} F_{k,2m}^M x_{2m} +
\sum^{\frac{M}{2} - 1}_{m = 0} F_{k,2m + 1}^M x_{2m + 1},
\nonumber
\end{equation}
где
\begin{eqnarray}
F_{k,2m}^{M} = e^{-i \omega k 2m} = e^{-i k m \frac{2\pi}{\frac{M}{2}}
} = F_{k,m}^{\frac{M}{2}},
\nonumber \\
F_{k,2m + 1}^{M} = e^{-i \omega k \left(2m+1\right)} = 
e^{-i \omega k}e^{-i k m \frac{2\pi}{\frac{M}{2}}} = 
e^{-2\pi i \frac{k}{M}}F_{k,m}^{\frac{M}{2}},
\nonumber
\end{eqnarray}
т. е.
\begin{equation}
\tilde{X}_k = \sum^{\frac{M}{2} - 1}_{m = 0} F_{k,m}^{\frac{M}{2}} x_{2m} +
\exp{\left(-2\pi i \frac{k}{M}\right)}
\sum^{\frac{M}{2} - 1}_{m = 0}  F_{k,m}^{\frac{M}{2}} x_{2m + 1}.
\label{eqAddFourierDiscretFourierFFT}
\end{equation}
Сложность вычислений по формуле
\eqref{eqAddFourierDiscretFourierFFT} определяется 
следующим соотношением
\begin{equation}
T\left( M \right) = 2 T\left( \frac{M}{2} \right) + O\left( M \right).
\label{eqAddFourierDiscretFourierFFTComplexity}
\end{equation}
В справедливости
\eqref{eqAddFourierDiscretFourierFFTComplexity} можно убедится
если заметить что вычисления сложности $T\left( M \right)$ в 
\eqref{eqAddFourierDiscretFourierFFT} распадаются 
на две подзадачи по вычислениям сложности $T\left( \frac{M}{2}
\right)$.

Используя основную теорему о рекуррентных соотношениях (случай 2)(
см. теор. \ref{addAlgoMasterTheorem}) 
получаем 
\begin{equation}
T\left( M \right) = O\left( M \log{M} \right).
\nonumber
\end{equation}

