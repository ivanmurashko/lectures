%% -*- coding:utf-8 -*-
\chapter{Матрица и оператор плотности}
\label{AddState}

\subsection{Чистые и смешанные состояния}
В общем случае состояние квантовой системы считается определенной если
известна волновая функция описывающая это состояние. Волновую функцию
можно получить следующим образом. Производим измерения собственных
чисел операторов соотвествующих полному набору. Число этих измеряемы
величин (квантовых чисел) равно числу степеней свободы системы. С
помощью найденных квантовых чисел мы находим волновую функцию которая
является собственной функцией для каждого из операторов полного набора
при этом собственные числа должны соотвествовать измеренным квантовым
числам. В общем случае (с точностью до постоянного множителя) такая
функция только одна. Можно принять что эта функция описывает состояние
в начальный момент времени. Состояние в последующие моменты времени
может быть найдено с помощью уравнения Шредингера. Такое состояние с
точно определенной волновой функцией называется 
\textbf{чистым состоянием}.

В ряде случаев волновая функция системы не может быть однозначно
определена, например для системы с большим числом степеней свободы. В
этом случае мы рассматриваем статистическую смесь состояний в которой
каздая волновая функция входит со своим статистическим весом. То есть
система может находится в состояниях описываемых волновыми функциями
\(\left\{\left|\psi_n\right>\right\}\). При этом вероятность
нахождения системы в некотором состоянии \(\left|\psi_n\right>\) из
этого набора равна \(p_n\). При этом очевидно 
\[
\sum_n p_n = 1.
\]
Состояние которое описывается смесью чистых состояний называется
\textbf{смешанным состоянием}.

\subsection{Матрица плотности}
Для системы находящейся в смешанном состоянии для расчета средних
величин некоторого оператора удобно воспользоваться формализмом
матрицы плотности, который  был предложен Джоном Фон Нейманом и
независимо от него Ландау и Блохом в 1927 году.

В \ref{eqAddDiracMidViaRho} было показано что среднее значение
некоторого оператора \(\hat{L}\) в чистом состоянии 
\(\left|\psi_n\right>\) можно записать как
\[
\left< \hat{L} \right>_{\psi_n} = Sp \left(\hat{\rho_n} \hat{L} \right),
\]
где
\[
\hat{\rho_n} = \hat{P}_n = \left|\psi_n\right>\left<\psi_n\right|.
\]

Для смешанного состояния формулу вычисления средних можно записать
как сумму средних в чистых состояниях с заданным весом:
\[
\left< \hat{L} \right>_{mix} = \sum_n p_n \left< \hat{L}
\right>_{\psi_n}. 
\]
Таким образом среднее в смешанном состоянии можно записать в следующем
виде 
\begin{eqnarray}
\left< \hat{L} \right>_{mix} = Sp \left(\hat{\rho} \hat{L} \right),
\end{eqnarray}
где
\begin{eqnarray}
\hat{\rho} = \sum_n p_n \hat{\rho_n} = 
\sum_n p_n \left|\psi_n\right>\left<\psi_n\right|.
\end{eqnarray}



