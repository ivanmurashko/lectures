%% -*- coding:utf-8 -*-
\section{Теория вычислимости}

Машина Тюринга $M$ (см. рис. \ref{figAddAlgoDecidability}) может
трактоваться как некоторая функция $m$ в качестве 
аргумента которой является набор символов представляющий собой
начальное состояние ленты $x$. В результате обработки $x$ машина $M$
может завершить свою работу, т. е. оказаться в некотором финальном
состоянии $f \in F \subset Q$. В этом случае говорят, что $M$
принимает $x$. В этом случае значение соответствующей функции 
$m(x) = 1$. 

\begin{definition}[Язык $L\left(M\right)$ машины Тьюринга $M$]
Множество исходных данных $x$ для которых машина Тьюринга $M$ завершает
свою работу (terminate) называется языком $L\left(M\right)$. Т. е.  
$\forall x \in L\left(M\right): m(x) = 1$.
\end{definition}

В случае зависания (halt) $M$, т. е. когда $M$
оказывается в состоянии дальнейший переход из которого не определен,
говорят, что $M$ отвергает $x$. В этом случае значение соответствующей
функции $m(x) =0$. 

Так же машина $M$ на входе $x$ может никогда не
завершить свою работу. В этом случае значение функции $m(x)$
не определено.    

\begin{definition}[Рекурсивный язык (Recursive language)]
Множество строк $L$
\footnote{множество начальных состояний ленты некоторой
манины Тьюринга}
называется рекурсивным языком если существует машина Тьюринга $M$
такая, что $\forall x \in L$ соотвествующая функция $m(x) = 1$, а 
$\forall x \notin L$ соотвествующая функция $m(x) = 0$. 
\end{definition}

Т. о. $M$ соотвествующая рекурсивному языку $L\left(M\right)$ 
останавливается на любых исходных данных. Множество рекурсивных языков
обозначается $R$.

\begin{definition}[Рекурсивно перечислимый язык (Recursively enumerable language)]
Множество строк $L$
называется рекурсивно перечислимым языком если существует машина Тьюринга $M$
такая, что $\forall x \in L$ $M$ завершает свою работу в некотором
финальном состоянии, т. е. соотвествующая функция $m(x) = 1$. 
\end{definition}

Т. о. $M$ соотвествующая рекурсивно перечислимому языку $L\left(M\right)$ 
останавливается на любых исходных данных принадлежащих языку, но при
этом может никогда не завершить свою работу при исходных данных не
принадлежащих языку $L\left(M\right)$. Множество рекурсивно
перечислимых языков обозначается $RE$. Очевидно  $R \subset RE$.


Машина Тьюринга задает класс функций для которых можно определить
такое понятие как вычислимость.

\input ./add/algo/figdecidability.tex

\begin{theorem}
\emph{(Теорема Райса)}
Задача об определении любого нетривиального свойства машины Тьюринга
является неразрешимой задачей
\end{theorem}

\begin{proof}
TBD
\end{proof}

