%% -*- coding:utf-8 -*-
\section{Булева логика. Булевы формулы КНФ}

\begin{definition}[Конъюнкция (логическое ``И'' ($a \& b$))]
Для логической операции ``И'' справедлива следующая таблица
истинности \ref{tblAddAlgoAND}.
\begin{table}
\centering
\begin{tabular}{|c|c|c|}
\hline
$a$ & $b$ & $a \& b$ \\ \hline
0  & 0 & 0 \\
0  & 1 & 0 \\
1  & 0 & 0 \\
1  & 1 & 1 \\ \hline
\end{tabular}
\caption{Конъюнкция $a \& b$}
\label{tblAddAlgoAND}
\end{table}
\end{definition}

\begin{definition}[Дизъюнкция (логическое ``ИЛИ'' ($a \| b$))]
Для логической операции ``ИЛИ'' справедлива следующая таблица
истинности \ref{tblAddAlgoOR}.
\begin{table}
\centering
\begin{tabular}{|c|c|c|}
\hline
$a$ & $b$ & $a \| b$ \\ \hline
0  & 0 & 0 \\
0  & 1 & 1 \\
1  & 0 & 1 \\
1  & 1 & 1 \\ \hline
\end{tabular}
\caption{Дизъюнкция $a \| b$}
\label{tblAddAlgoOR}
\end{table}
\end{definition}

\begin{definition}[Булева формула]
Булевой формулой называется совокупность булевых литералов
объединённых логическими операциями.
\end{definition}

\begin{definition}[КНФ - Конъюнктивная нормальная форма (SAT)]
Конъюнктивной нормальной формой в булевой логике называют такую булеву
формулу которая имеет вид конъюнкции дизъюнкций литералов.
\end{definition}

\begin{theorem}
\emph{(Приводимость к КНФ)}
Любая булева формула может быть приведена к КНФ. 
\end{theorem}

\begin{proof}
Для доказательства можно использовать: закон двойного отрицания, закон
де Моргана, дистрибутивность.
\end{proof}

\begin{theorem}
\emph{(Приводимость к 3-КНФ (3-SAT))}
Любая булева формула конъюнктивной нормальной форме может быть
приведена к 3-КНФ.  
\end{theorem}

\begin{proof}
TBD
\end{proof}

