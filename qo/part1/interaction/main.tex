%% -*- coding:utf-8 -*- 
\chapter{Взаимодействие света с атомом}
\label{chInteraction}

Рассматриваются вопросы, связанные с взаимодействием квантованного
электромагнитного поля с атомом. Используется упрощенная модель атома -
двухуровневый атом. Такое упрощение оправдано при резонансном
взаимодействии и широко используется в задачах квантовой электроники и
квантовой оптики. Большое внимание уделено рассмотрению
взаимодействия атома, моды резонатора (динамической системы) с
термостатом (диссипативной системой), которое ответственно за
релаксацию динамической системы.  

\input ./part1/interaction/light.tex
\input ./part1/interaction/hamiltonian.tex
\input ./part1/interaction/interaction1.tex
\input ./part1/interaction/nondestructive.tex
\input ./part1/interaction/interaction2.tex
\input ./part1/interaction/vaickopf.tex
\input ./part1/interaction/relax.tex
\input ./part1/interaction/interaction3.tex
\input ./part1/interaction/equation1.tex
\input ./part1/interaction/equation2.tex
\input ./part1/interaction/common.tex
\input ./part1/interaction/relax2.tex
\input ./part1/interaction/langevin.tex
\input ./part1/interaction/questions.tex


%% \begin{thebibliography}{99}
%% \bibitem{bCh2Inter_Met}  Квантовая оптика. Квантование
%%   электромагнитного поля (света): Метод. указания / СПбГТУ. СПб.,
%%   1994. 
%% \bibitem{bCh2Inter_Lu} Люисселл Уильям. Излучение и шумы в квантовой
%%   электронике. М.: Наука, 1972. 
%% \bibitem{bCh2Inter_Scally}  Скалли М. Квантовая теория лазера -
%%   проблема неравновесной статистической механики. Квантовые флуктуации
%%   излучения лазера / Сост. Арекки Ф., Скалли М., Хакен Г.,
%%   Вайдлих В. М.: Мир, 1974 .
%% \bibitem{bCh2Inter_Lacs} Лэкс М. Флуктуации и когерентные
%%   явления. М.: Мир, 1974. 
%% \end{thebibliography} 
