%% -*- coding:utf-8 -*- 
\section{Релаксация динамической системы. Метод матрицы плотности}
В квантовой электронике и квантовой оптике мы имеем дело с открытыми
системами. Такие системы необходимо рассматривать при помощи
матрицы плотности. Рассматривается малая динамическая система
(атом, молекулы, мода резонатора и т.п.), слабо связанная с
большой системой, называемой резервуаром (диссипативной
системой, термостатом), состоящим из большого числа атомов,
молекул или осцилляторов, находящихся в равновесии при
определенной температуре. Таким образом, в общем случае
рассматриваем динамическую систему ($A$), взаимодействующую с
резервуаром ($B$) (см. \autoref{figPart1Ch2_add2}). Оператор
плотности комбинированной системы ($AB$) обозначим $\hat{\rho}_{AB}$.
Свернутый по переменным резервуара (термостата) оператор плотности
описывает поведение динамической системы с учетом усредненного
взаимодействия с термостатом. Имеем: 
\begin{equation}
\hat{\rho}_{A} = Sp_{B} \hat{\rho}_{AB} = 
\sum_{\left(B\right)}
\bra{B}\hat{\rho}_{AB}\ket{B}.
\end{equation}
В начальный момент времени $t_0$ динамическая система и резервуар
рассматриваются как независимые (некоррелированные). Тогда 
\begin{equation}
\hat{\rho}_{AB}\left(t_0\right) = 
\hat{\rho}_{A}\left(t_0\right)
\otimes
\hat{\rho}_{B}\left(t_0\right)
\end{equation}
где  $Sp_B \hat{\rho}_{B} = 1$, $\otimes$ - знак внешнего или
тензорного произведения. \rindex{Тензорное произведение} 

\input ./part1/interaction/fig_add2.tex

Уравнение движения оператора плотности в представлении взаимодействия
имеет вид 
\[
\frac{d}{d t}\hat{\rho}_{AB} = 
- \frac{i}{\hbar}\left[\hat{V}_{AB}, \hat{\rho}_{AB}\right],
\]
где $\hat{V}_{AB}$ - гамильтониан взаимодействия между системами $A$ 
и $B$.  Решить это уравнение можно с помощью следующего итерационного процесса.
\begin{equation}
\hat{\rho}_{AB}\left(t\right) = 
\hat{\rho}^{(0)}_{AB}\left(t\right) +
\hat{\rho}^{(1)}_{AB}\left(t\right) +
\hat{\rho}^{(2)}_{AB}\left(t\right) + \dots,
\label{eqCh2_rho_iter}
\end{equation}
где член \(\hat{\rho}^{(0)}_{AB}\left(t\right)\) находится в предположении
отсутствия взаимодействия, т. е.
\[
\hat{\rho}^{(0)}_{AB}\left(t\right) = \hat{\rho}_{AB}\left(t_0\right) = 
\hat{\rho}_{A}\left(t_0\right)
\otimes
\hat{\rho}_{B}\left(t_0\right).
\]
Последующие члены итерационного сотношения \eqref{eqCh2_rho_iter} задаются следующим соотношением
\begin{eqnarray}
\hat{\rho}^{(s)}_{AB}\left(t\right) = - \frac{i}{\hbar}\int_{t_0}^{t} dt_s
\left[\hat{V}_{AB}\left(t_s\right), \hat{\rho}^{(s - 1)}_{AB}\left(t_s\right)
\right] = 
\nonumber \\
= \left(- \frac{i}{\hbar}\right)^{s}\int_{t_0}^{t} dt_{s}
\int_{t_0}^{t_s} dt_{s -1} \dots
\nonumber \\
\dots \int_{t_0}^{t_2} dt_1 
\left[\hat{V}_{AB}\left(t_s\right), 
\left[\hat{V}_{AB}\left(t_{s -1}\right),
\dotsc
\left[\hat{V}_{AB}\left(t_1\right),
\hat{\rho}^{(0)}_{AB}\left(t_0\right)
\right]
\dots
\right]
\right] =
\nonumber \\
= \left(- \frac{i}{\hbar}\right)^{s}\int_{t_0}^{t} dt_{s}
\int_{t_0}^{t_{s}} dt_{s - 1} \dots
\nonumber \\
\dots \int_{t_0}^{t_2} dt_1 
\left[\hat{V}_{AB}\left(t_s\right), 
\left[\hat{V}_{AB}\left(t_{s-1}\right),
\dotsc
\left[\hat{V}_{AB}\left(t_1\right),
\hat{\rho}_{A}\left(t_0\right)
\otimes
\hat{\rho}_{B}\left(t_0\right)
\right]
\dotsc
\right]
\right].
\label{eqCh2_rho_iter_member}
\end{eqnarray}

Подставляя \eqref{eqCh2_rho_iter_member} в \eqref{eqCh2_rho_iter} и производя усреднение 
по переменным термостата получим
\begin{eqnarray}
\hat{\rho}_{A}\left(t\right) = 
\hat{\rho}_{A}\left(t_0\right) + 
\nonumber \\
+ Sp_{B} \sum_{\left(s\right)} \left(- \frac{i}{\hbar}\right)^{s}\int_{t_0}^{t} dt_{s} \int_{t_0}^{t_s} dt_{s-1} \dots
\nonumber \\
\dots \int_{t_0}^{t_{2}} dt_1 
\left[\hat{V}_{AB}\left(t_s\right), 
\left[\hat{V}_{AB}\left(t_{s - 1}\right),
\dotsc
\left[\hat{V}_{AB}\left(t_1\right),
\hat{\rho}_{A}\left(t_0\right)
\otimes
\hat{\rho}_{B}\left(t_0\right)
\right]
\dots
\right]
\right].
\label{eqCh2_rho_sequance}
\end{eqnarray}
Обычно ограничиваются несколькими первыми членами
разложения. Например, первые два члена разложения равны 
\begin{eqnarray}
\hat{\rho}_{A}\left(t\right) = 
\hat{\rho}_{A}\left(t_0\right) - \frac{i}{\hbar} Sp_B \left\{
\int_{t_0}^{t} 
\left[\hat{V}_{AB}\left(t_1\right), \hat{\rho}_{A}\left(t_0\right)
\otimes
\hat{\rho}_{B}\left(t_0\right)
\right]d t_1
\right\} -
\nonumber \\
-
\frac{1}{\hbar^2}
Sp_B \left\{
\int_{t_0}^{t}d t_2 
\int_{t_0}^{t_2} d t_1
\left[\hat{V}_{AB}\left(t_2\right), 
\left[\hat{V}_{AB}\left(t_1\right), 
\hat{\rho}_{A}\left(t_0\right)
\otimes
\hat{\rho}_{B}\left(t_0\right)
\right]
\right]
\right\}.
\nonumber
\end{eqnarray}
