%% -*- coding:utf-8 -*- 
\section{Релаксация атома в случае простейшего
  резервуара, состоящего из гармонических осцилляторов}
Рассмотрим многомодовое электромагнитное поле, описываемое при помощи
операторов рождения и уничтожения $\hat{a}_k^{\dag}$,
$\hat{a}_k$. Частоту $k$-ой моды обозначим через $\omega_k$. 

Это электромагнитное поле взаимодествует с некоторым двухуровневым
атомом. Атом может находится в двух состояниях 
возбужденном $\ket{a}$ и невозбужденном
$\ket{b}$. Операторы перехода по прежнему 
$\hat{\sigma}$ и $\hat{\sigma}^{\dag}$. Частоту перехода обозначим через
\rindex{Гамильтониан}
$\Omega$. Гамильтониан взаимодействия в этом случае имеет вид  
\begin{equation}
\hat{V}\left(t\right) = \hbar
\sum_{(k)} g_k \hat{a}_k^{\dag} \hat{\sigma}e^{-i \left(
\Omega - \omega_k
\right) t} +
\mbox{э.с.}
\label{eqCh2_94}
\end{equation}
где $g_k$ - константа взаимодействия.

Таким образом системой $B$ (резервуаром) у нас будет многомодовое
электромагнитное поле, а исследуемой динамической системой $A$ -
двухуровневый атом. Оператор плотности резервуара (электромагнитного
поля) обозначим через $\hat{\rho}_R$. Оператор плотности исследуемой
динамической системы (атома) через $\hat{\rho}_{at}$.

Справедливы следующие соотношения:
\begin{equation}
Sp_R\left(
\hat{a}_k^{\dag}
\hat{a}_k
\hat{\rho}_R
\right) = \bar{n}_k
\label{eqCh2_96}
\end{equation}
где $\bar{n}_k$ -
среднее число фотонов в моде резервуара; 
\begin{eqnarray}
Sp_R\left(\hat{a}_k\hat{a}_k^{\dag}\hat{\rho}_R\right) = \bar{n}_k + 1,
\nonumber \\
Sp_R\left(\hat{a}_k\hat{a}_k\hat{\rho}_R\right) = 
Sp_R\left(\hat{a}_k^{\dag}\hat{a}_k^{\dag}\hat{\rho}_R\right) = 0.
\label{eqCh2_96_add}
\end{eqnarray}
Используя эти равенства, общее уравнение
\eqref{eqCh2_93} можно представить в виде:  
%FIXME!!! check it
\begin{eqnarray}
\dot{\hat{\rho}}_{at}\left(t\right) = 
- \int_{t_0}^t\sum_{(k)}g_k^2
\left\{
\hat{\sigma}^{\dag} \hat{\sigma}\hat{\rho}_{at}\left(t'\right)
\left(
\bar{n}_k + 1
\right)
e^{i\left(\Omega - \omega_k\right)
\left(t - t'\right)}
+
\right.
\nonumber \\
\left.
+\hat{\sigma}\hat{\sigma}^{\dag}
\hat{\rho}_{at}\left(t'\right)
\bar{n}_k
e^{-i\left(\Omega - \omega_k\right)
\left(t - t'\right)} -
\hat{\sigma}^{\dag}
\hat{\rho}_{at}\left(t'\right)
\hat{\sigma}
\bar{n}_k
e^{i\left(\Omega - \omega_k\right)
\left(t - t'\right)}
-
\right.
\nonumber \\
-
\left.
\hat{\sigma}
\hat{\rho}_{at}\left(t'\right)
\hat{\sigma}^{\dag}
\left(\bar{n}_k + 1\right)
e^{-i\left(\Omega - \omega_k\right)
\left(t - t'\right)}
\right\}dt'
+ \mbox{э.с.}
\label{eqCh2_97}
\end{eqnarray}

Суммирование по $k$,  учитывая квазинепрерывный спектр состояний
диссипативной системы, можно заменить интегрированием
\eqref{eqCh1_modenumber_kvazy_contig} 
\[
\sum_{(k)} \rightarrow \int D\left(\omega\right)d \omega,
\]  
где $D\left(\omega\right)$ спектральная плотность состояний. Заменяем
$D\left(\omega\right)$, $\bar{n}\left(\omega\right)$,
$g^2\left(\omega\right)$ на $D$, $\bar{n}$, $g^2$ - их значения при
$\omega = \Omega$,  считая, что эти функции медленно   
зависят от частоты. Воспользуемся равенством (см.
интегральное представление дельта-функции
\eqref{eq:delta_from_integral})
\begin{eqnarray}
\int_{0}^\infty
e^{\pm\left(\Omega - \omega\right)
\left(t - t'\right)}d \omega = \left|\nu = \omega - \Omega\right| =
\nonumber \\
=
\int_{-\Omega}^\infty
e^{ - \nu
\left(t \mp t'\right)}d \nu \approx
\int_{-\infty}^\infty
e^{ - \nu
\left(t \mp t'\right)}d \nu = 
\nonumber \\
= 2 \pi \delta \left(t \mp t'\right) 
\label{eqCh2_98}
\end{eqnarray}
и проведем интегрирование по времени. Это приводит нас к уравнению 
\begin{eqnarray}
\dot{\hat{\rho}}_{at} = -\frac{1}{2}\gamma \left\{
\bar{n}\left[
\hat{\sigma}\hat{\sigma}^{\dag}\hat{\rho}_{at} -
\hat{\sigma}^{\dag}\hat{\rho}_{at}\hat{\sigma}
\right] +
\right.
\nonumber \\
+
\left.
\left(\bar{n} + 1\right)\left[
\hat{\sigma}^{\dag}\hat{\sigma}\hat{\rho}_{at} -
\hat{\sigma}\hat{\rho}_{at}\hat{\sigma}^{\dag}
\right]
\right\}
+ \mbox{э.с.}
\label{eqCh2_99}
\end{eqnarray}
где $\gamma =4 \pi D g^2$, $\bar{n}$ взяты при частоте моды $\Omega$.
Уравнение \eqref{eqCh2_99} имеет тот же вид, что и уравнение
\eqref{eqCh2_64}, полученное ранее. Уравнения 
полностью совпадают, если  принять $\gamma = \frac{\omega}{Q}$ и
сделать замену $\hat{\sigma} \rightarrow \hat{a}$, $\hat{\sigma}^{\dag}
\rightarrow \hat{a}^{\dag}$.  Это еще раз подтверждает, что окончательный
результат не зависит от природы резервуара.   

При температуре резервуара  $T = 0$  фотоны отсутствуют и $\bar{n} =
0$.  Положим, что в начальный момент атом возбужден 
\[
\hat{\rho}_{at}\left(t_0\right) = 1.
\]
Используя свойства \eqref{eqCh2_8} - \eqref{eqCh2_task1} операторов 
$\hat{\sigma}$ и $\hat{\sigma}^{\dag}$, получим   
\begin{eqnarray}
\dot{\rho}_{aa} = \bra{a}\dot{\hat{\rho}}_{at}\ket{a} =
-\frac{1}{2}\gamma
\bra{a}\hat{\sigma}^{\dag}\hat{\sigma}\hat{\rho}_{at}\ket{a}
+ 
\nonumber \\
+ \frac{1}{2}\gamma
\bra{a}\hat{\sigma}\hat{\rho}_{at}\hat{\sigma}^{\dag}\ket{a}
+ \mbox{к.с.} = 
\nonumber \\
= 
- \frac{1}{2}\gamma
\bra{a}\hat{\rho}_{at}\ket{a} + \mbox{к.с.} = 
- \gamma \rho_{aa}.
\label{eqCh2_101}
\end{eqnarray}
Здесь мы использовали соотношения
\(\hat{\sigma}^{\dag}\ket{a} = 0\),
\(\hat{\sigma}^{\dag}\ket{b} = \ket{a}\),
\(\hat{\sigma}\ket{a} = \ket{b}\),
\(\hat{\sigma}\ket{b} = 0\)
и сопряженные им. Окончательный результат 
\begin{equation}
\dot{\hat{\rho}}_{aa} = - \gamma \rho_{aa}.
\label{eqCh2_102}
\end{equation}
Это означает, что в результате спонтанной эмиссии вероятность
нахождения атома на верхнем уровне экспоненциально убывает. 
