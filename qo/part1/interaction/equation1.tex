%% -*- coding:utf-8 -*- 
\section{Уравнение для матрицы плотности поля в представлении чисел
  заполнения}
\label{ch2_7}
В представлении чисел заполнения (чисел фотонов) матричные элементы
определяются равенством
\[
\rho_{nm} = \bra{n}\hat{\rho}\ket{m}.
\]
Для получения нужного уравнения помножим
\eqref{eqCh2_rho_final2} слева на $\bra{n}$ и справа на  
$\ket{m}$.  Пользуясь
свойствами операторов $\hat{a}$ и $\hat{a}^{\dag}$,  получим систему
уравнений, связывающих между собой матричные элементы 
\begin{eqnarray}
\dot{\rho}_{nm} = - \frac{1}{2}
\left(
R_a\left(n + 1 + m + 1\right) + 
R_b\left(n + m\right)
\right)\rho_{nm} +
\nonumber \\
+ 
R_a\sqrt{nm}\rho_{n - 1, m - 1} +
R_b\sqrt{\left(n + 1\right)\left(m + 1\right)}\rho_{n + 1, m + 1}.
\label{eqCh2_task5}
\end{eqnarray}
Для диагональных элементов $n = m$,  следовательно
\begin{eqnarray}
\dot{\rho}_{nn} = - 
\left(
R_a\left(n + 1\right) + 
R_b\left(n\right)
\right)\rho_{nn} +
\nonumber \\
+ 
R_a n \rho_{n - 1, n - 1} +
R_b\left(n + 1\right)\rho_{n + 1, n + 1}.
\label{eqCh2_51}
\end{eqnarray}

\input ./part1/interaction/fig8.tex

Это уравнение можно рассматривать как баланс потоков вероятностей
чисел фотонов. Условно это изображено на \autoref{figPart1Ch2_8}. В
тепловом равновесии потоки должны быть равны. Отсюда, используя
принцип детального равновесия, получим равенства: 
\begin{eqnarray}
R_a n \rho_{n - 1, n - 1} = R_b n \rho_{nn},
\nonumber \\
R_a \left(n + 1\right) \rho_{n n} = R_b 
\left(n + 1\right) \rho_{n + 1, n + 1}
\label{eqCh2_52}
\end{eqnarray}
Из \eqref{eqCh2_52} имеем
\begin{equation}
\rho_{n + 1, n + 1} = \frac{R_a}{R_b}\rho_{nn} = 
e^{-\frac{\hbar \omega}{k_B T}}\rho_{nn},
\label{eqCh2_53}
\end{equation}
так как  
\(
\frac{R_a}{R_b} = 
\exp \left(-\frac{\hbar \omega}{k_B T}\right)
\)
(резервуар находится в равновесии при температуре
$T$). Используя последовательно \eqref{eqCh2_53}, начиная с  $n = 0$, 
получим:  
\begin{equation}
\rho_{nn} = \rho_{00} 
e^{-\frac{n \hbar \omega}{k_B T}} =
\left(1 - e^{-\frac{\hbar \omega}{k_B T}}\right) 
e^{-\frac{n \hbar \omega}{k_B T}},
\end{equation}
где $\rho_{00}$ найдено из условия $\sum_{(n)}\rho_{nn} = 1$.

Среднее число фотонов в моде, как и следовало ожидать, определяется
формулой Планка 
\begin{equation}
\bar{n} = \left<n\right> = 
\sum_{(n)} n \rho_{nn} = 
\frac{1}{ e^{\frac{\hbar \omega}{k_B T}} - 1}.
\end{equation}

Из всего сказанного следует, что со временем температура излучения
становится равной температуре атомного пучка (резервуара).  
Изменение среднего числа фотонов во времени получим из уравнения 
\eqref{eqCh2_51}
\begin{eqnarray}
\frac{d}{d t}\left<n\left(t\right)\right> = \sum_{(n)}n \dot{\rho} = 
\nonumber \\
= \sum_{(n)}\left(-R_a\left(n^2 + n\right)\rho_{nn} - R_b n^2
\rho_{nn} + \right.
\nonumber \\
+ \left.
R_b \left(n^2 + n\right) \rho_{n +1, n+ 1} +
R_a n^2 \rho_{n - 1, n - 1}
\right).
\end{eqnarray}
Заменим переменные суммирования  $m= n + 1$  в третьей сумме и  $m = n
- 1$  в четвертой сумме. Получим:
\begin{eqnarray}
\frac{d}{d t}\left<n\right> = 
-R_a \sum_{(n)}\left(n^2 + n\right)\rho_{nn}
 - R_b \sum_{(n)} n^2 \rho_{nn} +
\nonumber \\
+ R_b\sum_{(m)}\left(m^2 - m\right)\rho_{m,m} 
+ R_a\sum_{(m)}\left(m^2 +2 m + 1\right)\rho_{m,m} = 
\nonumber \\
= R_a \sum_{(m)}\left( m + 1\right)\rho_{m,m} - R_b\sum_{(m)} m
\rho_{m,m} =
\nonumber \\
= \left(R_a - R_b\right) \left<n\right> + R_a.
\label{eqCh2_57}
\end{eqnarray}
В равновесии
\[
\frac{d}{d t}\left<n\right> = 0,
\]  
и мы получаем прежнее соотношение
\begin{equation}
\left<n_{(\infty)}\right> = \bar{n} = 
\frac{R_a}{R_b - R_a} = \frac{1}{\frac{R_b}{R_a} - 1} = 
\frac{1}{e^{\frac{\hbar \omega}{k_B T}} - 1}.
\end{equation}

Уравнение \eqref{eqCh2_57} описывает изменение числа фотонов (энергии)
во времени в результате релаксации (взаимодействия с диссипативной
системой). Классическое уравнение, описывающее этот процесс имеет вид
\begin{equation}
\frac{d}{d t}\left<n\right> = 
- \frac{\omega}{Q}\left<n\right> + \frac{\omega}{Q} \left<n\right>_{\mbox{рав.}},
\nonumber
\end{equation}
где $\left<n\right>_{\mbox{рав.}} = \bar{n}_T$ равновесное значение
$\left<n\right>$ при температуре $T$, а $Q$ - добротность резонатора.
\rindex{Добротность}
Следовательно можно положить 
\begin{eqnarray}
R_b - R_a = \frac{\omega}{Q},
\nonumber \\
R_a = \bar{n}_T \frac{\omega}{Q},
\nonumber \\
R_b = \frac{\omega}{Q} \left(1 + \bar{n}_T\right),
\label{eqCh2_RabQw}
\end{eqnarray}
т. е. введенные нами величины $R_a$ и $R_b$ выражаются через
классическую величину $\frac{\omega}{Q}$, характеризующую потери в
резонаторе. 

Другая величина, интересующая нас - это среднее электромагнитное поле
\begin{eqnarray}
\bar{E}_{(f)} = E_0 \sin k z
Sp\left(\hat{\rho}\left(\hat{a}^{\dag} + \hat{a}\right)\right) = 
\nonumber \\
= E_0 \sin k z Sp\left(\hat{\rho}\hat{a}\right) + \mbox{к.с.} = 
\nonumber \\
= E_1 Sp\left(\hat{\rho}\hat{a}\right) + \mbox{к.с.} =
\left<E\right> + \mbox{к.с.}, 
\nonumber
\end{eqnarray}
где $\left<E\right>$ - аналитический сигнал классического
поля. 
Уравнение, которому удовлетворяет поле, может быть получено при
помощи уравнения движения матрицы плотности. Запишем уравнение для
статистического оператора поля моды  
\eqref{eqCh2_rho_final2} и
используя для $R_a$, $R_b$ их выражения через $Q$ и $\bar{n}_T$
  \eqref{eqCh2_RabQw}, имеем
\begin{eqnarray}
\dot{\hat{\rho}} =
- \frac{\omega}{2Q}\bar{n}_T
\left(\hat{a}\hat{a}^{\dag}\hat{\rho} - 
2 \hat{a}^{\dag}\hat{\rho}\hat{a} + \hat{\rho}\hat{a}\hat{a}^{\dag}
\right)
- 
\nonumber \\
- \frac{\omega}{2Q}\left(\bar{n}_T + 1\right)
\left(\hat{a}^{\dag}\hat{a}\hat{\rho} - 
2 \hat{a}\hat{\rho}\hat{a}^{\dag}
+ \hat{\rho}\hat{a}^{\dag}\hat{a}
\right)
\label{eqCh2_eq1_add1}
\end{eqnarray}

Поскольку 
\begin{equation}
\left<E\right> = E_1 Sp\left(\hat{\rho}\hat{a}\right), \quad
\frac{d \left<E\right>}{d t} = E_1 Sp\left(\frac{d \hat{\rho}}{dt}\hat{a}\right),
\label{eqCh2_eq1_add2}
\end{equation}
Следовательно, уравнение \eqref{eqCh2_eq1_add1} 
надо помножить на $E_1 \hat{a}$ и взять $Sp$. В результате получим:
\begin{eqnarray}
\dot{\left<E\right>} =
- \frac{\omega E_1}{2Q}\bar{n}_T
\left\{Sp\left(\hat{a}\hat{a}^{\dag}\hat{\rho}\hat{a} - 
2 \hat{a}^{\dag}\hat{\rho}\hat{a}\hat{a} +
\hat{\rho}\hat{a}\hat{a}^{\dag}\hat{a}\right) + 
\right.
\nonumber \\
+\left.
Sp\left(\hat{a}^{\dag}\hat{a}\hat{\rho}\hat{a} - 
2 \hat{a}\hat{\rho}\hat{a}^{\dag}\hat{a}
+ \hat{\rho}\hat{a}^{\dag}\hat{a}\hat{a}
\right)
\right\}
- 
\nonumber \\
- \frac{\omega E_1}{2Q}
Sp\left(\hat{a}^{\dag}\hat{a}\hat{\rho}\hat{a} - 
2 \hat{a}\hat{\rho}\hat{a}^{\dag}\hat{a}
+ \hat{\rho}\hat{a}^{\dag}\hat{a}\hat{a}
\right)
\label{eqCh2_eq1_add3}
\end{eqnarray}
Известно, что под знаком $Sp$ можно производить круговую перестановку
операторов. Применим это к \eqref{eqCh2_eq1_add3}.
Например
\begin{eqnarray}
Sp\left(\hat{a}\hat{a}^{\dag}\hat{\rho}\hat{a} - 
2 \hat{a}^{\dag}\hat{\rho}\hat{a}\hat{a} +
\hat{\rho}\hat{a}\hat{a}^{\dag}\hat{a}\right) = 
\nonumber \\
= Sp\left(\hat{a}\hat{a}^{\dag}\hat{\rho}\hat{a} - 
2 \hat{a}\hat{a}^{\dag}\hat{\rho}\hat{a} +
\hat{\rho}\hat{a}\hat{a}^{\dag}\hat{a}\right) = 
\nonumber \\
= Sp\left(\hat{a}\hat{a}^{\dag}\hat{\rho}\hat{a} - 
2 \hat{a}\hat{a}^{\dag}\hat{\rho}\hat{a} +
\hat{\rho}\hat{a}\left(\hat{a}\hat{a}^{\dag} - 1\right)\right) = 
\nonumber \\
Sp\left(\hat{a}\hat{a}^{\dag}\hat{\rho}\hat{a} - 
2 \hat{a}\hat{a}^{\dag}\hat{\rho}\hat{a} +
\left(\hat{a}\hat{a}^{\dag} - 1\right)\hat{\rho}\hat{a}\right) = 
- Sp\left(\hat{\rho}\hat{a}\right).
\label{eqCh2_eq1_add4}
\end{eqnarray}
Проделав то же самое со второй скобкой \eqref{eqCh2_eq1_add3}, получим
\begin{eqnarray}
Sp\left(\hat{a}^{\dag}\hat{a}\hat{\rho}\hat{a} - 
2 \hat{a}\hat{\rho}\hat{a}^{\dag}\hat{a}
+ \hat{\rho}\hat{a}^{\dag}\hat{a}\hat{a} \right) = 
\nonumber \\
= 
Sp\left(\hat{\rho}\hat{a}\hat{a}^{\dag}\hat{a} - 
2 \hat{\rho}\hat{a}^{\dag}\hat{a}\hat{a}
+ \hat{\rho}\hat{a}^{\dag}\hat{a}\hat{a} \right) = 
\nonumber \\
= 
Sp\left(\hat{\rho}\left(\hat{a}^{\dag}\hat{a} + 1\right)\hat{a} - 
2 \hat{\rho}\hat{a}^{\dag}\hat{a}\hat{a}
+ \hat{\rho}\hat{a}^{\dag}\hat{a}\hat{a} \right) = 
Sp\left(\hat{\rho}\hat{a}\right).
\label{eqCh2_eq1_add5}
\end{eqnarray}
Подставив \eqref{eqCh2_eq1_add4} и \eqref{eqCh2_eq1_add5} в 
\eqref{eqCh2_eq1_add3}, имеем
\begin{eqnarray}
\dot{\left<E\right>} =
- \frac{\omega E_1}{2Q}\bar{n}_T
\left\{Sp\left(\hat{\rho}\hat{a}\right) -
Sp\left(\hat{\rho}\hat{a}\right)\right\} -
\nonumber \\
- \frac{\omega E_1}{2Q}Sp\left(\hat{\rho}\hat{a}\right) = 
- \frac{\omega}{2Q}\left<E\right>.
\label{eqCh2_61}
\end{eqnarray}

Таким образом, мы связали параметры резервуара (параметры
пучка) с классической величиной $Q$ и со средним числом фотонов в моде 
при температуре резервуара. Это позволяет написать уравнения движения
матрицы плотности поля в общем виде, при этом конкретная модель
резервуара не имеет значения 
\begin{eqnarray}
\dot{\rho}_{nm} = - \frac{\omega}{2 Q}
\left(2 \bar{n}_T\left( n + m + 1\right) + n + m \right)\rho_{nm} +
\nonumber \\
+ \frac{\omega \bar{n}_T}{Q}\sqrt{nm}\rho_{n - 1, m - 1} +
\frac{\omega}{Q}\left(\bar{n}_T + 1\right)
\sqrt{\left(n + 1\right)\left(m + 1\right)}
\rho_{n + 1, m + 1}
\label{eqCh2_63}
\end{eqnarray}
В операторном виде это уравнение записывается так
\begin{eqnarray}
\dot{\rho} = - \frac{\omega}{2 Q}
\left\{
\bar{n}_T\left(\hat{a}\hat{a}^{\dag}\hat{\rho} - 
\hat{a}^{\dag}\hat{\rho}\hat{a}\right)
\right. +
\nonumber \\
+
\left .
\left(\bar{n}_T + 1\right)\left(\hat{a}^{\dag}\hat{a}\hat{\rho} - 
\hat{a}\hat{\rho}\hat{a}^{\dag}\right)
\right\} + \mbox{э. с.}
\label{eqCh2_64}
\end{eqnarray}
Запись уравнения \eqref{eqCh2_rho_final2}, \ref{eqCh2_64} в
представлении чисел заполнения является одной из многих. Часто удобно
пользоваться другими представлениями.  
