%% -*- coding:utf-8 -*- 
\section{Спонтанное излучение. \\
  Приближение Вайскопфа-Вигнера.}
Выше нами получена формула для скорости спонтанного излучения,
справедливая для достаточно малых времен. В то же время решая,
например, задачу о времени жизни возбужденного атома, нельзя
ограничиваться малыми временами. В этом случае применяется уже другой
метод, называемый приближением Вайскопфа-Вигнера.

Рассмотрим этот метод.Мы выяснили, что атом переходит из возбужденного
состояния в основное из-за взаимодействия со всеми модами
пространства, даже если они не возбуждены и находятся в вакуумном
состоянии. Гамильтониан взаимодействия в этом случае имеет вид
\begin{equation}
\hat{V}_I = \hbar \sum_{k} g_k \hat{\sigma}^{\dag}\hat{a}_k e^{i\left(
\omega_{ab} - \omega_k
\right)t} + \mbox{э. с.},
\label{eqCh2VaiskopfV}
\end{equation}
где $g_k$ константа взаимодействия с $k$-ой модой. От ранее
использованного нами гамильтониана \eqref{eqCh2_task22}
\eqref{eqCh2VaiskopfV} отличается суммированием по всем модам
пространства квантования. 

Допустим, что в начальный момент времени $t=0$ атом возбужден, а в
поле отсутствуют фотоны. Тогда начальное состояние имеет вид
\begin{equation}
\left|\psi\left(0\right)\right> =
 \left|a, \left\{0\right\}\right>,
\nonumber
\end{equation}
то есть атом возбужден, а все моды находятся в вакуумном состоянии. В
последующие времена из-за взаимодействия атом-поле, состояние системы
будет иметь вид
\begin{equation}
\left|\psi\left(t\right)\right> =
\sum_{k} C_{bk}\left(t\right) \ket{b, 1_k}
+
C_{a}\left(t\right) \left|a, \left\{0\right\}\right>,
\label{eqCh2Vaiskopf3}
\end{equation}
где $\ket{b, 1_k}$ состояние, когда атом находится в основном
состоянии, а в $k$-ой моде возбужден один фотон. \rindex{фотон}

Уравнение Шредингера в представлении взаимодействия имеет вид
\begin{equation}
\frac{d}{dt} \left|\psi\left(t\right)\right> =
- \frac{i}{\hbar} \hat{V}_I \left|\psi\left(t\right)\right>.
\label{eqCh2Vaiskopf4}
\end{equation}
Подставляя сюда \eqref{eqCh2Vaiskopf3} и умножая уравнение
\eqref{eqCh2Vaiskopf4} последовательно на
$\left<a, \left\{0\right\}\right|$ и $\bra{b, 1_k}$ получаем
систему уравнений для амплитуд вероятностей 
$C_{a}\left(t\right)$ и $C_{bk}\left(t\right)$.
\begin{eqnarray}
\dot{C}_{a}\left(t\right) = - i \sum_{k} g_k e^{i \left(\omega_{ab} - 
  \omega_k\right)t} C_{bk}\left(t\right),
\nonumber \\
\dot{C}_{bk}\left(t\right) = - i g_k e^{- i \left(\omega_{ab} -
  \omega_k\right)t} C_{a}\left(t\right).
\label{eqCh2Vaiskopf5}
\end{eqnarray}
Проинтегрируем второе уравнение \eqref{eqCh2Vaiskopf5}по времени от $0$ до
$t$. 
\begin{equation}
C_{bk}\left(t\right) = - i g_k \int_0^{t} e^{- i \left(\omega_{ab} -
  \omega_k\right)t'} C_{a}\left(t'\right) dt'.
\label{eqCh2Vaiskopf6}
\end{equation}
Подставив \eqref{eqCh2Vaiskopf6} в первое уравнение системы
\eqref{eqCh2Vaiskopf5}, получим следующее интегро-дифференциальное
уравнение
\begin{equation}
\dot{C}_{a}\left(t\right) = - \sum_{k} g_k^2 
\int_0^t
e^{i \left(\omega_{ab} - \omega_k\right)\left(t - t'\right)}  
C_{a}\left(t'\right) dt'.
\label{eqCh2Vaiskopf7}
\end{equation}

Сделаем теперь ряд упрощений (приближение Вайскопфа-Вигнера). Будем
считать распределение мод квазинепрерывным и заменим суммирование по
$k$ интегрированием, используя соотношение
\eqref{eqCh1_modenumber_1pre}:
\begin{equation}
d N = 2 \left(\frac{L}{2 \pi} \right)^3 k^2 d k d \Omega = 
2 \frac{V}{\left(2 \pi\right)^3}  k^2 d k d \Omega
\nonumber
\end{equation}
Тогда получим \eqref{eqCh1_modenumber_kvazy_contig}
\begin{eqnarray}
\sum_{k} g_k^2 
\int_0^t
e^{i \left(\omega_{ab} - \omega_k\right)\left(t - t'\right)}  
C_{a}\left(t'\right) dt' = 
\nonumber \\
= 2 \frac{V}{\left(2 \pi\right)^3} \int_{4\pi}d \Omega \int_0^{\infty}
g_k^2 k^2 dk  \int_0^t dt'
e^{i \left(\omega_{ab} - \omega_k\right)\left(t - t'\right)}  
C_{a}\left(t'\right).
\label{eqCh2Vaiskopf8pre}
\end{eqnarray}
В \eqref{eqCh2Vaiskopf8pre} 
\begin{equation}
g_k^2 = \frac{\omega_k\left|p\right|^2 sin^2 \theta}{4 \hbar
  \varepsilon_0 V},
\label{eqCh2VaiskopfGk}
\end{equation}
$\theta$ - угол между направлениями $\vec{k}$ и $\vec{p}$.
Выражение \eqref{eqCh2VaiskopfGk} получено с учетом усреднения по
поляризациям \eqref{eqCh2_PolyarMedian}.

Переходя в \eqref{eqCh2Vaiskopf8pre} от $k$ к $\omega$ с помощью соотношений 
\begin{equation}
k = \frac{\omega_k}{c}, \quad k^2 dk = \frac{\omega_k^2 d \omega_k}{c^3}
\nonumber
\end{equation}
и обозначая для удобства \(\omega_k = \omega\),
получим
\begin{equation}
\dot{C}_{a}\left(t\right) = - 
2 \frac{V}{\left(2 \pi c\right)^3} \int_{4\pi}d \Omega \int_0^{\infty}
g^2\left(\omega\right) \omega^2 d\omega  \int_0^t dt'
e^{i \left(\omega_{ab} - \omega\right)\left(t - t'\right)}  
C_{a}\left(t'\right).
\label{eqCh2Vaiskopf8}
\end{equation}

Для приближенного вычисления интеграла в \eqref{eqCh2Vaiskopf8}
дополнительно используем ряд упрощающих допущений.
Проинтегрируем \eqref{eqCh2Vaiskopf8} сперва по частоте (т. е. меняем
порядок интегрирования, считая что это возможно). Из структуры 
\eqref{eqCh2Vaiskopf8} видно, что главный вклад в интеграл по времени
дает область частот $\omega \approx \omega_{ab}$. По этой причине
можно принять 
\[
\omega^2 g^2\left(\omega\right) \approx 
\omega_{ab}^2 g^2\left(\omega_{ab}\right).
\]
В этом приближении интеграл по частоте будет иметь вид (см.
интегральное представление дельта-функции
\eqref{eq:delta_from_integral}):  
\begin{eqnarray}
\int_0^{\infty}d \omega e^{i\left(\omega_{ab} - \omega\right)\left(t -
  t'\right)}  = \left|\nu = \omega - \omega_{ab}\right| =
\int_{- \omega_{ab}}^{\infty}d \nu e^{-i \nu\left(t - t'\right)} \approx
\nonumber \\
\approx \int_{- \infty}^{\infty} d \nu e^{-i \nu\left(t - t'\right)} = 
\int_{- \infty}^{\infty} d \nu e^{i \nu\left(t' - t\right)} =
2 \pi \delta\left(t' - t\right).
\label{eqCh2Vaiskopf9}
\end{eqnarray}

Подставляя \eqref{eqCh2Vaiskopf9} в уравнение \eqref{eqCh2Vaiskopf8} и
интегрируя по времени, получаем
\begin{equation}
\dot{C}_{a}\left(t\right) = - 
2 \frac{V}{\left(2 \pi c\right)^3} \int_{4\pi}d \Omega 
g^2\left(\omega_{ab}\right) \omega_{ab}^2   
2 \pi C_{a}\left(t\right) = - \frac{\Gamma}{2} C_{a}\left(t\right).
\label{eqCh2Vaiskopf10}
\end{equation}
Подставив сюда выражение для $g_k^2$ \eqref{eqCh2VaiskopfGk} и
производя интегрирование по углам (по $d \Omega$), найдем выражение для
коэффициента затухания $\Gamma$:
\begin{equation}
\Gamma = \frac{\omega_{ab}^3 \left|p\right|^2}{3 \pi c^2 \hbar}
\sqrt{\frac{\mu_0}{\varepsilon_0}}. 
\label{eqCh2Vaiskopf11}
\end{equation}
Заметим, что выражение \eqref{eqCh2Vaiskopf11} совпадает с полученным
ранее другим способом \eqref{eqCh2_Wspon_final}.

Из \eqref{eqCh2Vaiskopf10} следует, что $\Gamma$ характеризует
скорость изменения вероятности $\left|C_{a}\right|^2$. Действительно
из \eqref{eqCh2Vaiskopf10} имеем:
\begin{eqnarray}
\dot{C}_{a}C_{a}^{*} = - \frac{\Gamma}{2}C_{a}C_{a}^{*},
\nonumber \\
\dot{C}_{a}^{*}C_{a} = - \frac{\Gamma}{2}C_{a}^{*}C_{a},
\nonumber
\end{eqnarray}
откуда имеем
\begin{equation}
\frac{d C_{a}C_{a}^{*}}{dt} = -\Gamma \left(C_{a}C_{a}^{*}\right),
\nonumber
\end{equation}
или иначе
\begin{equation}
\frac{d \left|C_{a}\right|^2}{dt} = -\Gamma \left|C_{a}\right|^2.
\nonumber
\end{equation}
Решение имеет вид (см. \autoref{fig:part1:vaickopf})
\[
\left|C_{a}\right|^2 = e^{- \Gamma t},
\]
если начальное значение $\left.\left|C_{a}\right|^2\right|_{t = 0} =
1$, т. е. атом в начале был возбужден.

\input ./part1/interaction/figvaickopf.tex 

\begin{remark}[Приближение Вайскопфа-Вигнера и осцилляции Раби]
Если мы сравним \autoref{fig:part1:rabi} и
\autoref{fig:part1:vaickopf} то сможем увидеть, что эти два графика
очень сильно отличаются несмотря на то что описывают вроде бы одну и
туже модельную систему. В связи с этим возникает вопрос когда мы можем
применять одномодовое приближение, и соответственно получать
осцилляции Раби (см. \autoref{fig:part1:rabi}), а в каких случаях
необходимо пользоваться многомодовым приближением типа
Вайскопфа-Вигнера (см. \autoref{fig:part1:vaickopf}). 

Формально мы можем считать что взаимодействие с каждой модой
электромагнитного поля дает свой вклад в результирующие вероятности,
вместе с тем в соответствии \eqref{eqCh2_prob_C_bn} изменение
вероятности (при малых $t$)
\[
\Delta \left|C_{a, n}\left(t\right)\right|^2 = -\Delta \left|C_{b, n +
  1}\left(t\right)\right|^2 \sim \left(n + 1\right).
\]
Т. о. те моды в которых число фотонов велико дает больший вклад, чем
те в которых оно мало. Следовательно, если у нас имеется мода с
большим числом фотонов $n \gg 1$, то влиянием вакуумных мод с числом
фотонов $n = 0$ в этом случае можно пренебречь. В противном случае,
особенно если все моды равноправны, необходимо учитывать все моды и
применять приближения типа Вайскопфа-Вигнера.
\end{remark}
