%% -*- coding:utf-8 -*- 
\section{Уравнение движения статистического оператора поля моды в
  представлении когерентных состояний}
В квантовой оптике и квантовой электронике часто удобно пользоваться
когерентными состояниями $\left|\alpha\right>$.  В этом случае
статистический оператор может быть записан в диагональном
представлении \eqref{eqCh1_rhorepresent}  
\begin{equation}
\hat{\rho}\left(t\right) = \int
P\left(\alpha, t\right)\left|\alpha\right>\left<\alpha\right| d^2 \alpha,
\label{eqCh2_65}
\end{equation}
Подставляя это выражение в уравнение движения статистического
оператора \eqref{eqCh2_rho_final2}, \ref{eqCh2_64}, получим: 
\begin{eqnarray}
\int \dot{P}\left(\alpha, t\right)\left|\alpha\right>\left<\alpha\right|
d^2 \alpha  = 
\nonumber \\
= -\frac{1}{2}\int P\left(\alpha, t\right)\left[R_a
\left(\hat{a}\hat{a}^{\dag}\left|\alpha\right>\left<\alpha\right|-\hat{a}^{+}\left|\alpha\right>\left<\alpha\right|\hat{a}\right)
\right. +
\nonumber \\
+
\left.
R_b
\left(\hat{a}^{\dag}\hat{a}\left|\alpha\right>\left<\alpha\right|-\hat{a}\left|\alpha\right>\left<\alpha\right|\hat{a}^{+}\right)
\right]d^2 \alpha +\mbox{э.с.}
\label{eqCh2_66}
\end{eqnarray}

Справедливы следующие равенства
\begin{eqnarray}
\hat{a}^{\dag}\left|\alpha\right>\left<\alpha\right| = 
\left(\frac{\partial}{\partial \alpha} +
\alpha^{*}\right)\left|\alpha\right>\left<\alpha\right|, 
\nonumber \\
\left|\alpha\right>\left<\alpha\right|\hat{a} = 
\left(\frac{\partial}{\partial \alpha^{*}} +
\alpha\right)\left|\alpha\right>\left<\alpha\right|, 
\label{eqCh2_67}
\end{eqnarray}

Эти равенства легко доказываются, если воспользоваться одной из форм
определения когерентного состояния
\[
\left|\alpha\right> = e^{\alpha \hat{a}^{\dag} -
  \frac{1}{2}\alpha\alpha^{*}} \left|0\right>.
\]
Отсюда имеем 
\[
\left|\alpha\right>\left<\alpha\right| = e^{\alpha \hat{a}^{\dag} -
  \frac{1}{2}\alpha\alpha^{*}} \left|0\right>
\left<0\right|e^{\alpha^{*} \hat{a} -
  \frac{1}{2}\alpha^{*}\alpha}.
\]
Дифференцируя это выражение по $\alpha$,  получим:
\begin{eqnarray}
\frac{\partial}{\partial \alpha}\left|\alpha\right>\left<\alpha\right|
= \left(\hat{a}^{\dag} -
\alpha^{*}\right) e^{\alpha \hat{a}^{\dag} -
  \frac{1}{2}\alpha\alpha^{*}} \left|0\right>
\left<0\right|e^{\alpha^{*} \hat{a} -
  \frac{1}{2}\alpha^{*}\alpha} = 
\nonumber \\
= \left(\hat{a}^{\dag} -
\alpha^{*}\right)\left|\alpha\right>\left<\alpha\right| 
\nonumber
\end{eqnarray}
следовательно,
\begin{equation}
\hat{a}^{\dag}\left|\alpha\right>\left<\alpha\right| = 
\left(\frac{\partial}{\partial \alpha} +
\alpha^{*}\right)\left|\alpha\right>\left<\alpha\right|.
\label{eqCh2_68}
\end{equation}
Таким же образом показывается справедливость второго равенства
\begin{eqnarray}
\frac{\partial}{\partial \alpha^{*}}\left|\alpha\right>\left<\alpha\right|
=  e^{\alpha \hat{a}^{\dag} -
  \frac{1}{2}\alpha\alpha^{*}} \left|0\right>
\left<0\right|e^{\alpha^{*} \hat{a} -
  \frac{1}{2}\alpha^{*}\alpha} 
\left(\hat{a} -
\alpha\right). 
\nonumber
\end{eqnarray}
Следовательно,
\begin{equation}
\left|\alpha\right>\left<\alpha\right|\hat{a} = 
\left(\frac{\partial}{\partial \alpha^{*}} +
\alpha\right)\left|\alpha\right>\left<\alpha\right|, 
\label{eqCh2_68a}
\end{equation}

Используя \eqref{eqCh2_67} и известные соотношения для операторов рождения и
уничтожения  $\hat{a}\left|\alpha\right> = \alpha\left|\alpha\right>$,
$\left<\alpha\right|\hat{a}^{\dag} = \alpha^{*}\left<\alpha\right|$,
приведем уравнение \eqref{eqCh2_66} к виду:   
\begin{eqnarray}
\int \dot{P}\left(\alpha,
t\right)\left|\alpha\right>\left<\alpha\right| d^2 \alpha  = 
\nonumber \\
= -\frac{1}{2}R_a\int
P\left(\alpha,t\right)
\left\{
-\left(
\frac{\partial^2}{\partial \alpha \partial \alpha^{*}} +
\alpha^{*}\frac{\partial}{\partial \alpha^{*}}
\right)
\left|\alpha\right>\left<\alpha\right| 
\right\}
d^2 \alpha - 
\nonumber \\
-\frac{1}{2}R_b\int P\left(\alpha,t\right)
\alpha \frac{\partial}{\partial \alpha}
\left|\alpha\right>\left<\alpha\right| 
d^2 \alpha + \mbox{э.с.}
\label{eqCh2_69}
\end{eqnarray}

Нам необходимо рассмотреть входящие в \eqref{eqCh2_69} интегралы,
преобразовав их при помощи интегрирования по частям.
В интегралах, входящих в \eqref{eqCh2_69} $d^2\alpha = dx dy$, если
$\alpha = x + i y$, $x = Re\,\alpha$, $x = Im\,\alpha$.
Следовательно для интегрирования по частям нужно подинтегральную
функцию выразить через $x$, $y$, произвести интегрирование по частям,
а, затем вернуться к прежним переменным. Но можно поступить
иначе. Формально считаем $\alpha$ и $\alpha^{*}$ независимыми
переменными и выражаем $dx dy$ через $d\alpha d\alpha^{*}$.
\[
d^2\alpha = dx dy = d\alpha d\alpha^{*} \left|J\right|,
\]
где $\left|J\right|$ якобиан перехода, который в нашем случае является
постоянным $\left|J\right| = \frac{1}{2}$. Теперь можно
проинтегрировать по частям, а в окончательном выражении перейти к
прежним обозначениям 
\[
d\alpha d\alpha^{*} \left|J\right| \rightarrow d^2\alpha.
\]
 Имеем  
\begin{eqnarray}
\int P \alpha \frac{\partial}{\partial \alpha}
\left|\alpha\right>\left<\alpha\right| 
d^2 \alpha = 
\nonumber \\
= \left.\alpha P \left|\alpha\right>\left<\alpha\right|
\right|_{-\infty}^{+\infty} - 
\int \frac{\partial}{\partial \alpha}\left(\alpha P\right)
\left|\alpha\right>\left<\alpha\right|d^2 \alpha = 
\nonumber \\
= 
- \int \frac{\partial}{\partial \alpha}\left(\alpha P\right)
\left|\alpha\right>\left<\alpha\right|d^2 \alpha.
\label{eqCh2_70}
\end{eqnarray}

При этом мы полагали, что $\alpha P\left(\alpha, t\right) \rightarrow
0$, при $\left|\alpha\right| \rightarrow \infty$.  Интеграл со второй
производной после двойного интегрирования по частям преобразуется к
виду: 
\begin{eqnarray}
\int P \frac{\partial^2}{\partial \alpha \partial \alpha^{*}}
\left|\alpha\right>\left<\alpha\right| 
d^2 \alpha = 
\nonumber \\
= \int \left(\frac{\partial^2}{\partial \alpha \partial \alpha^{*}}
P\right) \left|\alpha\right>\left<\alpha\right|  
d^2 \alpha
\label{eqCh2_71}
\end{eqnarray}
Подставляя все это в равенство \eqref{eqCh2_69}, получим:
\begin{eqnarray}
\int \dot{P} \left(\alpha, t\right) 
\left|\alpha\right>\left<\alpha\right| 
d^2 \alpha = 
\nonumber \\
= - \int \left\{
\frac{1}{2}\left(R_a - R_b\right)
\left[
\frac{\partial}{\partial \alpha}
\left(P \alpha\right) + c.c.
\right]
\right.
-
\nonumber \\
- \left.
R_a \frac{\partial^2 P}{\partial \alpha \partial \alpha^{*}}
\right\}
\left|\alpha\right>\left<\alpha\right| 
d^2 \alpha
\label{eqCh2_72}
\end{eqnarray}

Это равенство удовлетворится, если множитель при  
$\left|\alpha\right>\left<\alpha\right|$  положить равным 
нулю. Таким образом, получим 
\begin{equation}
\dot{P}\left(\alpha, t\right) = 
-\frac{1}{2}\left(R_a - R_b\right)
\left(
\frac{\partial}{\partial \alpha}\left(\alpha P\right) + c.c.
\right)
+ R_a
\frac{\partial^2 P}{\partial \alpha \partial \alpha^{*}}
\label{eqCh2_73}
\end{equation}
Это уравнение типа уравнения Фоккера-Планка, которое используется при
решении классических задач статистической физики, например, задач о
броуновском движении и других аналогичных задач. Решив
\eqref{eqCh2_73} при соответствующих начальных и граничных условиях,
легко при помощи \eqref{eqCh2_65} написать выражение для
статистического оператора 
\[
\hat{\rho}\left(t\right) = 
\int P\left(\alpha, t\right)
\left|\alpha\right>\left<\alpha\right| 
d^2 \alpha
\]
и с помощью \eqref{eqCh1_middleO} и \eqref{eqCh1_113} вычислять
средние значения наблюдаемых величин, связанных с рассматриваемой
системой   
\[
\left<O\right>= \int P\left(\alpha, t\right) \left<\alpha\right|\hat{O}\left|\alpha\right>d^2
\alpha =
\int  P\left(\alpha, t\right)O^{\left(n\right)}\left(\alpha, \alpha^{*}\right)d^2\alpha,
\]
где $O^{\left(n\right)}$ нормальное представление оператора $\hat{O}$ \eqref{eqCh1_normalO}.
Коэффициенты  $R_a$ и  $R_b$,  входящие в \eqref{eqCh2_73}, можно, как
и прежде \eqref{eqCh2_RabQw}, выразить через  $Q$  и $\bar{n}_T$,
тогда уравнение примет вид 
\begin{equation}
\frac{\partial}{\partial t}P\left(\alpha, t\right) = 
\frac{1}{2}\frac{\omega}{Q}\left[
\frac{\partial}{\partial \alpha}\left(
\alpha P\left(\alpha, t\right)
\right) + \mbox{к. с.}
\right]
+
\frac{\omega \bar{n}_T}{Q} \frac{\partial^2  P\left(\alpha,
  t\right)}{\partial \alpha \partial \alpha^{*}}.
\label{eqCh2_74}
\end{equation}

В качестве примера вычислим среднее значение оператора положительно
частотной части электрического поля $E_1\hat{a}$.   
\begin{eqnarray}
\frac{d \left<E_1\hat{a}\right>}{d t} = 
E_1\int \alpha \dot{P}d^2 \alpha =
\nonumber \\
=E_1 \frac{1}{2}\frac{\omega}{Q}
\int \left[
\frac{\partial}{\partial \alpha}\left(\alpha P\right) + \mbox{к. с.}
\right]\alpha d^2 \alpha +
\nonumber \\
+E_1\frac{\omega \bar{n}_T}{Q}\int \alpha
\frac{\partial^2 P}{\partial \alpha \partial \alpha^{*}}
d^2 \alpha
\label{eqCh2_75}
\end{eqnarray}
Интегрируя второй член по $\alpha^{*}$ получим
\begin{eqnarray}
\int\alpha\frac{\partial^2 P}{\partial \alpha \partial \alpha^{*}}
d^2 \alpha = 
\left.\frac{\partial P}{\partial
  \alpha}\alpha\right|_{-\infty}^{\infty} - 
\nonumber \\
- \int \frac{\partial P}{\partial \alpha} \alpha
\frac{\partial \alpha}{\partial \alpha^{*}} = 0,
\nonumber
\end{eqnarray}
т. к. $\frac{\partial P}{\partial \alpha}$ предполагается достаточно
быстро стремящимся к 0 на бесконечности, а 
$\frac{\partial \alpha}{\partial \alpha^{*}} = 0$, т. к. 
$\alpha$ и $\alpha^{*}$ - независимые переменные.
Первое слагаемое можно расписать следующим образом
\begin{eqnarray}
\int \alpha \left(
\frac{\partial}{\partial \alpha}\left(\alpha P\right) + 
\frac{\partial}{\partial \alpha^{*}}\left(\alpha^{*} P\right)
\right) d^2 \alpha = 
\nonumber \\
= - \int \alpha P \frac{\partial \alpha}{\partial \alpha} d^2 \alpha -
\int \alpha^{*} P \frac{\partial \alpha}{\partial \alpha^{*}} d^2 \alpha  
= - \int \alpha P  d^2 \alpha
= - \left<\hat{a}\right>.
\nonumber
\end{eqnarray}
Таким образом окончательно имеем: 
\begin{equation}
\frac{d \left<E_1\hat{a}\right>}{d t} = 
-\frac{\omega}{2 Q}\left<E_1\hat{a}\right>
\label{eqCh2_76}
\end{equation}
- результат, который мы получили ранее другим путем. Уравнение
\eqref{eqCh2_74} можно записать в различных системах координат,
например, в полярной $\alpha = r e^{i \theta}$, $\alpha^{*} = r e^{- i
  \theta}$.  Проделав соответствующие преобразования, получим 
\begin{eqnarray}
\frac{\partial}{\partial t}P\left(r, \theta, t\right) = 
\frac{1}{2}\frac{\omega}{Q}
\frac{\partial}{\partial r}
\left(r^2 P\left(r, \theta, t\right)\right) +
\nonumber \\
+ \frac{1}{2}\frac{\omega}{Q}
\frac{\bar{n}_T}{2 r^2}
\left(
r \frac{\partial}{\partial r} r \frac{\partial}{\partial r} +
\frac{\partial^2}{\partial \theta^2}
\right)
P\left(r, \theta, t\right).
\label{eqCh2_77}
\end{eqnarray}
Если воспользоваться координатами  $x$ и $y$, 
$\alpha = x + i y$, $\alpha^{*} = x - i y$, то есть 
$x = \frac{1}{2}\left(\alpha + \alpha^{*}\right)$,  
$y = - \frac{i}{2}\left(\alpha - \alpha^{*}\right)$,  можно уравнение
представить в виде 
\begin{eqnarray}
\frac{\partial}{\partial t}P\left(x, y, t\right) = 
\frac{1}{2}\frac{\omega}{Q}
\left(
\frac{\partial}{\partial x} x +
\frac{\partial}{\partial y} y
\right)
P\left(x, y, t\right) +
\nonumber \\
+
\frac{1}{4}
\frac{\omega}{Q}\bar{n}_T
\left(
\frac{\partial^2}{\partial x^2} +
\frac{\partial^2}{\partial y^2}
\right)
P\left(x, y, t\right).
\label{eqCh2_77a}
\end{eqnarray}

При решении конкретной задачи удобно воспользоваться уравнением,
записанным либо в прямоугольной, либо в полярной системе координат.   
