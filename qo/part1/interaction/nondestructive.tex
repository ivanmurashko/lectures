%% -*- coding:utf-8 -*- 
\section{Интерферометр Рамси. Квантовые неразрушающие измерения}

Наблюдение фотона \rindex{фотон!неразрушающие измерения} обычно предполагает его разрушение, например в
при использовании фотодетектора энергия фотона преобразуется в
электрический сигнал при этом сам фотон исчезает.

Интересный эксперимент \cite{Nogues1999Nv400p239} использующий
осцилляции Раби продемонстрировал возможность наблюдения фотона без
его разрушения. При этом использовался интерферометр Рамси (см
\autoref{figPart1RamsiInterferometer})

\input ./part1/interaction/figramsi.tex

\input ./part1/interaction/figramsia.tex

Схема работы следующая. Используются атомы со структурой уровней
изображенной на \autoref{figPart1RamsiAtom}. Частота перехода
$\omega_{ab}$ между уровнями $a$ и $b$ совпадает с частотой излучения
в зонах $R_1$ и $R_2$ интерферометра  (см
\autoref{figPart1RamsiInterferometer}). Зона $C$ содержит исследуемое
электромагнитное поле для которого мы хотим ответить на вопрос есть в
нем один фотон или нет, т. е. находится оно в состоянии
$\ket{1}$ или в вакуумном состоянии
$\ket{0}$. Исследуемое поле в зоне $C$ имеет частоту
совпадающую с частотой перехода $\omega_{ac}$ между уровнями $a$ и
$c$.  

Размер зон $R_{1,2}$ подбирается таким образом, чтобы время
взаимодействия $t_R$ атома с электромагнитным полем в них соответствовало
следующему соотношению
\begin{equation}
  \omega_R^{(R)} t_R = \frac{\pi}{2},
  \nonumber
\end{equation}
где $\omega_R^{(R)}$ частота Раби, соответствующая частоте перехода
$\omega_{ab}$.
Время взаимодействия $t_C$ в зоне $C$ определяется выражением
\begin{equation}
  \omega_R^{(C)} t_C = 2 \pi,
  \nonumber
\end{equation}
где $\omega_R^{(C)}$ частота Раби, соответствующая частоте перехода
$\omega_{ac}$.

Для взаимодействия в зонах $R_{1,2}$ справедливы следующие
соотношения \eqref{eqPart1RabiAbsorbtion}, \ref{eqPart1RabiEmission}:
\begin{eqnarray}
  \ket{a} \rightarrow \frac{1}{\sqrt{2}}\left(
  \ket{a} - i \ket{b}  
  \right),
  \nonumber \\
  \ket{b} \rightarrow \frac{1}{\sqrt{2}}\left(
  -i \ket{a} + \ket{b}  
  \right).
  \label{eqPart1RamsiRzone}
\end{eqnarray}
Выражение \eqref{eqPart1RamsiRzone} может быть переписано в матричном
виде $\left|\psi\right> \rightarrow R \left|\psi\right>$,
где
\[
R = \frac{1}{\sqrt{2}} \left(
\begin{array} {cc}
1 & -i
\\
-i & 1 
\end{array}
\right).
\]

Зона $C$ никак не влияет на состояние $\ket{b}$, для состояния
$\ket{a}$ в случае вакуумного состояния $\ket{0}$
(отсутствия фотона)
\begin{eqnarray}
  \ket{a} \rightarrow \ket{a},
  \nonumber
\end{eqnarray}
т.е. в случае отсутствия фотона состояние атома не
меняется. Действительно если рассматривать двухуровневую систему,
образованную состояниями $\ket{a}$ и $\ket{c}$, то
$\ket{a}\ket{0}$ будет являться минимально возможным
энергетическим состоянием, из которого система (двухуровневый атом и
электромагнитное поле) не сможет перейти ни в
какое другое состояние.
В этом случае для $C$ имеем
\[
C_0 = \left(
\begin{array} {cc}
1 & 0
\\
0 & 1 
\end{array}
\right).
\]

Для случая присутствия фотона из \eqref{eqPart1RabiAbsorbtion}
\begin{eqnarray}
  \ket{a} \rightarrow -\ket{a},
  \nonumber
\end{eqnarray}
и в этом случае $C$ имеет вид
\[
C_1 = \left(
\begin{array} {cc}
-1 & 0
\\
0 & 1 
\end{array}
\right).
\]

Таким образом если изначально атом испускаемый источником $S$
находится в состоянии $\ket{a} = \left(
\begin{array} {c}
1
\\
0
\end{array}
\right)$, то в случае отсутствия фотона в зоне $C$ на детекторе $D$ мы
получим атом в состоянии
\begin{eqnarray}
  R_2 C_0 R_1 \ket{a} =
  \nonumber \\
  =
  \frac{1}{2}
  \left(
  \begin{array} {cc}
    1 & -i
    \\
    -i & 1 
  \end{array}
  \right)
  \left(
  \begin{array} {cc}
    1 & 0
    \\
    0 & 1 
  \end{array}
  \right)
  \left(
  \begin{array} {cc}
    1 & -i
    \\
    -i & 1 
  \end{array}
  \right)
  \left(
  \begin{array} {c}
    1
    \\
    0
  \end{array}
  \right) =
  \nonumber \\
  =
  %% >> R=[1,-i;-i,1];
  %% >> C0=[1,0;0,1];
  %% >> a=[1;0];
  %% >> 1/2*R*C0*R*a
  %% ans =
  %% 0 + 0i
  %% 0 - 1i
  %% >>
  -i 
  \left(
  \begin{array} {c}
    0
    \\
    1
  \end{array}
  \right) =
  -i \ket{b},
  \nonumber
\end{eqnarray}
т.е. атом будет наблюдаться в невозбужденном состоянии
$\ket{b}$.

Для случая присутствия одного фотона в зоне $C$ имеем
\begin{eqnarray}
  R_2 C_1 R_1 \ket{a} =
  \nonumber \\
  =
  \frac{1}{2}
  \left(
  \begin{array} {cc}
    1 & -i
    \\
    -i & 1 
  \end{array}
  \right)
  \left(
  \begin{array} {cc}
    -1 & 0
    \\
    0 & 1 
  \end{array}
  \right)
  \left(
  \begin{array} {cc}
    1 & -i
    \\
    -i & 1 
  \end{array}
  \right)
  \left(
  \begin{array} {c}
    1
    \\
    0
  \end{array}
  \right) =
  \nonumber \\
  =
  %% >> C1=[-1,0;0,1];
  %% >> R=[1,-i;-i,1];
  %% >> a=[1;0];
  %% >> 1/2*R*C1*R*a
  %% ans =
  %% -1
  %% 0
  %% >> 
  - 
  \left(
  \begin{array} {c}
    1
    \\
    0
  \end{array}
  \right) =
  - \ket{a},
  \nonumber
\end{eqnarray}
т.е. атом будет наблюдаться в возбужденном состоянии
$\ket{a}$. Т. о. можно сделать вывод о присутствии фотона в
зоне $C$ без разрушающего воздействия на него.



