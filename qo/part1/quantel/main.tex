%% -*- coding:utf-8 -*- 
\chapter{Квантование электромагнитного поля} 
\label{chQuantel}

Квантовая электродинамика служит основой квантовой оптики. Ниже
излагаются положения квантовой теории электромагнитного поля,
необходимые для изучения и понимания квантовой оптики.

\input ./part1/quantel/separation.tex
\input ./part1/quantel/hamilton.tex
\input ./part1/quantel/quantemf.tex
\input ./part1/quantel/freefield.tex
\input ./part1/quantel/pho.tex
\input ./part1/quantel/hamiltonform.tex
\input ./part1/quantel/quantemfsep.tex
\input ./part1/quantel/propertyaa.tex
\input ./part1/quantel/defemf.tex
\input ./part1/quantel/multimode.tex
\input ./part1/quantel/cohmode.tex
\input ./part1/quantel/miscmode.tex
\input ./part1/quantel/represention.tex

\section{Упражнения}
\begin{enumerate}
\item Доказать ортогональность полей собственных колебаний
  (\ref{eqCh1_task1}).  
\item Доказать для плоских волн равенства (\ref{eqCh1_task2}).
\item Исходя из равенства (\ref{eqCh1_task3_1}), получить выражение
  (\ref{eqCh1_task3_2}) для оператора импульса квантованного
  электромагнитного поля. 
\item Доказать формулу Бейкера-Хаусдорфа (\ref{eqPart1Ch1_BeikerHausdorf})
%% \item Показать, что равенство (\ref{eqCh1_task4}) соответствует оператору плотности для
%% теплового возбуждения фотонов в моду.
\end{enumerate}

