%% -*- coding:utf-8 -*-
\section{Упражнения}
\begin{enumerate}
\item Доказать ортогональность полей собственных колебаний
  \eqref{eqCh1_task1}.  
\item Доказать для плоских волн равенства \eqref{eqCh1_task2}.
\item Исходя из равенства \eqref{eqCh1_task3_1}, получить выражение
  \eqref{eqCh1_task3_2} для оператора импульса квантованного
  электромагнитного поля. 
\item Доказать формулу Бейкера-Хаусдорфа \eqref{eqPart1Ch1_BeikerHausdorf}
%% \item Показать, что равенство \eqref{eqCh1_task4} соответствует оператору плотности для
%% теплового возбуждения фотонов в моду.
\item Сколько мод \label{qQuantelNumberMods} электромагнитного поля с
  длиной волны $\lambda \ge 500 \mbox{нм}$ находятся в кубе
  квантования со стороной $L=1 \mbox{мм}$
  \cite{courseIntroQuantumOpticsCoursera} 
\end{enumerate}
