%% -*- coding:utf-8 -*- 
\section{Квантовое состояние электромагнитного поля с определенной
  энергией (с определенным числом фотонов)}
Сначала рассмотрим одну моду электромагнитного поля (простой
гармонический осциллятор). Вектор состояния с определенной энергией
$\left|E_n\right>$ удовлетворяет следующему уравнению
\begin{equation}
\hat{\mathcal{H}} \left|E_n\right> = E_n \left|E_n\right>
\end{equation}
Здесь и далее будем пользоваться формализмом Дирака (см. прил.
\ref{AddDirac}). Используя соотношение (\ref{eqCh1_mainpropertyaa}),
получим 
\begin{equation}
\hat{\mathcal{H}} \hat{a}\left|E_n\right> = 
\left(\hat{a}\hat{\mathcal{H}} -
\hbar\omega\hat{a}\right)\left|E_n\right> =
\left(E_n - \hbar \omega\right)\hat{a}\left|E_n\right>
\end{equation}
т. е. $\hat{a}\left|E_n\right>$ также является вектором состояния с
энергией $E_n - \hbar \omega$.  Отсюда следует, что
\[
\hat{a}\left|E_n\right> = \left|E_n - \hbar \omega \right>
\] 
Следовательно, оператор $\hat{a}$ понижает энергию состояния на $\hbar
\omega$,  где $\omega$ - частота моды (осциллятора). Часто $\hat{a}$
называют понижающим оператором, или оператором уничтожения. Наименьшая
энергия должна быть положительной, и понижение энергии не может
продолжаться неограниченно. Для произвольного вектора состояния
ожидаемое значение энергии 
\begin{equation}
\left< \Phi \right| \hbar \omega \left({a}^{+} {a}  +
\frac{1}{2}\right)\left| \Phi \right> = 
\hbar \omega \left< \Phi' \right. \left| \Phi' \right> + \frac{1}{2}
\hbar \omega,
\end{equation}
где  $\hat{a} \left| \Phi \right> = \left| \Phi' \right>$,  
$\left< \Phi \right| \hat{a}^{+}  = \left< \Phi' \right|$.  Поскольку
норма вектора состояния должна быть положительной, 
наименьшее значение энергии будет при  
\(
\left< \Phi' \right. \left| \Phi' \right> = 0.
\)
Это означает, что $\hat{a}\left|0\right> = 0$,  где
$\left|\Phi\right> = \left|0\right>$ -  вектор состояния с наименьшей
энергией. Наименьшая энергия  
\begin{equation}
E_0 = \frac{\hbar \omega}{2}
\end{equation}
называется энергией нулевых колебаний. Для проверки этого утверждения
можно написать
\begin{eqnarray}
\hat{\mathcal{H}} \left|0\right> = 
\hbar \omega \left(\hat{a}^{+} \hat{a} +
\frac{1}{2}\right) \left|0\right> = 
\nonumber \\
= 
\hbar \omega \hat{a}^{+} \hat{a} \left|0\right> +
\frac{\hbar \omega}{2}\left|0\right> =
\nonumber \\
= \frac{\hbar \omega}{2} \left|0\right> = 
E_0 \left|0\right>.
\label{eqProper0state}
\end{eqnarray}


С помощью (\ref{eqCh1_mainpropertyaa}) можно получить
\begin{eqnarray}
\hat{\mathcal{H}} \hat{a}^{+}\left|0\right> = 
\hat{\mathcal{H}} \left|1\right> =
\left(\hat{a}^{+} \hat{\mathcal{H}} + \hbar \omega \hat{a}^{+} \right)
\left|0\right> = 
\nonumber \\
= \hbar \omega \left(1 + \frac{1}{2}\right)
\hat{a}^{+} \left|0\right> = 
\hbar \omega \left(1 + \frac{1}{2}\right)
\left|1\right>
\end{eqnarray}
где через $\left|1\right> = \hat{a}^{+} \left|0\right>$ обозначено
состояние с энергией $\hbar \omega \left(1 + \frac{1}{2}\right)$.

По индукции имеем
\begin{equation}
\hat{\mathcal{H}} \left(\hat{a}^{+}\right)^n\left|0\right> = 
\hat{\mathcal{H}} \left|n\right> 
= \hbar \omega \left(n + \frac{1}{2}\right)
\left(\hat{a}^{+}\right)^n\left|0\right> = 
\hbar \omega \left(n + \frac{1}{2}\right)
\left|n\right>
\label{eqCh1_aplusinduction}
\end{equation}
где
$\left|n\right> = \left(\hat{a}^{+}\right)^n\left|0\right>$   
без нормировки, которую проведем далее -  состояние с энергией  
$\hbar \omega \left(n + \frac{1}{2}\right)$,  $n$  -  целое
положительное число.

Видим, что оператор  $\hat{a}^{+}$  повышает энергию состояния на
$\hbar \omega$.  Его можно рассматривать как оператор рождения частицы
- фотона с энергией  $\hbar \omega$.  О фотоне как о частице лучше
говорить в случае разложения поля по плоским волнам. Тогда это будет
частица с энергией $\hbar \omega$ и импульсом $\hbar \vec{k}$,  как
это следует из (\ref{eqCh1_task3_2}). 
  
Соотношения 
$\hat{a} \left|n\right> = \left|n - 1\right>$
и
$\hat{a}^{+} \left|n\right> = \left|n + 1\right>$
определяют ненормированные векторы состояния. Определим нормирующий
множитель. Предположим, что  
$\hat{a} \left|n\right> = S_n \left|n - 1\right>$,  где 
$\left|n\right>$ и $\left|n - 1\right>$ нормированы к  1,  а $S_n$
является нормирующим множителем. Отсюда получим 
\[
S_n^2\left<n - 1\right.\left|n - 1\right> =
\left<n\right|\hat{a}^{+}\hat{a}\left|n\right> = 
n  \left<n\right.\left|n\right>
\]
т.к. оператор    
$\hat{a}^{+}\hat{a} = \hat{n}$
является оператором числа фотонов, собственным числом
которого является число фотонов. Это видно из формулы
(\ref{eqCh1_aplusinduction}). Действительно, из равенства
\[
\hat{\mathcal{H}} \left|n\right> =
\hbar \omega \left(
\hat{a}^{+}\hat{a} + \frac{1}{2}
\right)
\left|n\right> = 
\hbar \omega \left(n + \frac{1}{2}\right)
\left|n\right>,
\]
получаем:
\[
\hat{n}\left|n\right> = \hat{a}^{+}\hat{a} \left|n\right> = n
\left|n\right>. 
\]
Из условия нормировки следует: $\left<n\right.\left|n\right> = 1$   и
$S_n^2 = n$,  откуда $S_n = \sqrt{n}$ и, следовательно, имеем: 
\begin{equation}
\hat{a}\left|n\right> = \sqrt{n}\left|n - 1\right>
\end{equation}
По аналогии, с помощью коммутационных соотношений
(\ref{eqCh1_aacomutation}), а также \ref{eqAddDirac_operator_property1} и
  \ref{eqAddDirac_operator_property2} из приложения \ref{AddDirac},
  получаем 
\begin{eqnarray}
\hat{a}^{+}\left|n\right> = S_{n+1}\left|n + 1\right>,
\quad 
\left<n\right|\hat{a} = S_{n + 1}\left<n + 1\right|,
\nonumber \\
\left<n\right|\hat{a}\hat{a}^{+}\left|n\right> = S_{n+1}^2
\left<n + 1\right|\left.n + 1\right> = 
\left<n\right|\hat{a}^{+}\hat{a} + 1\left|n\right> = 
\left(n + 1\right)\left<n\right.\left|n\right>,
\nonumber \\
S_{n+1}^2 = n + 1.
\end{eqnarray}
Следовательно, имеем равенство
\begin{equation}
\hat{a}^{+}\left|n\right> = \sqrt{n + 1}\left|n + 1\right>,
\end{equation}

Собственные состояния оператора числа фотонов $\hat{n}$ являются ортонормированными. 
Действительно из того факта что оператор $\hat{n}$ является эрмитовым:
\[
\hat{n}^{+} = \left(\hat{a}^{+}\hat{a}\right)^{+} = 
\hat{a}^{+} \left(\hat{a}^{+}\right)^{+} = 
\hat{a}^{+}\hat{a} = \hat{n}
\]
следует (см. прил. \ref{AddDirac}), что собственные функции этого оператора,
соответствующие разным собственным числам, ортогональны, т. е.
\begin{equation}
\left<n\right|\left.n'\right> = 0, \mbox{ если } n \ne n'.
\label{eqOrtoN}
\end{equation}

Дадим сводку соотношений, в которые входят операторы $\hat{a}$ и $\hat{a}^{+}$:
\begin{eqnarray}
\hat{\mathcal{H}} = \hbar \omega \left(\hat{a}^{+}\hat{a} +
\frac{1}{2} \right),
\quad
\hat{a}\left|0\right> = 0,
\quad
\hat{a}^{+}\hat{a}\left|n\right> = \hat{n}\left|n\right>,
\nonumber \\
\left[\hat{a}, \hat{a}^{+}\right] = \hat{a} \hat{a}^{+} - \hat{a}^{+}
\hat{a} = 1,
\quad
\hat{a}\left|n\right> = \sqrt{n}\left|n - 1\right>
\nonumber \\
\hat{\mathcal{H}}\left|n\right> = \hbar \omega \left(\hat{a}^{+}\hat{a} +
\frac{1}{2} \right)\left|n\right>,
\nonumber \\
\hat{a}^{+}\left|n\right> = \sqrt{n + 1}\left|n + 1\right>,
\quad
\left|n\right> = \frac{1}{\sqrt{n!}}\left(\hat{a}^{+}\right)^n\left|0\right>
\end{eqnarray}
и сопряженные равенства
\begin{eqnarray}
\left<0\right|\hat{a}^{+} = 0,
\quad
\left<n\right|\hat{a} = \sqrt{n + 1}\left<n + 1\right|
\nonumber \\
\left<n\right|\hat{a}^{+} = \sqrt{n}\left<n - 1\right|,
\quad
\left<n\right| =  \frac{1}{\sqrt{n!}} \left<0\right|\left(\hat{a}^{+}\right)^n.
\end{eqnarray}

Для простейшей модели резонатора имеем
\[
\hat{E}\left(z, t\right) = E_1\left( \hat{a} +
\hat{a}^{+}\right) \sin k_n z
\]
где $E_1 = \sqrt{\frac{\hbar \omega}{\varepsilon_0 V}}$  поле,
соответствующее одному фотону в моде.  

Рассмотрим некоторые свойства энергетических состояний, т.е. состояний
с определенным числом фотонов. Покажем, что среднее значение
электрического поля в этом состоянии равно нулю: 
\begin{eqnarray}
\left<n\right|\hat{E}\left|n\right> = 
E_1 \sin k_n z \left( \left<n\right|\hat{a}\left|n\right> +
\left<n\right|\hat{a}^{+}\left|n\right>\right) =
\nonumber \\
= E_1 \sin k_n z \left( \left<n\right|\left.n - 1\right> \sqrt{n} +
\left<n\right|\left.n + 1\right> \sqrt{n + 1}
\right) = 0
\label{eqCh1_E_middle}
\end{eqnarray}
что следует из ортогональности состояний (\ref{eqOrtoN})
$\left<n\right|\left.n'\right> = 0$.

Среднее значение оператора квадрата электрического поля отлично от нуля:
\begin{eqnarray}
\left<n\right|\hat{E}^2\left|n\right> = 
E_1^2 \sin^2 k_n z \left<n\right|
\left(
\hat{a}^{+} \hat{a}^{+} + \hat{a} \hat{a}^{+} + \hat{a}^{+} \hat{a} +
\hat{a} \hat{a}
\right)
\left|n\right> =
\nonumber \\
= 2 E_1^2 \sin^2 k_n z \left( n + \frac{1}{2}
\right).
\label{eqCh1_E2_middle}
\end{eqnarray}

Обычно рассматриваемые в квантовой оптике поля не находятся в
стационарном состоянии с определенной энергией (с определенным числом
фотонов). Однако произвольное состояние можно представить как
суперпозицию состояний $\left|n\right>$ : 
\begin{equation}
\left|\psi\right> = \sum_{(n)} C_n \left|n\right>
\end{equation}
где $\left|C_n\right|^2$ - вероятность при измерении обнаружить в моде
$n$ фотонов; $\sum_{(n)} \left|C_n\right|^2 = 1$. 
В приложении \ref{AddDirac} показано, что
\[
C_n = \left< n \right.\left| \psi \right>, \quad
\left| \psi \right> = \sum_{(n)} \left< n \right.\left| \psi \right>
\left| n \right> =
\sum_{(n)} \left| n \right>\left< n \right.\left| \psi \right>,
\quad
\sum_{(n)} \left| n \right>\left< n \right| = \hat{I},
\]
где $\hat{I}$ - единичный оператор.

\begin{remark}[О состояниях с определенной энергией в квантовой
    механике]
  Стоит отметить, что состояния с определенной энергией не нарушают 
  ввиду соотношение неопределенности Гейзенберга для пары
  энергия - время
  (см. прил. \ref{AddHeisenbergUncertaintyPrincipleEnergyTime}):
  \[
  \Delta E \Delta t \ge \frac{\hbar}{2},
  \]
  (см. прил. \ref{AddHeisenbergUncertaintyPrincipleEnergyTime}).

  Вместе с тем, если посмотреть на оператор энергии гармонического
  осциллятора в виде
  \[
  \hat{\mathcal{H}} =\frac{1}{2} \left(\hat{p}^2 +
  \omega^2\hat{q}^2\right)
  \]
  и воспользоваться выражениями (\ref{eqCh1_qpdef}) можно получить,
  что $\left<n\right|\hat{q}\left|n\right> =
  \left<n\right|\hat{p}\left|n\right>$. Вместе с тем
  \begin{eqnarray}
    \left<n\right|\hat{q}^2\left|n\right> = \frac{\hbar}{2 \omega}
    \left[
      \left<n\right|\hat{a}^2\left|n\right> +
      \left<n\right|\left(\hat{a}^{+}\right)^2\left|n\right> +
      \left<n\right|\hat{a}^{+}\hat{a}\left|n\right> +
      \left<n\right|\hat{a}\hat{a}^{+}\left|n\right>
      \right] =
    \nonumber \\
    = \frac{\hbar}{2 \omega}
    \left[n + n + 1\right],
    \nonumber
  \end{eqnarray}
  a также
  \begin{eqnarray}
    \left<n\right|\hat{p}^2\left|n\right> = - \frac{\hbar \omega}{2}
    \left[
      \left<n\right|\hat{a}^2\left|n\right> +
      \left<n\right|\left(\hat{a}^{+}\right)^2\left|n\right> -
      \left<n\right|\hat{a}^{+}\hat{a}\left|n\right> -
      \left<n\right|\hat{a}\hat{a}^{+}\left|n\right>
      \right] =
    \nonumber \\
    = - \frac{\hbar \omega}{2}
    \left[n + n + 1\right].
    \nonumber
  \end{eqnarray}
  Т. о.
  \[
  \Delta p = \sqrt{\left<n\right|\hat{p}^2\left|n\right> -
    \left<n\right|\hat{p}\left|n\right>^2} =
  \sqrt{\frac{\hbar \omega}{2}\left(2n + 1\right)}
  \]
  и
  \[
  \Delta q = \sqrt{\left<n\right|\hat{q}^2\left|n\right> -
    \left<n\right|\hat{q}\left|n\right>^2} =
  \sqrt{\frac{\hbar}{2\omega}\left(2n + 1\right)}
  \]
  или
  \[
  \Delta p \Delta q = \frac{\hbar}{2}\left(2n + 1\right) \ge \frac{\hbar}{2},
  \]
  что находится в соответствии с соотношениями неопределенности
  Гейзенберга (\ref{eqAddHeisenbergUncertaintyPrinciple}).

  Таким образом состояния с определенной энергией это такие состояния
  в которых несмотря на невозможность определить без погрешностей $p$
  и $q$ имеется возможность определения величины 
  \(
  \frac{1}{2} \left(p^2 +
  \omega^2q^2\right)
  \). Это отражает тот факт, что для квантовых систем возможны
  ситуации при которых возможны составные события при отсутствии
  элементарных.

  Это может привести нас к рассуждению о том, что фотон это
  виртуальная частица, которая отсутствует в реальном физическом мире
  \cite{Lamb1995}. Вместе с тем математический аппарат связанный с
  понятием фотона как то, операторы рождения $\hat{a}^{+}$ и
  уничтожения $\hat{a}$, состояния электромагнитного поля с
  определенной энергией $\{\left|n\right>\}$ (которые могут быть
  использованы в качестве базисных состояний), представляется удобным
  для теоретического описания.

  Стоит так же отметить, что с точки зрения формального определения
  неклассического состояния (\ref{eqPart3_Nonclass_Nonclass7}),
  $\left|n\right>$ является неклассическим 
  состоянием света поскольку из (\ref{eqCh4_26}) следует $G^{(2)} <
  1$.

  С другой стороны, если посмотреть на минимально возможную энергию
  моды электромагнитного поля то из соотношения
  (\ref{eqCh1_hamilton_one_mode}) 
  \[
  \mathcal{H} = \frac{1}{2}\left(\omega^2 q^2 + p^2\right)
  \]
  следует, что в классическом случае минимально возможная нулевая
  энергия достигается при $p = 0, q=0$, но в силу
  (\ref{eqAddHeisenbergUncertaintyPrinciple}) нулевые значения
  невозможны в квантовом случае, а с учетом неопределенности измерений
  получим 
  \begin{eqnarray}
    \frac{1}{2}\left(\omega^2 (\Delta q)^2 + (\Delta p)^2\right) \ge
    \nonumber \\
    \ge \omega \Delta q \Delta p \ge \frac{\hbar \omega}{2}
    \nonumber
  \end{eqnarray}
  то есть минимально возможная энергия (энергия вакуума) определяется
  неравенствами Гейзенберга для пары координата - импульс.  
  \label{rem:antiphoton}
\end{remark}
