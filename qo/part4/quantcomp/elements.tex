%% -*- coding:utf-8 -*- 
\section{Квантовые логические элементы}
Каким образом может быть сконструирован элемент осуществляющий
преобразование $\hat{U}_f$ (\ref{eqQuantCompQuantComp}).

Набор квантовых вентилей называют универсальным, если любое унитарное
преобразование можно аппроксимировать с заданной точностью конечной
последовательностью вентилей из этого набора.

\subsection{Преобразование Адамара}

Одним из базовых квантовых логических элементов является
преобразование Адамара (см. рис. \ref{figQuantCompHadamar1}), которое
определяется следующими соотношениями
\begin{eqnarray}
\hat{H} \left|0\right> = \left|+\right> =  
\frac{\left|0\right> + \left|1\right> }{\sqrt{2}},
\nonumber \\
\hat{H} \left|1\right> = \left|-\right> = 
\frac{\left|0\right> - \left|1\right> }{\sqrt{2}},
\nonumber
\end{eqnarray}

\input ./part4/quantcomp/fighadamar1.tex

Это преобразование используется для получения суперпозиции состояний
содержащие все возможные значения аргумента вычисляемой функции
(см. рис. \ref{figQuantCompHadamar2}). 

\input ./part4/quantcomp/fighadamar2.tex

\subsection{Управляющие элементы}

\input ./part4/quantcomp/figcnot.tex

\input ./part4/quantcomp/figcelem.tex

\input ./part4/quantcomp/figphase.tex
 


