%% -*- coding:utf-8 -*- 
\subsection{Нахождение периода функций с помощью квантового
  преобразования Фурье}

Для определения периода функции \eqref{eqPart4QuantCompShorClassPart}
используется схема представленная на
\autoref{figQuantCompQuantPeriodFinding}.

\input part4/quantcomp/figquantperiodfinding.tex

Первым элементом стоит преобразование Адамара $n$ кубит, которое
подготавливает исходное состояние в виде:
\begin{equation}
\ket{in} = \frac{1}{\sqrt{2^n}}\sum_{x=0}^{2^n - 1} \ket{x}
\otimes\ket{0}.
\nonumber
\end{equation}

После элемента вычисляющего функцию $\hat{U}_f$ имеем для состояния
\begin{equation}
\hat{U}_f\ket{in} = \frac{1}{\sqrt{2^n}}\sum_{x=0}^{2^n - 1} \ket{x}
\otimes\left|f\left(x\right)\right>.
\nonumber
\end{equation} 

\input part4/quantcomp/figshorquant.tex

После измерения значения функции в списке координат останутся только
те элементы для которых значение функции будет равно измеренному
значению. В результате на вход элемента, измеряющего преобразование
Фурье подается состояние вида 
\begin{equation}
\ket{in'} = \sum_{x'} \ket{x'},
\nonumber
\end{equation} 
где все ненулевые элементы имеют одинаковую амплитуду и следуют с
периодом равным периоду исследуемой функции. При этом начальное
значение будет со сдвигом который зависит от эксперимента (в разных
экспериментах будет разный сдвиг). В силу леммы
\ref{lemmaAddFourierDiscretFourierShiftTime} фурье образ будет
одинаковым для различных измерений функций.

Далее в силу леммы \ref{lemmaAddFourierDiscretFourierPeriod}
(о периодичности) следует что наиболее вероятные отсчеты (максимумы
вероятности) следуют с периодом связанным с исходным периодом
функции. Таким образом в результате нескольких экспериментов период
искомой функции может быть найден с требуемым уровнем вероятности
(см. \autoref{picPart4QuantCompShorQuantPart}).

\begin{example}
\emph{Нахождение периода функции $f\left(x\right) = 2^x \mod 21$}
\label{exPart4QuantCompShorQuantPeriodFinding}
В качестве примера рассмотрим задачу о нахождении периода функции 
$f\left(x, a\right) = a^x \mod{N}$ при $a=2$, $N = 21$ см. 
\autoref{picPart4QuantCompShorQuantPart}

Число отсчетов  $M$ должно быть степенью двойки. В нашем примере мы
выбираем $M = 2^6 = 64$ в качестве числа отсчетов. Таким образом
необходимо 6 кубит для нашего примера.

Исходное состояние после  преобразования Адамара имеет вид:
\begin{equation}
\ket{in} = \frac{1}{8}\sum_{x = 0}^{63}\ket{x} \otimes \ket{0},
\nonumber
\end{equation}
где $\ket{x}$ представляет собой тензорное произведение 
\index{Тензорное произведение}
6 кубит
которые кодируют бинарное представление аргумента исследуемой
функции. Например при $x=5_{10}=000101_2$ имеем
\[
\ket{x} = \ket{0}\otimes \ket{0}\otimes
\ket{0}\otimes 
\ket{1}\otimes \ket{0}\otimes \ket{1}
\]

После вычисления функции мы имеем состояние вида (см. верхний график
на \autoref{picPart4QuantCompShorQuantPart})
\begin{eqnarray}
\hat{U}_f\ket{in} = \frac{1}{8}\sum_{x = 0}^{63}\ket{x}
\otimes \left|f\left(x\right)\right> = 
\nonumber \\
=
\frac{1}{8}
\left(
\ket{0}\otimes\ket{2} + 
\ket{1}\otimes\ket{4} + 
\ket{2}\otimes\ket{8} + \dots +
\right.
\nonumber \\
\left.
+
\ket{62}\otimes\ket{8} +
\ket{63}\otimes\ket{16}
\right).
\label{eqPart4QuantCompShorPFExample1}
\end{eqnarray}

Если результат измерения функции был равен $1$, то из всей суммы
\eqref{eqPart4QuantCompShorPFExample1} останутся члены для которых
значение функции равно $1$ (см. средний график
на \autoref{picPart4QuantCompShorQuantPart}):
\begin{equation}
\ket{in'} = \frac{1}{\sqrt{10}}\left( 
\ket{5}\otimes\ket{1} +
\ket{11}\otimes\ket{1} +
\ket{17}\otimes\ket{1} +
\dots +
\ket{60}\otimes\ket{1}
\right).
\label{eqPart4QuantCompShorPFExample2}
\end{equation} 
Выражение \eqref{eqPart4QuantCompShorPFExample2} содержит 10 членов
одинаковой амплитуды, поэтому нормирующий множитель имеет вид
$\frac{1}{\sqrt{10}}$.

Преобразование фурье для последовательности
\eqref{eqPart4QuantCompShorPFExample2} изображено на нижнем графике
\autoref{picPart4QuantCompShorQuantPart}. Наиболее вероятными
значениями результата измерения фурье образа будут значения
соотвествующие локальным максимумам которые повторяются с периодом 
$\frac{M}{r}\approx10.67$ откуда можно найти период искомой функции
$r=6$. 

\end{example}
