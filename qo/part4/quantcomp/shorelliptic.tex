%% -*- coding:utf-8 -*-
\section{Алгоритм Шора и дискретный логарифм на эллиптических кривых} 
Рассмотрим эллиптическую кривую 
\[
E\left(\mathbb{F}\right) = \{
(x,y) \in \mathbb{F}_p \times \mathbb{F}_p, y^2 = x^3 +a x + b,
\}
\]
с заданной базовой точкой $g \in E\left(\mathbb{F}_p\right)$ такой что: 
\[
n g = 0.
\]
Требуется решить следующую задачу: при заданном $q
\in E\left(\mathbb{F}_p\right)$ найти такое $x$, что
\begin{equation}
x g = q \mod n
\label{eq:part4:quantcomp:discretlogeliptic}
\end{equation}

Рассмотрим следующую вспомогательную функцию
\begin{equation}
f(x_1, x_2) = x_1 q + x_2 g = \left(x x_1 + x_2\right) g,
\label{eq:part4:quantcomp:discretlogeliptic:f}
\end{equation}
где $q,g \in E\left(\mathbb{F}_p\right)$ и взяты из условий нашей
задачи \eqref{eq:part4:quantcomp:discretlogeliptic}. Данная функция
аналогична \eqref{eq:part4:quantcomp:discretlogfunc}
использованной в решении задачи дискретного логарифма. Далее
производится измерение этой функции. Результатом этого измерения
является некоторая точка $c \in E\left(\mathbb{F}_p\right)$. Вместе с
тем из \eqref{eq:part4:quantcomp:discretlogeliptic:f} следует что 
$c \in \left<g\right>$, т.е. $\exists x_0$ такая что $c = x_0 g$. 

Т.о., по аналогии с \eqref{eq:part4:quantcomp:shordiscretlog:fprime}, мы
составляем следующую функцию 
\begin{equation}
\label{eq:part4:quantcomp:shorelliptic:fprime}
f'\left(x_1, x_2\right) = 
\begin{cases}
1, x x_1 + x_2 \equiv x_0 \mod n \\
0, x x_1 + x_2 \not\equiv x_0 \mod n \\
\end{cases}
\end{equation}

Координаты $(j_1,j_2)$ максимума Фурье образа $\tilde{f'}$ дают в
соответствии с формулой
\eqref{eq:part4:quantcomp:discretlog:periodfourier} некоторое значение
искомого числа $x$. В нашем случае практически всегда $n \ne M$
поэтому мы можем использовать тольк приближенную оценку
\[
x \approx \frac{j_1}{j_2}.
\]

\begin{example}[Дискретное логарифмирование на эллиптической кривой]
Рассмотрим задачу из примера \ref{ex:add:discretmath:ecdh}
Кривая и базовая точка (см. прим. \ref{ex:add:elliptic:basepoint})
имеют вид
\[
(p,a,b,g,n,h) = (97, -7, 10, (96,93), 41, 2)
\]
Допустим, что нам известен публичный ключ Алисы
\[
A = (37, 35)
\]
и мы хотим найти такое $x \in \{0,1, \dots 40\}$ что
$x g = A$, как это следует из примера \ref{ex:add:discretmath:ecdh}
ответом является $x = d_a  = 5$.
Нашей исследуемой функцией будет являться
\[
f\left(x_1,x_2\right) = x_1 A + x_2 g = x_1 (37,35) + x_2 (96,93)
\]

\input ./part4/quantcomp/figellipticdiscretlog1.tex

В качестве результата измерения выберем $c = g$, т.е. $x_0 = 1$.
График функции $f'(x_1, x_2)$, соответствующей этому измерению
изображен на \autoref{fig:part4:quantcomp:dle1}.

\input ./part4/quantcomp/figellipticdiscretlog2.tex

Фурье образ функции $f'(x_1, x_2)$, изображен на
\autoref{fig:part4:quantcomp:dle2}. Из него можно найти искомое $x =
5$. 

\end{example}
