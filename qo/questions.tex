%% -*- coding:utf-8 -*- 
\documentclass[12pt,a4paper]{article} 
\usepackage[utf8]{inputenc}
\usepackage[russian]{babel}
\usepackage[dvips]{color}
\usepackage[xdvi]{graphics}
\usepackage{longtable}

\begin{document}
\Russian

\title{Экзаменационные билеты \\ ``Квантовая оптика''}
%\author{И.\,В.~Мурашко\thanks{e-mail: bourbaki@mail.ru}}
\author{}
\date{Санкт-Петербург, 2015}
\maketitle
\newpage


\section*{I семестр}

\subsection*{Билет 1} 
\begin{enumerate}
\item Разложение электромагнитного поля по модам (типам колебаний).
Гамильтонова форма уравнений электромагнитного поля. Квантование
электромагнитного поля. 
\item Взаимодействие электромагнитного поля резонатора
  (гармонического осциллятора) с резервуаром атомов, находящихся при
  температуре $T$. Уравнение для матрицы плотности поля в представлении чисел
  заполнения.
\end{enumerate}

\subsection*{Билет 2} 
\begin{enumerate}
\item Разложение поля по плоским волнам в свободном пространстве. 
Плотность состояний. Гамильтонова форма уравнений поля при разложении по плоским
волнам. Квантование электромагнитного поля при разложении его по
плоским волнам.
\item Релаксация динамической системы. Метод матрицы плотности. 
\end{enumerate}

\subsection*{Билет 3} 
\begin{enumerate}
\item Свойства операторов $ \hat a $ и $ \hat a ^+ $. Квантовое
состояние электромагнитного поля  с определенной энергией. 
\item Взаимодействие электромагнитного поля резонатора
  (гармонического осциллятора) с резервуаром атомов, находящихся при
  температуре $T$. Уравнение движения статистического оператора поля моды в
  представлении когерентных состояний
\end{enumerate}

\subsection*{Билет 4} 
\begin{enumerate}
\item Многомодовые состояния. 
\item Излучение и поглощение атомом света. 
Гамильтониан системы атом-поле
\end{enumerate}

\subsection*{Билет 5} 
\begin{enumerate}
\item Когерентные состояния. 
\item Спонтанное излучение. Приближение Вайскопфа-Вигнера.
\end{enumerate}

\subsection*{Билет 6} 
\begin{enumerate}
\item Смешанные состояния электромагнитного поля. 
\item Взаимодействие
атома с модой электромагнитного поля. 
\end{enumerate}

\subsection*{Билет 7} 
\begin{enumerate}
\item Представление оператора плотности через когерентные
  состояния.
\item Взаимодействие атома с многомодовым полем. Спонтанные переходы.
\end{enumerate}

\subsection*{Билет 8} 
\begin{enumerate}
\item Свойства операторов рождения $\hat{a}^+$ и уничтожения $\hat{a}$.
\item Интерферометр Рамси. Квантовые неразрушающие измерения
\end{enumerate}

\section*{II семестр}

\subsection*{Билет 1} 
\begin{enumerate}
\item Модель лазера
\item Когерентные свойства света.
 Когерентность второго порядка.
 Когерентность высших порядков.
\end{enumerate}

\subsection*{Билет 2} 
\begin{enumerate}
\item Теория лазерной генерации
\item Сжатые состояния: применения сжатых состояний и их
  неклассичность. 
\end{enumerate}

\subsection*{Билет 3} 
\begin{enumerate}
\item Статистика лазерных фотонов
\item Когерентные свойства света.
 Когерентность второго порядка.
 Когерентность высших порядков.
\end{enumerate}

\subsection*{Билет 4} 
\begin{enumerate}
\item Теория лазера. Представление когерентных состояний
\item Фотоэффект
\end{enumerate}

\subsection*{Билет 5} 
\begin{enumerate}
\item Статистика лазерных фотонов
\item Счет и статистика фотонов.
Связь статистики фотонов со статистикой фотоотсчетов.
Распределение фотоотсчетов для когерентного и хаотического
  света.
\end{enumerate}

\subsection*{Билет 6} 
\begin{enumerate}
\item Модель лазера
\item Квантовое выражение для распределения фотоотсчетов.
\end{enumerate}

\subsection*{Билет 7} 
\begin{enumerate}
\item Теория лазерной генерации
\item Эксперименты по счету фотонов. Применение техники счета
  фотонов для спектральных измерений.
\end{enumerate}

\subsection*{Билет 8} 
\begin{enumerate}
\item Квантовое описание оптических интерференционных экспериментов
\item Неклассический свет.
\end{enumerate}

\subsection*{Билет 9} 
\begin{enumerate}
\item Интерферометр Маха-Цендера
\item Сжатые состояния. Сжатие квадратурного состояния. Генерация
  сжатых состояний. Наблюдение сжатых состояний
\end{enumerate}

\subsection*{Билет 10} 
\begin{enumerate}
\item Теория лазера. Представление когерентных состояний. Естественная
  ширина линии излучения лазера.
\item Перепутанные состояния: определение, генерация, регистрация
\end{enumerate}

\subsection*{Билет 11} 
\begin{enumerate}
\item Когерентные свойства света. Когерентность первого порядка
\item Перепутанные состояния: применения. Квантовая
  телепортация. Квантовая криптография
\end{enumerate}

\subsection*{Билет 12} 
\begin{enumerate}
\item Когерентные свойства света. Когерентность второго порядка
\item Неравенства Белла. Неклассичность перепутанных состояний
\end{enumerate}

\end{document}
