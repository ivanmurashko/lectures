%% -*- coding:utf-8 -*- 
\section{Классическое описание поляризационных свойств света}
Поле произвольной монохроматической плоской волны можно представить в
следующем виде:
\begin{equation}
\vec{E} = \vec{E_0} e^{-i \left(\omega t - \vec{k}\vec{r}\right)},
\nonumber
\end{equation}
где вектор 
\[
\vec{E_0} = E_{x}\vec{e}_x + E_{y}\vec{e}_y
\]
определяет интенсивность и поляризационные свойства электромагнитного
излучения. Для описания поляризационных свойств используется
комплексный вектор поляризации - вектор Джонса:
\rindex{вектор Джонса}
\begin{equation}
\vec{e} = \alpha \vec{e}_x + \beta \vec{e}_y.
\label{eqEntangJones}
\end{equation}
Компоненты этого вектора, $\alpha$ и $\beta$ могут быть представлены
как точки на некторой сфере называемой сферой Пуанкаре. Координаты
этих точек определяются двумя углами $\theta$ и $\varphi$
\begin{eqnarray}
\alpha = \frac{E_x}{\left|E_x\right|^2 + \left|E_y\right|^2} = 
 cos \, \frac{\theta}{2},
\nonumber \\
\beta = \frac{E_y}{\left|E_x\right|^2 + \left|E_y\right|^2} = 
 e^{i\varphi} sin \, \frac{\theta}{2}.
%\label{eqEntangJones2}
\nonumber
\end{eqnarray}
Для измерения компонентов вектора Джонса $\alpha$ и $\beta$ может быть
использована схема, представленная на
\autoref{figPart3EntangJones}. Здесь пучок поляризованного света
направляется на призму Николя $P$, разделяющую $x$ и $y$ компоненты
этого пучка, которые подаются на два фотодетектора $D_x$ и $D_y$.

\input ./part2/polarization/figjones.tex

Кроме вектора Джонса, для описания поляризационных свойств часто
используют вектор Стокса, четыре компоненты которого
имеют размерность интенсивности и могут быть легко измерены
экспериментально. Вектор Стокса может быть определен следующим
образом:
\rindex{вектор Стокса}
\rindex{параметры Стокса!классический случай}
\begin{eqnarray}
S_0 = \left|E_x\right|^2 + \left|E_y\right|^2,
\nonumber \\
S_1 = \left|E_x\right|^2 - \left|E_y\right|^2,
\nonumber \\
S_2 = E_x^{*} E_y + E_x E_y^{*} = 2 Re \left(E_x^{*} E_y\right),
\nonumber \\
S_3 = \frac{E_x^{*} E_y - E_x E_y^{*}}{i} = 2 Im \left(E_x^{*}
E_y\right),
\label{eqEntangStokes}
\end{eqnarray}
где значения амплитуд $E_{x,y}$ взяты в некоторый момент времени $t$. 
Параметр $S_0$ в \eqref{eqEntangStokes} определяет интенсивность волны
в момент времени $t$,
а остальные три параметра $S_1$, $S_2$ и $S_3$ - поляризационные
свойства. 

\input ./part2/polarization/figstokes.tex

Для измерения параметров Стокса может быть использована схема,
представленная на \autoref{figPart3EntangStokes}
\cite{bEntangKlyshko}. Для измерения $S_1$, $S_2$ и $S_3$ необходимо
разделить исходный пучок на три части, каждая из которых подается на
детектор, изображенный на \autoref{figPart3EntangStokes}.

\input ./part2/polarization/figs2.tex

Для измерения $S_1$ используется призма Николя $P$, которая разделяет 
$x$ и $y$ поляризованные компоненты пучка,
так что разность токов двух фотодетекторов пропорцианальна $S_1$, при
этом сумма токов будет пропорциональна $S_0$. 
Для измерения $S_2$ призма поворачивается на угол $\xi =
\frac{\pi}{4}$, так что разность токов будет пропорциональна $S_2$
(см. \autoref{figPart3EntangS2}). 
Для измерения $S_3$ перед призмой ставиться фазовая пластина
$\frac{\lambda}{4}$ с ориентацией $\frac{\pi}{4}$, в результате
разность токов фотодетекторов будет пропорциональна $S_3$.

Результаты показаний детекторов усредняются по времени. Таким образом можно
говорить о том, что измеряются усредненные параметры Стокса - 
$\left<S_k\right>$, которые могут быть также использованы для описания
частично поляризованного света. Для количественной характеристики
такого излучения используется величина, называемая
степенью поляризации:
\begin{equation}
P = \frac{\sqrt{\left<S_1\right>^2 + \left<S_2\right>^2 +
    \left<S_3\right>^2}}{\left<S_0\right>}.
\label{eqEntangPolyarDegree}
\nonumber
\end{equation}
Степень поляризации полностью поляризованного света $P = 1$. Для  полностью не
поляризованного света $P = 0$.
