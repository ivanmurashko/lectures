%% -*- coding:utf-8 -*- 
\section{Счет и статистика фотонов}
Мы убедились, что квантование оптического поля и существование квантов
света (фотонов) сохраняет общую картину интерференционных
явлений, если наблюдение проводится достаточно долго, для того
чтобы проявилась усредненная картина. В наших формулах это
соответствует усреднению при помощи матрицы плотности. Однако
существование фотонов дает возможность проведения
экспериментов нового типа, основанных на счете фотонов, и
исследования их статистических закономерностей. Эти методы
называются методами счета фотонов. Суть их в следующем:
исследуемый свет подается на фотодетектор, соединенный со
счетчиком, считающим число фотоэлектронов, зафиксированных за
определенное время. Затвор перед фотодетектором (или
блокировка схемы) контролирует продолжительность счета. При
счетчике, установленном на нуль, затвор открывается на время
$T$,  и регистрируется число фотоэлектронов. Через время,
большее, чем время корреляции $\tau_c$,  все повторяется снова много
раз. По результатам измерений можно определить $P_m\left(T\right)$ -
вероятность зарегистрировать $m$  отсчетов фотоэлектронов за
время $T$: 
\begin{equation}
P_m\left(T\right) = \frac{N_m}{N},
\label{eqCh4_40}
\end{equation}
где $N_m$ - число измерений, в которых зафиксировано $m$
фотоэлектронов, $N$ - полное число измерений, которое должно быть
велико. Очевидно, предполагается, что световой поток
стационарен. Полученное распределение содержит информацию о
спектральных свойствах световых пучков. Первой здесь является задача,
как по статистике фотоотсчетов, которую мы можем измерить, найти
статистику фотонов, нужную нам для получения сведений о свойствах
световых пучков.  
