%% -*- coding:utf-8 -*- 
\section{Квантовое выражение для распределения фотоотсчетов}
Полученное выше выражение для распределения фотоотсчетов основано на
полуклассическом подходе. Полностью квантовое выражение имеет
вид \cite{bLoudon1976}: 
\begin{equation}
P_m\left(T\right) = Sp \left\{
\hat{\rho}\hat{N}
\left[
\frac{\left(\beta \hat{\bar{I}}\left(t\right) T\right)^m}{m!}
e^{- \beta \hat{\bar{I}}\left(t\right) T}
\right]
\right\} 
\label{eqCh4_59}
\end{equation}
где
\begin{eqnarray}
\hat{\bar{I}}\left(t\right) = \frac{1}{T}\int_t^{t + T}2
\varepsilon_0 c \hat{\bar{E}}^{(-)}\left(t'\right)
\hat{\bar{E}}^{(+)}\left(t'\right)dt' =
\nonumber \\
= \frac{2}{T}\int_t^{t + T}
\sqrt{\frac{\varepsilon_0}{\mu_0}} \hat{\bar{E}}^{(-)}\left(t'\right)
\hat{\bar{E}}^{(+)}\left(t'\right)dt'
\nonumber
\end{eqnarray}
соответствует усредненному за период счета оператору потока энергии,
$\beta$ - квантовая эффективность фотоприемника. Формула
\eqref{eqCh4_59} по виду очень 
похожа на формулу Манделя. Отличие в том, что на месте классических
полей стоят их операторы, усреднение проводится квантово-механически
при помощи статистического оператора, и присутствует оператор
нормального упорядочения $\hat{N}$.  Действие этого оператора сводится
к тому, что в операторе, на который он действует, следует расположить 
операторы поглощения справа от операторов рождения. При разложении
оператора, входящего в \eqref{eqCh4_59}, в степенной ряд появятся
члены вида  $\left(\hat{\bar{E}}^{(-)}\hat{\bar{E}}^{(+)}\right)^n$.  
Действие на них оператора $\hat{N}$  определяется выражением 
\[
\hat{N}\left(\hat{\bar{E}}^{(-)}\hat{\bar{E}}^{(+)}\right)^n = 
\left(\hat{\bar{E}}^{(-)}\right)^n\left(\hat{\bar{E}}^{(+)}\right)^n
\]
Подобные члены соответствуют когерентности высших порядков, то есть
распределение фотоотсчетов зависит от когерентностей всех
степеней. Формула \eqref{eqCh4_59} может быть обоснована следующим образом:
положим, что поле находится в когерентном состоянии
$\left|\left\{\alpha_k\right\}\right>$.  Тогда 
вероятность эмиссии электрона в интервале времени $dt$ равна 
\[
Pdt = \sigma
\left<\left\{\alpha_k\right\}\right|
\hat{E}^{(-)} \hat{E}^{(+)}
\left|\left\{\alpha_k\right\}\right> dt = 
\sigma \left(E^{*} E\right)dt,
\]
где $\sigma$ характеризует эффективность фотокатода, а $E$ -
классическое поле, соответствующее состоянию $\left|\left\{\alpha_k\right\}\right>$
\begin{eqnarray}
\hat{E}^{(+)} \left|\left\{\alpha_k\right\}\right> = E \left|\left\{\alpha_k\right\}\right>,
\nonumber \\
\left<\left\{\alpha_k\right\}\right|\hat{E}^{(-)} = E^{*} \left<\left\{\alpha_k\right\}\right|.
\nonumber
\end{eqnarray}

Далее рассмотрение ведется таким же образом, как и при
полуклассическом подходе. Получается аналогичная формула: 
\begin{equation}
P_m\left(\left.T\right|\left\{\alpha_k\right\}\right) = 
\frac{\left(\sigma \overline{E^\ast E} T\right)^m}{m!}
e^{- \sigma T \overline{E^\ast E}}
\label{eqCh4_60}
\end{equation}
где
\[
\overline{E^\ast E} = \frac{1}{T} \int_t^{t + T}E^\ast\left(t'\right)
E\left(t'\right)dt'.
\]
Это выражение можно представить в операторном виде:
\begin{equation}
P_m\left(\left.T\right|\left\{\alpha_k\right\}\right) = 
\left<\left\{\alpha_k\right\}\right|
\hat{N}
\left\{
\frac{\left(\sigma \overline{\hat{E}^{(-)} \hat{E}^{(+)}} T\right)^m}{m!}
e^{- \sigma T \overline{\hat{E}^{(-)} \hat{E}^{(+)}}}
\right\}
\left|\left\{\alpha_k\right\}\right>,
\label{eqCh4_61}
\end{equation}
где $P_m\left(\left.T\right|\left\{\alpha_k\right\}\right)$ -
вероятность счета $m$ фотоэлектронов за время $T$,  если поле
находится в состоянии $\left|\left\{\alpha_k\right\}\right>$.
Оператор, стоящий в фигурных скобках, можно 
рассматривать как оператор счета $m$ фотоэлектронов за время $T$.  Если
поле находится в статистически смешанном состоянии со статистическим
оператором, определяемым в диагональном представлении функцией 
$P_m\left(\left.T\right|\left\{\alpha_k\right\}\right)$
\[
\hat{\rho} = \int_{\left\{\alpha_k\right\}}
P\left(\left\{\alpha_k\right\}\right)
\left|\left\{\alpha_k\right\}\right>
\left<\left\{\alpha_k\right\}\right|
\prod_k d^2\alpha_k,
\]
среднее значение оператора, входящего в выражение \eqref{eqCh4_61},
согласно \eqref{eqCh1_middleO}, будет равно 
\begin{eqnarray}
P_m\left(T\right) =
\int_{\left\{\alpha_k\right\}}
P\left(\left\{\alpha_k\right\}\right)
\left<\left\{\alpha_k\right\}\right|
\cdot
\nonumber \\
\cdot 
\hat{N}
\left\{
\frac{\left(\sigma \overline{\hat{E}^{(-)} \hat{E}^{(+)}} T\right)^m}{m!}
e^{- \sigma T \overline{\hat{E}^{(-)} \hat{E}^{(+)}}}
\left|\left\{\alpha_k\right\}\right>
\right\}
\prod_k d^2\alpha_k
\label{eqCh4_62}
\end{eqnarray}
Полученную формулу можно записать в произвольном представлении, так
как операция $Sp$ не зависит от представления. Имеем 
\begin{equation}
P_m = Sp\left(
\hat{\rho}
\hat{N}
\left\{
\frac{\left(\sigma \overline{\hat{E}^{(-)} \hat{E}^{(+)}} T\right)^m}{m!}
e^{- \sigma T \overline{\hat{E}^{(-)} \hat{E}^{(+)}}}
\right\}
\right),
\label{eqCh4_63}
\end{equation}
где
\[
\overline{\hat{E}^{(-)} \hat{E}^{(+)}} = \frac{1}{T} \int_t^{t + T}\hat{E}^{(-)}\left(t'\right)
\hat{E}^{(+)}\left(t'\right)dt'.
\]

Формула \eqref{eqCh4_63} полностью соответствует формуле
\eqref{eqCh4_59}, взятой из \cite{bLoudon1976}. Расчеты по
формуле \eqref{eqCh4_63} несколько 
сложнее, чем по формуле Манделя. Для примера рассмотрим несколько
простейших случаев. Сперва рассмотрим одномодовые состояния. Тогда
\eqref{eqCh4_63} можно представить в следующем виде 
\[
P_m = Sp\left(
\hat{\rho}
\hat{N}
\left\{
\frac{\left(\gamma \overline{\hat{a}^{\dag} \hat{a}} T\right)^m}{m!}
e^{- \gamma T \overline{\hat{a}^{\dag} \hat{a}}}
\right\}
\right)
\]
где $\gamma$ - коэффициент, характеризующий в этой записи
эффективность фотокатода. Оператор 
\[
\hat{N}
\left\{
\frac{\left(\gamma \overline{\hat{a}^{\dag} \hat{a}} T\right)^m}{m!}
e^{- \gamma T \overline{\hat{a}^{\dag} \hat{a}}}
\right\}
\]
имеет в представлении чисел заполнения (чисел фотонов) только
диагональные члены. Тогда $Sp\left(\dots\right)$ будет определяться
только диагональными элементами матрицы плотности, 
которые обозначим $P_n = \left<n\right|\hat{\rho}\left|n\right>$.

Тогда \eqref{eqCh4_63} можно представить в виде:
\begin{eqnarray}
P_m = \sum_n P_n 
\left<n\right|
\hat{N}
\frac{\left(\gamma \hat{a}^{\dag} \hat{a} T\right)^m}{m!}
e^{- \gamma T \hat{a}^{\dag} \hat{a}}
\left|n\right> = 
\nonumber \\
=
\sum_n P_n 
\frac{\left(\gamma T\right)^m}{m!}
\left<n\right|
\sum_l\left(-1\right)^l
\frac{\left(\gamma T\right)^l}{l!}
\left(\hat{a}^{\dag}\right)^{l + m}
\left(\hat{a}\right)^{l + m}
\left|n\right> = 
\nonumber \\
=
\sum_{n = m}
P_n 
\frac{\left(\gamma T\right)^m}{m!}
\sum_{l = 0}^{n - m}\left(-1\right)^l
\frac{\left(\gamma T\right)^l}{l!}
\frac{n!}{\left(n - m - l\right)!},
\label{eqCh4_64}
\end{eqnarray}
что следует из соотношения:
\begin{eqnarray}
\left<n\right|
\left(\hat{a}^{\dag}\right)^{l + m}
\left(\hat{a}\right)^{l + m}
\left|n\right> = 
\nonumber \\
= \left\{
n \left(n - 1\right)\left(n - 2\right) \dotsc
\left(n - m - l + 1\right)
\right\} = 
\frac{n!}{\left(n - m - l\right)!}.
\nonumber
\end{eqnarray}
Выражение \eqref{eqCh4_64} можно подвергнуть дальнейшим
преобразованиям. Известно разложение (бином Ньютона): 
\[
\left(1 - \gamma T\right)^{n - m} = 
\sum_{l = 0}^{n -m}
\left(-1\right)^l
\left(\gamma T\right)^l
\frac{\left(n - m\right)!}{l!\left(n - m - l\right)!}.
\]
Отсюда имеем:
\begin{equation}
P_m\left(T\right) = 
\sum_{n - m}^{\infty}
P_n 
\frac{n!}{m!\left(n - m\right)!}
\left(\gamma T\right)^m
\left(1 - \gamma T\right)^{n - m}.
\label{eqCh4_64a}
\end{equation}
Заметим, что, как известно из теории вероятности (см. распределение
Бернулли), общий член суммы определяет вероятность того, что из $n$
фотонов $m$ будет зарегистрировано, а  $n - m$  останутся
незарегистрированными. Применим общую формулу \eqref{eqCh4_64} к случаю
хаотического света. Тогда  
\[
P_n = \frac{\bar{n}^n}{\left(\bar{n} + 1\right)^{n + 1}},
\]
и из \eqref{eqCh4_64} получим 
\begin{eqnarray}
P_m\left(T\right) = 
\sum_{n - m}^{\infty}
\frac{\bar{n}^n}{\left(\bar{n} + 1\right)^{n + 1}}
\left(\gamma T\right)^m
\left(1 - \gamma T\right)^{n - m} 
\frac{n!}{m!\left(n - m\right)!}
=
\nonumber \\
=
\left(\gamma T\right)^m
\sum_{l = 0}^{\infty}
\frac{\bar{n}^{l + m}}{\left(\bar{n} + 1\right)^{l + m + 1}}
\left(1 - \gamma T\right)^{l}
\frac{\left(l + m\right)!}{m! l!}. 
\label{eqCh4_65}
\end{eqnarray}

При преобразовании сделана замена индексов суммирования: $l = n - m$,
$n = l + m$.

Известно разложение \cite{bDvait1973}:
\[
\left(1 - x\right)^{-\left(m + l\right)} = 
\sum_{l = 0}^\infty
\frac{x^l \left(m + l\right)!}{m! l!}.
\]
Используя его, получим:
\begin{eqnarray}
P_m\left(T\right) = 
\frac{\left(\gamma T \bar{n}\right)^m}{\left(1 + \bar{n}\right)^{m +
    1}}
\left(
1 - \frac{\bar{n}}{1 + \bar{n}}\left(1 - \gamma T\right)
\right)^{-\left(m + 1\right)} = 
\nonumber \\
=
\frac{\left(\gamma T \bar{n}\right)^m}{\left(1 + \bar{n}\right)^{m +
    1}} 
\left\{
\frac{1 - \gamma T \bar{n}}{1 + \bar{n}} 
\right\}^{-\left(m + 1\right)} = 
\frac{\left(\gamma T \bar{n}\right)^m}{\left(1 + \gamma T \bar{n}\right)^{m +
    1}} =
\frac{\bar{m}^m}{\left(\bar{m} + 1\right)^{m + 1}},
\label{eqCh4_66}
\end{eqnarray}
где $\bar{m} = \gamma T \bar{n}$.  Мы получили такой же результат, как
и при полуклассическом 
рассмотрении. Распределение $P_m\left(T\right)$ повторяет $P_n$,  но в
другом 
масштабе. Вместо $\bar{n}$ стоит $\bar{m} = \gamma T
\bar{n}$. Рассмотрим теперь другой предельный случай: поле находится в
когерентном состоянии. Его удобно рассматривать в представлении
когерентных состояний. Имеем  
\begin{eqnarray}
P_m\left(T\right) = 
\left<\alpha\right|
\hat{N}
\frac{\left(\gamma \hat{a}^{\dag} \hat{a} T\right)^m}{m!}
e^{- \gamma T \hat{a}^{\dag} \hat{a}}
\left|\alpha\right> = 
\nonumber \\
=
\frac{\left(\gamma \left|\alpha\right|^2 T\right)^m e^{-\gamma
    \left|\alpha\right|^2 T}}{m!} = 
\frac{\bar{m}^m}{m!}e^{-\bar{m}},
\label{eqCh4_67}
\end{eqnarray}
где $\bar{m} = \gamma \left|\alpha\right|^2 T = \gamma \bar{n} T$.

Распределение фотонов для когерентного состояния, как известно
\eqref{eqCh1_PuassonCoh}, 
равно  
\[
P_n = \frac{\bar{n}^n e^{-\bar{n}}}{n!}.
\]
Таким образом, и в этом случае распределение фотоотсчетов
повторяет распределение фотонов, но в другом масштабе. Вместо
$\bar{n}$ стоит $\bar{m}$.  В общем случае  такой простой связи не
будет. Из проведенного нами вывода квантовой формулы фотоотсчетов
следует, что пока $P\left(\left\{\alpha_k\right\}\right)$ в выражении 
для статистического оператора можно интерпретировать как распределение
вероятности, расхождения результатов полуклассического рассмотрения
(формулы Манделя), и полностью квантового рассмотрения не будет. Если
же $P\left(\left\{\alpha_k\right\}\right)$ в некоторых областях может
быть отрицательно, результаты будут различаться.  
