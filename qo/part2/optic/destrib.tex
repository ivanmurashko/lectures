%% -*- coding:utf-8 -*- 
\section{Распределение фотоотсчетов для когерентного и хаотического
  света}
Для того чтобы получить представление, как связаны между собой
статистика фотоотсчетов и статистика падающего на фотодетектор света,
применим формулу \eqref{eqCh4_49} к двум предельным случаям:
постоянного по амплитуде (интенсивности) света и хаотического света с
флуктуирующей 
интенсивностью. Начнем с наиболее простого случая, когда интенсивность падающего
света постоянна. В этом случае $\bar{I}\left(t, T\right)$ не зависит
от $t$ и $T$, $\bar{I}\left(t, T\right) = I_0 = const$,  поэтому  
второе усреднение не нужно. То же самое получим, если в
\eqref{eqCh4_49} будем считать $P\left(\bar{I}\right) =
\delta\left(\bar{I} - I_0\right)$.  Окончательное выражение для случая 
постоянного по интенсивности света имеет вид: 
\begin{equation}
P_m = \frac{\bar{m}^m}{m!}e^{- \bar{m}}.
\label{eqCh4_50}
\end{equation}
где $\bar{m} = \xi I_0 T$ - среднее число фотоотсчетов, регистрируемых
фотодетектором за время $T$.  Таким образом, для постоянной
интенсивности распределение фотоотсчетов является распределением
Пуассона. \rindex{распределение Пуассона}
Заметим, что при достаточно большом $T$ (значительно большем
времени корреляции $\tau_c$), $\bar{I}\left(t, T\right)$  в любом
случае\footnote{в том числе для хаотического света}  будет стремиться к
постоянной величине, а распределение фотоотсчетов - к распределению
Пуассона. \rindex{распределение Пуассона}
  
В другом предельном случае, когда $T \ll \tau_c$,  можно считать
$\bar{I}\left(t, T\right) = I\left(t\right)$ - мгновенной
интенсивности. В случае хаотического света $P\left(I\right) =
\frac{1}{\bar{I}} e^{- \frac{I}{\bar{I}}}$ \cite{bLoudon1976}.  
По формуле Манделя имеем:
\begin{eqnarray}
P_m\left(T\right) = \frac{1}{\bar{I}}\int_0^{\infty} e^{- \frac{I}{\bar{I}}}
\frac{\left(\xi I T\right)^m}{m!} e^{-
  \xi I T} d I = 
\nonumber \\
= \int_0^{\infty} \frac{y^m\left(\bar{I} \xi
  T\right)^m}{m!\left(1 + \bar{m}\right)^{m + 1}} e^{-y} dy.
\label{eqCh4_51}
\end{eqnarray}
Здесь сделана замена переменных:
\[
y = I \left(\frac{1}{\bar{I}} + \xi T\right),
\]
\[
I = \frac{y \bar{I}}{1 + \xi \bar{I} T},
\]
\[
\bar{m} = \xi \bar{I} T.
\]
Окончательно получаем
\begin{equation}
P_m\left(T\right) = 
\frac{\bar{m}^m}{\left(1 + \bar{m}\right)^{m + 1} m!}
\int_0^{\infty}y^m e^{-y}dy = 
\frac{\bar{m}^m}{\left(1 + \bar{m}\right)^{m + 1}}
\label{eqCh4_52}
\end{equation}
так как
\[
\int_0^{\infty}y^m e^{-y}dy = m!.
\]

Мы получили, что распределение фотоотсчетов повторяет распределение
фотонов для хаотического поля, но с измененным масштабом. Значительно
сложней определить распределение фотоотсчетов для времени счета $T$,
соответствующего промежуточному случаю $T \approx \tau_c$.  Здесь
возможны только численные расчеты. Результаты таких расчетов приведены
на \autoref{figPart4Ch2_6} \cite{bLoudon1976}. Они позволяют судить о характере
изменения распределения фотоотсчетов при увеличении отношения
$T/\tau_c$.  Из графиков видно, что характер распределения меняется
вблизи $T = \tau_c$.  Это позволяет, проводя измерения при различных
$T$,  оценить $\tau_c$ и ширину спектра падающего 
света $\sim 1/\tau_c$. 

\input ./part2/optic/fig6.tex
