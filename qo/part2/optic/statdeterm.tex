%% -*- coding:utf-8 -*- 
\section{Определение статистики фотонов через распределение
  фотоотсчетов}
Обычно основной интерес представляет статистика фотонов, которая
характеризует световой пучок. Чтобы ее определить по измеренным
$P_m\left(T\right)$, следует обратить \eqref{eqCh4_49}, выразив
$P\left(\bar{I}\right)$ через $P_m\left(T\right)$. 
   
Существует несколько способов нахождения $P\left(\bar{I}\right)$.  (О
способах обращения см. \cite{bDvait1973}.) Остановимся на одном
из них.  Рассмотрим выражение \eqref{eqCh4_49}
\[
P_m\left(T\right) = 
\int_0^{\infty}
P\left(\bar{I}\right)
\frac{\left(\xi \bar{I}T\right)^m}{m!} e^{-
  \xi \bar{I} T} 
d \bar{I}.
\]
Из этого равенства по измеренным $P_m\left(T\right)$ нужно определить
плотность вероятности  $P\left(\bar{I}\right)$,  характеризующую
статистику падающего света.   

Перейдем к новой переменной $u = \xi \bar{I} T$ и определим новую
плотность вероятности 
\[
P\left(u\right) = \frac{1}{\xi T} P\left(\bar{I}\right).
\]
Имеем: 
\[
P_T\left(u\right) du = \frac{1}{\xi T} P\left(\bar{I}\right) \xi T d
\bar{I} = P\left(\bar{I}\right) d \bar{I}.
\]
Отсюда: 
\[
\int_0^{\infty}P_T\left(u\right) du = 
\int_0^{\infty}P\left(\bar{I}\right) d\bar{I} = 1, 
\]
как и должно быть. Запишем уравнение \eqref{eqCh4_49} в этих  
переменных 
\begin{equation}
P_m\left(T\right) = 
\int_0^{\infty}
P_T\left(u\right)
\frac{u^m}{m!} e^{-u} 
d u.
\label{eqCh4_53}
\end{equation}
Далее используем ортогональные полиномы Лагерра
\begin{eqnarray}
L_n\left(y\right) = \sum_{k = 0}^{n}{n\choose k}
\frac{\left(-y\right)^k}{k!},
\nonumber \\
{n\choose k} = C_k^n = \frac{n!}{k!\left(n - k\right)!}
\label{eqCh4_54}
\end{eqnarray}
удовлетворяющие условиям ортогональности 
\begin{equation}
\int_0^{\infty}L_p\left(y\right)L_q\left(y\right)e^{-y} dy =
\delta_{pq},
\label{eqCh4_TaskLager1}
\end{equation}
которым заменой переменных $y \rightarrow 2y$ можно придать вид
\begin{equation}
2 \int_0^{\infty}L_p\left(2 y\right)L_q\left(2 y\right)e^{-2 y} dy =
\delta_{pq}. 
\label{eqCh4_TaskLager2}
\end{equation}
Разложим теперь $P_T\left(u\right)$ в ряд по полиномам Лагерра:
\begin{equation}
P_T\left(u\right) = 
\sum_{n = 0}^{\infty}
A_n L_n\left(2u \right)
e^{-u} 
\label{eqCh4_55}
\end{equation}
Используя условия ортогональности \eqref{eqCh4_TaskLager2}, получим для
коэффициентов разложения $A_n$ выражение  
\begin{equation}
A_n = 2 \int_0^{\infty}
L_n\left(2 u\right)P_T\left(u\right)e^{-u}du.
\label{eqCh4_56}
\end{equation}
Подставим в \eqref{eqCh4_56} $L_n\left(2 u\right)$ в виде ряда
\eqref{eqCh4_54}. Получим: 
\begin{eqnarray}
A_n = 2 \sum_{k = 0}^{k = n}
{n\choose k}\left(- 2\right)^k\int_0^{\infty}
P_T\left(y\right)\frac{y^k}{k!}e^{-y} dy = 
\nonumber \\
= 2 \sum_{k = 0}^{k = n}
{n\choose k}
\left(- 2\right)^k
P_k\left(T\right),
\label{eqCh4_57}
\end{eqnarray}
где использовано выражение  
\[
\int_0^{\infty}P\left(y\right)\frac{y^k}{k!}e^{-y}dy =
P_k\left(T\right). 
\]
Будем считать, что из измерений нам
известно достаточное число значений $P_k\left(T\right)$. Это позволит
вычислить достаточное количество коэффициентов разложения $A_n$ по
формуле \eqref{eqCh4_57}. Подставляя их в ряд \eqref{eqCh4_55} для
$P_T\left(u\right)$ и переходя к исходным переменным, получим
\begin{equation}
P\left(\bar{I}\right) = \xi T\sum_{k = 0}^{\infty}
A_n L_n\left(2 \xi T \bar{I}\right)e^{- \xi T \bar{I}}.
\label{eqCh4_58}
\end{equation}
При помощи этого выражения в принципе можно обратить формулу
Манделя. Трудность здесь связана с тем, что нам известно только
конечное число $P_n\left(T\right)$ с ограниченной точностью. 

