%% -*- coding:utf-8 -*- 
\section{Статистика лазерных фотонов}
Для уравнения \eqref{eqCh3_6} можно наглядно представить потоки
вероятностей, изображенные на \autoref{figPart2Ch1_2}. 

\input ./part2/laser/fig2.tex

Член $\frac{\omega}{Q} n \rho_{nn}$ представляет поток из состояния
$\ket{n}$ в состояние $\ket{n - 1}$ из-за 
поглощения фотонов атомами первого резервуара. То же самое можно
сказать и о члене $\frac{\omega}{Q} \left(n + 1\right) \rho_{n + 1, n
  + 1}$ - потоке вероятности из состояния $\ket{n + 1}$ в состояние
$\ket{n}$ по тем же причинам. 

Член $\left[A - \left(n + 1\right) B\right]\left(n + 1\right)
\rho_{nn}$ представляет поток из 
состояния $\ket{n}$ в состояние  
$\ket{n + 1}$ и характеризует рождение фотонов из-за
взаимодействия с активными атомами, с учетом влияния насыщения. То же
относится к члену   - $\left(A - n B\right)n \rho_{n - 1, n - 1}$
потоку вероятности из состояния $\ket{n - 1}$ в состояние
$\ket{n}$.  

$\rho_{nn}$ имеет смысл вероятности обнаружения $n$ фотонов в лазерной
моде. В переходном режиме $\rho_{nn}\left(t\right)$ зависит от
времени. В стационарном режиме $\dot{\rho}_{nn} = 0$
и из уравнения \autoref {eqCh3_6} можно установить статистику фотонов в режиме
стационарной генерации \cite{bScally1974}. 

Приравняем, например, потоки вероятностей между состояниями 
$\ket{n + 1}$ и $\ket{n}$
(используя принцип детального равновесия). Получим: 
\begin{equation}
\frac{\omega}{Q}\left(n + 1\right)\rho_{n + 1, n + 1} =
\left[A - \left(n + 1\right)B\right]\left(n + 1\right)\rho_{nn}
\label{eqCh3_9}
\end{equation}
Равенство \eqref{eqCh3_9} дает итерационное соотношение
\begin{equation}
\rho_{n + 1, n + 1} = 
\frac{A - \left(n + 1\right)B}{\omega/Q} \rho_{nn}
\label{eqCh3_10}
\end{equation}
которое позволяет выразить $\rho_{nn}$ через $\rho_{00}$.  Имеем
\begin{equation}
\rho_{nn} = \rho_{00}\prod_{k = 1}^n
\frac{A - k B}{\omega/Q} 
\label{eqCh3_11}
\end{equation}
$\rho_{00}$ может быть найдено из условия нормировки
$\sum_{(n)}\rho_{nn} = 1$.

Рассмотрим качественно при помощи \eqref{eqCh3_11} $\rho_{nn}$ как
функцию $n$, когда лазер работает выше порога, т. е. когда $A >
\frac{\omega}{Q}$. В стационарном режиме усиление равно потерям (в среднем), то есть 
\[
A - \bar{n}_{st}B = \frac{\omega}{Q},
\]
откуда среднее число фотонов в стационарном режиме равно
\[
\bar{n}_{st} = \frac{A - \omega/Q}{B}.
\] 

Для $n \ll \bar{n}_{st}$, $B n \ll A$ формулу \eqref{eqCh3_11} можно
представить в виде 
\[
\rho_{nn} \approx \left(\frac{A Q}{\omega}\right)^n.
\]
Эта величина возрастает с ростом $n$ эксоненциально, так как $\frac{A Q}{\omega} >
1$, то есть при малых $n$ с ростом $n$ вероятность растет, но рост 
замедляется по мере увеличения $n$.  Для больших $n$ сомножитель
$\frac{A - nB}{\frac{\omega}{Q}}$
приближается к единице, когда $n$ приближается к $\bar{n}_{st}$. 
При $n \approx \bar{n}_{st}$ сомножитель становится равным  1,  и
распределение достигает максимума: 
\[
\left.\frac{A - B k}{\omega/Q}\right|_{k = \bar{n}_{st}}
\rightarrow \frac{A - B \bar{n}_{st}}{\omega/Q} = 
\frac{A - A + \omega/Q}{\omega/Q} = 1.
\] 

При $n > \bar{n}_{st}$ и $k > \bar{n}_{st}$ сомножители уменьшаются,
достигая нуля при $k = \frac{A}{B}$. Заметим, 
что для $k = \frac{A}{B}$ сомножитель становится приблизительно равным
нулю, и члены с большими значениями $k$ следует отбросить, так как
$\rho_{nn}$ не может быть отрицательно. Возникшая трудность связана с
тем, что мы пользовались теорией возмущений и, следовательно,
полагали, что число фотонов не слишком велико. Наша теория
справедлива, пока $n < \frac{A}{B}$.  Эти трудности 
исчезают, если пользоваться теорией большого сигнала, развитой в
\cite{bScally1974}. Из сказанного следует, что $\rho_{nn}$ как
функция  $n$ (распределение фотонов в лазерном поле) сперва растет,
достигая максимума вблизи $n = \bar{n}_{st}$, а затем убывает до нуля.   

\input ./part2/laser/fig3.tex

На \autoref{figPart2Ch1_3} качественно изображена кривая распределения
фотонов для 
случая превышения порога. На пороге $A = \frac{\omega}{Q}$.  Тогда
сомножитель $\frac{A - B k}{\omega/Q} = 1 - \left(\frac{B}{A}\right)k$
при любом $k$ будет меньше единицы. Кривая распределения будет монотонно
убывать. Ниже порога $A < \frac{\omega}{Q}$, тогда $\frac{A - B
  k}{\omega/Q} \approx \frac{A Q}{\omega} \ll 1$ и $\rho_{nn}$
экспоненциально затухает, как $\left(\frac{A
  Q}{\omega}\right)^n$. Зависимость $\rho_{nn}$ от $n$ для всех трех
случаев может быть численно рассчитана по формуле
\eqref{eqCh3_11}. Результаты такого расчета представлены на
\autoref{figPart2Ch1_4}. 

\input ./part2/laser/fig4.tex

\input ./part2/laser/fig5.tex

Из теории большого сигнала \cite{bScally1974}, которую мы здесь
не рассматриваем, следует, что при очень большом превышении порога 
$\rho_{nn} \rightarrow e^{-\bar{n}}\frac{\bar{n}^n}{n!}$, то есть стремится 
к распределению, характерному для когерентного состояния. При
умеренном превышении порога распределение фотонов в лазерном излучении 
заметно отличается от распределения фотонов в когерентном
состоянии. Отличие распределения фотонов в лазерном поле от
распределения Пуассона 
\rindex{распределение Пуассона}
(соответствующего когерентному состоянию) при
умеренной накачке наглядно представлено на \autoref{figPart2Ch1_5}.
