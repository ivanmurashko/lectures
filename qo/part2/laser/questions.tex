%% -*- coding:utf-8 -*-
\section{Упражнения}
\begin{enumerate}
\item Вывести \eqref{eqCh3_4} из \eqref{eqCh3_3}.
%\item Получить выражение \eqref{eqCh3_8} для $\left<n\right>$.
\item Из \eqref{eqCh3_5} получить выражение для диагональных элементов
  матрицы плотности \eqref{eqCh3_6}.
\item Из \eqref{eqCh3_5} получить выражение для оператора плотности в
  представлении когерентных состояний \eqref{eqCh3_7}.
\item Представить уравнение \eqref{eqCh3_7} в полярных координатах.
\item Получить уравнение \eqref{eqCh3_17} из общего уравнения
  \eqref{eqCh3_12} .
\item Воспользовавшись \label{qLaserBandwidth} формулами
  \eqref{eqCh3_21a} и \eqref{eqCh3_22}, 
  оценить ''естественную'' ширину линии генерации лазера для $A
  \approx \frac{\omega}{Q}$ (малое превышение порога), 
$\frac{\omega}{Q} = \Delta \omega = 10^6 \mbox{Гц}$, 
$n_{st} = 10^6 - 10^7$ фотонов в моде. 
\end{enumerate}
