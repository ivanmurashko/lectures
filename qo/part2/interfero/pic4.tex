%%Created by jPicEdt 1.4.1_03: mixed JPIC-XML/LaTeX format
%%Mon Mar 10 19:47:01 MSK 2008
%%Begin JPIC-XML
%<?xml version="1.0" standalone="yes"?>
%<jpic x-min="10" x-max="100" y-min="0" y-max="65" auto-bounding="true">
%<multicurve stroke-width= "0.5"
%	 fill-style= "none"
%	 points= "(40,30);(40,30);(20,10);(20,10)"
%	 />
%<multicurve fill-style= "none"
%	 right-arrow= "head"
%	 points= "(30,0);(30,0);(30,40);(30,40)"
%	 />
%<multicurve fill-style= "none"
%	 points= "(30,50);(30,50);(30,40);(30,40)"
%	 />
%<text text-vert-align= "center-v"
%	 anchor-point= "(50,10)"
%	 text-hor-align= "center-h"
%	 text-frame= "noframe"
%	 fill-style= "none"
%	 >
%
%</text>
%<multicurve stroke-width= "0.5"
%	 fill-style= "none"
%	 points= "(100,60);(100,60);(80,40);(80,40)"
%	 />
%<multicurve stroke-width= "0.5"
%	 fill-style= "none"
%	 points= "(100,30);(100,30);(80,10);(80,10)"
%	 />
%<multicurve stroke-width= "0.5"
%	 fill-style= "none"
%	 points= "(40,60);(40,60);(20,40);(20,40)"
%	 />
%<multicurve fill-style= "none"
%	 points= "(20,40);(20,40);(20,45);(20,45)"
%	 />
%<multicurve fill-style= "none"
%	 points= "(25,45);(25,45);(25,50);(25,50)"
%	 />
%<multicurve fill-style= "none"
%	 points= "(30,50);(30,50);(30,55);(30,55)"
%	 />
%<multicurve fill-style= "none"
%	 points= "(35,55);(35,55);(35,55);(35,55)"
%	 />
%<multicurve fill-style= "none"
%	 points= "(35,55);(35,55);(35,60);(35,60)"
%	 />
%<multicurve fill-style= "none"
%	 points= "(35,60);(35,60);(35,60);(35,60)"
%	 />
%<multicurve fill-style= "none"
%	 points= "(40,60);(40,60);(40,60);(40,60)"
%	 />
%<multicurve fill-style= "none"
%	 points= "(40,60);(40,60);(40,65);(40,65)"
%	 />
%<multicurve fill-style= "none"
%	 points= "(80,10);(80,10);(80,5);(80,5)"
%	 />
%<multicurve fill-style= "none"
%	 points= "(85,15);(85,15);(85,10);(85,10)"
%	 />
%<multicurve fill-style= "none"
%	 points= "(90,20);(90,20);(90,15);(90,15)"
%	 />
%<multicurve fill-style= "none"
%	 points= "(95,25);(95,25);(95,20);(95,20)"
%	 />
%<multicurve fill-style= "none"
%	 points= "(100,30);(100,30);(100,25);(100,25)"
%	 />
%<multicurve fill-style= "none"
%	 right-arrow= "head"
%	 points= "(30,50);(30,50);(100,50);(100,50)"
%	 />
%<multicurve fill-style= "none"
%	 right-arrow= "head"
%	 points= "(90,20);(90,20);(90,60);(90,60)"
%	 />
%<multicurve fill-style= "none"
%	 points= "(60,20);(60,20);(90,20);(90,20)"
%	 />
%<multicurve fill-style= "none"
%	 right-arrow= "head"
%	 points= "(10,20);(10,20);(60,20);(60,20)"
%	 />
%<text text-vert-align= "center-v"
%	 anchor-point= "(10,25)"
%	 text-hor-align= "center-h"
%	 text-frame= "noframe"
%	 fill-style= "none"
%	 >
%$\hat{a}_0$
%</text>
%<text text-vert-align= "center-v"
%	 anchor-point= "(35,5)"
%	 text-hor-align= "center-h"
%	 text-frame= "noframe"
%	 fill-style= "none"
%	 >
%$\hat{a}_1$
%</text>
%<text text-vert-align= "center-v"
%	 anchor-point= "(50,25)"
%	 text-hor-align= "center-h"
%	 text-frame= "noframe"
%	 fill-style= "none"
%	 >
%$hat{a}_2$
%</text>
%<text text-vert-align= "center-v"
%	 anchor-point= "(35,35)"
%	 text-hor-align= "center-h"
%	 text-frame= "noframe"
%	 fill-style= "none"
%	 >
%$\hat{a}_3$
%</text>
%<text text-vert-align= "center-v"
%	 anchor-point= "(100,45)"
%	 text-hor-align= "center-h"
%	 text-frame= "noframe"
%	 fill-style= "none"
%	 >
%$\hat{a}_4$
%</text>
%<text text-vert-align= "center-v"
%	 anchor-point= "(80,60)"
%	 text-hor-align= "center-h"
%	 text-frame= "noframe"
%	 fill-style= "none"
%	 >
%$\hat{a}_5$
%</text>
%<text text-vert-align= "center-v"
%	 anchor-point= "(50,55)"
%	 text-hor-align= "center-h"
%	 text-frame= "noframe"
%	 fill-style= "none"
%	 >
%$e^{i\varphi}$
%</text>
%<text text-vert-align= "center-v"
%	 anchor-point= "(15,5)"
%	 text-hor-align= "center-h"
%	 text-frame= "noframe"
%	 fill-style= "none"
%	 >
%$M_1$
%</text>
%<text text-vert-align= "center-v"
%	 anchor-point= "(100,65)"
%	 text-hor-align= "center-h"
%	 text-frame= "noframe"
%	 fill-style= "none"
%	 >
%$M_2$
%</text>
%</jpic>
%%End JPIC-XML
%LaTeX-picture environment using emulated lines and arcs
%You can rescale the whole picture (to 80% for instance) by using the command \def\JPicScale{0.8}
\ifx\JPicScale\undefined\def\JPicScale{1}\fi
\unitlength \JPicScale mm
\begin{picture}(100,65)(0,0)
\linethickness{0.5mm}
\multiput(20,10)(0.12,0.12){167}{\line(1,0){0.12}}
\linethickness{0.3mm}
\put(30,0){\line(0,1){40}}
\put(30,40){\vector(0,1){0.12}}
\linethickness{0.3mm}
\put(30,40){\line(0,1){10}}
\put(50,10){\makebox(0,0)[cc]{}}

\linethickness{0.5mm}
\multiput(80,40)(0.12,0.12){167}{\line(1,0){0.12}}
\linethickness{0.5mm}
\multiput(80,10)(0.12,0.12){167}{\line(1,0){0.12}}
\linethickness{0.5mm}
\multiput(20,40)(0.12,0.12){167}{\line(1,0){0.12}}
\linethickness{0.3mm}
\put(20,40){\line(0,1){5}}
\linethickness{0.3mm}
\put(25,45){\line(0,1){5}}
\linethickness{0.3mm}
\put(30,50){\line(0,1){5}}
\linethickness{0.3mm}
\linethickness{0.3mm}
\put(35,55){\line(0,1){5}}
\linethickness{0.3mm}
\linethickness{0.3mm}
\linethickness{0.3mm}
\put(40,60){\line(0,1){5}}
\linethickness{0.3mm}
\put(80,5){\line(0,1){5}}
\linethickness{0.3mm}
\put(85,10){\line(0,1){5}}
\linethickness{0.3mm}
\put(90,15){\line(0,1){5}}
\linethickness{0.3mm}
\put(95,20){\line(0,1){5}}
\linethickness{0.3mm}
\put(100,25){\line(0,1){5}}
\linethickness{0.3mm}
\put(30,50){\line(1,0){70}}
\put(100,50){\vector(1,0){0.12}}
\linethickness{0.3mm}
\put(90,20){\line(0,1){40}}
\put(90,60){\vector(0,1){0.12}}
\linethickness{0.3mm}
\put(60,20){\line(1,0){30}}
\linethickness{0.3mm}
\put(10,20){\line(1,0){50}}
\put(60,20){\vector(1,0){0.12}}
\put(10,25){\makebox(0,0)[cc]{$\hat{a}_0$}}

\put(35,5){\makebox(0,0)[cc]{$\hat{a}_1$}}

\put(50,25){\makebox(0,0)[cc]{$\hat{a}_2$}}

\put(35,35){\makebox(0,0)[cc]{$\hat{a}_3$}}

\put(100,45){\makebox(0,0)[cc]{$\hat{a}_4$}}

\put(80,60){\makebox(0,0)[cc]{$\hat{a}_5$}}

\put(50,55){\makebox(0,0)[cc]{$e^{i\varphi}$}}

\put(15,5){\makebox(0,0)[cc]{$M_1$}}

\put(100,65){\makebox(0,0)[cc]{$M_2$}}

\end{picture}
