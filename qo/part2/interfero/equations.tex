%% -*- coding:utf-8 -*- 
\section{Операторные соотношения}
Будем рассматривать падающее и прошедшее поля как двухмерные
см. \autoref{figPart2Interfero_2}. Операторы $\hat{a}_0$ и 
$\hat{a}_1$ - операторы падающего поля, относящиеся к двум разным модам
(нулевой и первой). Операторы прошедшего поля $\hat{a}_2$ и 
$\hat{a}_3$ (вторая и третья моды). Естественно, что операторы должны
удовлетворять коммутационным соотношениям:
\begin{eqnarray}
\left[\hat{a}_0, \hat{a}_0^{\dag}\right] = 
\left[\hat{a}_1, \hat{a}_1^{\dag}\right] = 
\left[\hat{a}_2, \hat{a}_2^{\dag}\right] = 
\left[\hat{a}_3, \hat{a}_3^{\dag}\right] = 1,
\nonumber \\
\left[\hat{a}_0, \hat{a}_1^{\dag}\right] = 
\left[\hat{a}_2, \hat{a}_3^{\dag}\right] = 0.
\label{eqPart2Interfero5}
\end{eqnarray}

\input ./part2/interfero/fig2.tex

Связь между модами возникает при отражении от зеркала и прохождении
через него. Она описывается следующими уравнениями:
\begin{eqnarray}
\hat{a}_2 = t' \hat{a}_0 + r \hat{a}_1,
\nonumber \\
\hat{a}_3 = r' \hat{a}_0 + t \hat{a}_1.
\label{eqPart2Interfero6}
\end{eqnarray}

Падающее и прошедшее поля должны удовлетворять коммутационным
соотношениям \eqref{eqPart2Interfero5}, откуда имеем  с учетом 
\eqref{eqPart2Interfero3}
\begin{eqnarray}
\left[\hat{a}_2, \hat{a}_2^{\dag}\right] = 
\left[t' \hat{a}_0 + r \hat{a}_1, t'^{*} \hat{a}_0^{\dag} + r^{*}
  \hat{a}_1^{\dag}\right] =
\nonumber \\ = 
\left|r\right|^2 + \left|t\right|^2 = 1.
\label{eqPart2InterferoTask2a}
\end{eqnarray}
Аналогично получаем 
\begin{equation}
\left[\hat{a}_3, \hat{a}_3^{\dag}\right] = 1 
\label{eqPart2InterferoTask2b}
\end{equation}
и далее
\begin{equation}
\left[\hat{a}_2, \hat{a}_3^{\dag}\right] = 
\left[t' \hat{a}_0 + r \hat{a}_1, 
r'^{\dag} \hat{a}_0^{\dag} + t^{*} \hat{a}_1^{\dag}\right] = 
t' r'^{*} + r t^{*} = 0.
\label{eqPart2InterferoTask2c}
\end{equation}
Таким образом мы получили что выполнение всех требуемых коммутационных
соотношений. Заметим что если возбуждена только мода 1 (вход 1), то
поле нулевой моды (вход 0) нельзя отменить как это делается в
классическом случае. В квантовом случае мода находится в вакуумном
состоянии, и ее поле не равно нулю. Если это не учитывать, то
коммутационные соотношения \eqref{eqPart2Interfero5} будут нарушены.

\input ./part2/interfero/fig3.tex

Положим, что на зеркало (делитель световых пучков), изображенное на
\autoref{figPart2Interfero_3}, подается на вход 1 (первая мода) -
когерентное состояние $\left|\alpha\right>$, а на нулевой вход
(нулевая мода) вакуумное состояние $\ket{0}$, т. е.
\begin{eqnarray}
\hat{a}_0 \ket{0} = 0,
\nonumber \\
\hat{a}_1 \left|\alpha\right> = \alpha \left|\alpha\right>.
\nonumber
\end{eqnarray}
Таким образом на входе мы имеем двухмодовое состояние 
\[
\left|\psi\right> = \ket{0}_0 \left|\alpha\right>_1.
\]
Из \eqref{eqPart2Interfero6} следует 
\begin{equation}
\hat{a}_3\left|\psi\right> = 
\left(r' \hat{a}_0 + t \hat{a}_1\right)\ket{0}_0
\left|\alpha\right>_1 = 
t \alpha \left|\psi\right>,
\nonumber
\end{equation}
откуда видно, что состояние на выходе 3 является когерентным состоянием
с ослабленной амплитудой $\alpha_3 = t \alpha$. 
Аналогичным образом получим
\begin{equation}
\hat{a}_2\left|\psi\right> = 
r \alpha \left|\psi\right>, \, \alpha_2 = r\alpha.
\label{eqPart2InterferoTask3}
\end{equation}
Таким образом изменение амплитуды $\alpha$ получается таким же как в
классическом случае, а состояние остается когерентным.

