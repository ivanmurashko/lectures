%% -*- coding:utf-8 -*- 
\input preamble.tex

\begin{document}
\Russian
\input title.tex

\section*{I семестр}

\subsection*{Билет 1} 
\begin{enumerate}
\item Разложение электромагнитного поля по модам (типам колебаний).
Гамильтонова форма уравнений электромагнитного поля. Квантование
электромагнитного поля. 
\item Общая теория взаимодействия динамической системы с
  термостатом (диссипативной системой, резервуаром)
\end{enumerate}

\subsection*{Билет 2} 
\begin{enumerate}
\item Разложение поля по плоским волнам в свободном пространстве. 
Плотность состояний. Гамильтонова форма уравнений поля при разложении по плоским
волнам. Квантование электромагнитного поля при разложении его по
плоским волнам.
\item Релаксация динамической системы. Метод матрицы плотности. 
\end{enumerate}

\subsection*{Билет 3} 
\begin{enumerate}
\item Свойства операторов $ \hat a $ и $ \hat a ^+ $. Квантовое
состояние электромагнитного поля  с определенной энергией. 
\item Затухание
(релаксация) поля и атома в случае простейшего резервуара, состоящего из гармонических осцилляторов
\end{enumerate}

\subsection*{Билет 4} 
\begin{enumerate}
\item Многомодовые состояния. 
\item Излучение и поглощение атомом света. 
Гамильтониан системы атом-поле
\end{enumerate}

\subsection*{Билет 5} 
\begin{enumerate}
\item Когерентные состояния. 
\item Взаимодействие электромагнитного поля резонатора
  (гармонического осциллятора) с резервуаром атомов, находящихся при
  температуре $T$. Уравнение для матрицы плотности поля в представлении чисел
  заполнения. Уравнение движения статистического оператора поля моды в
  представлении когерентных состояний
\end{enumerate}

\subsection*{Билет 6} 
\begin{enumerate}
\item Смешанные состояния электромагнитного поля. 
\item Взаимодействие
атома с модой электромагнитного поля. 
\end{enumerate}

\subsection*{Билет 7} 
\begin{enumerate}
\item Представление оператора плотности через когерентные
  состояния.
\item Взаимодействие атома с многомодовым полем. Вынужденные и
  спонтанные переходы.
\end{enumerate}

\section*{II семестр}

\subsection*{Билет 1} 
\begin{enumerate}
\item Модель лазера
\item Когерентные свойства света.
 Когерентность второго порядка.
 Когерентность высших порядков.
\end{enumerate}

\subsection*{Билет 2} 
\begin{enumerate}
\item Теория лазерной генерации
\item Счет и статистика фотонов
\end{enumerate}

\subsection*{Билет 3} 
\begin{enumerate}
\item Статистика лазерных фотонов
\item Когерентные свойства света.
 Когерентность второго порядка.
 Когерентность высших порядков.
\end{enumerate}

\subsection*{Билет 4} 
\begin{enumerate}
\item Теория лазера. Представление когерентных состояний
\item Фотоэффект
\end{enumerate}

\subsection*{Билет 5} 
\begin{enumerate}
\item Статистика лазерных фотонов
\item Счет и статистика фотонов.
Связь статистики фотонов со статистикой фотоотсчетов.
Распределение фотоотсчетов для когерентного и хаотического
  света.
\end{enumerate}

\subsection*{Билет 6} 
\begin{enumerate}
\item Модель лазера
\item Квантовое выражение для распределения фотоотсчетов.
\end{enumerate}

\subsection*{Билет 7} 
\begin{enumerate}
\item Теория лазерной генерации
\item Эксперименты по счету фотонов. Применение техники счета
  фотонов для спектральных измерений.
\end{enumerate}

\end{document}
