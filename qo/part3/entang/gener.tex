%% -*- coding:utf-8 -*- 
\section{Получение Белловских состояний}
\label{subsecPart3NonclassEntanglBelGener}

Для получения перепутанных состояний может быть использован процесс
параметрического рассеяния света, рассмотренный нами ранее для
описания процесса получения сжатых состояний электромагнитного
поля (см. \ref{pNonClassGenerSqueezed}). Тогда нами рассматривался
фазовый синхронизм I рода при котором поляризации двух фотонов
одинаковы. При фазовом синхронизме второго рода поляризации фотонов
различны. При этом фотоны распространяются вдоль двух конусов, как
показано на рис. \ref{figEntangGen}. 

\input ./part3/entang/figgen1.tex

Вдоль одного конуса излучение поляризовано
как обыкновенные волны, а вдоль второго как необыкновенные. Таким
образом в точках пересечения конусов (см. рис. \ref{figEntangGen2})
состояние света записывается следующим образом
\begin{equation}
\left|\psi\right> =
  \frac{1}{\sqrt{2}}\left(
  \left|x\right>_1\left|y\right>_2 + e^{i \alpha}
  \left|y\right>_1\left|x\right>_2
  \right),
\nonumber
\end{equation}
где разность фаз $\alpha$ возникает из за разности показателей
преломления для обыкновенного и необыкновенного фотонов. Используя
дополнительную двулучепреломляющую фазовую пластинку можно получить
любое значение разности фаз
$\alpha$, например $0$ или $\pi$. В результате мы получим следующие белловские состояния:
\begin{eqnarray}
  \left|\psi^{+}\right>_{12} = 
  \frac{1}{\sqrt{2}}\left(
  \left|x\right>_1\left|y\right>_2 + 
  \left|y\right>_1\left|x\right>_2
  \right),
  \nonumber \\
  \left|\psi^{-}\right>_{12} = 
  \frac{1}{\sqrt{2}}\left(
  \left|x\right>_1\left|y\right>_2 - 
  \left|y\right>_1\left|x\right>_2
  \right).
  \label{eqEntangBellBaseGen1}
\end{eqnarray}

\input ./part3/entang/figgen2.tex

Теперь если перед одним из пучков, например первым, поместить полуволновую фазовую
пластинку, которая меняет поляризацию на ортогональную:
\begin{equation}
\left|x\right>_2 \rightarrow \left|y\right>_2, \, \left|y\right>_2 \rightarrow \left|x\right>_2,
\nonumber
\end{equation}
то из (\ref{eqEntangBellBaseGen1}) мы получим
\begin{eqnarray}
  \left|\phi^{+}\right>_{12} = 
  \frac{1}{\sqrt{2}}\left(
  \left|x\right>_1\left|x\right>_2 + 
  \left|y\right>_1\left|y\right>_2
  \right),
  \nonumber \\
  \left|\phi^{-}\right>_{12} = 
  \frac{1}{\sqrt{2}}\left(
  \left|x\right>_1\left|x\right>_2 - 
  \left|y\right>_1\left|y\right>_2
  \right).
  \label{eqEntangBellBaseGen2}
\end{eqnarray}

Таким образом, объединив (\ref{eqEntangBellBaseGen1}) и
(\ref{eqEntangBellBaseGen2}), мы получим полный базис (\ref{eqEntangBellBase}).



