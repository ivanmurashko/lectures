%% -*- coding:utf-8 -*- 
\section{Неравенство Белла для параметров Стокса}
\label{pPart3EntangleBell}
Ответ на вопрос о полноте квантовой механики может дать следующий
эксперимент \cite{bBell, bPhisQuantInfo, PhysRevLett.28.938}.

\rindex{фотон}Допустим, что рассматриваемая система состоит из двух фотонов с общей
волновой функцией вида
\begin{equation}
\left|\psi\right> = \frac{
\ket{x}_1\ket{y}_2 -
\ket{y}_1\ket{x}_2
}{\sqrt{2}}.
\label{eqBellInequalityState}
\end{equation}

Пусть детектор $D^{(1)}$ (\autoref{figEntangMes}) измеряет величину 
\[
\hat{A} = \hat{S}_1^{(1)}, \, (\xi = 0)
\] 
или
\[
\hat{A}' = \hat{S}_2^{(1)},\,(\xi =\frac{\pi}{4}).
\]
Как уже отмечалось ранее, собственными числами операторов $\hat{A}$ и
$\hat{A}'$ будут $\pm 1$, т. е. показаниями детектора $D^{(1)}$ могут
быть только два числа: $\pm 1$.

Приемник $D^{(2)}$ будет измерять следующие величины 
\[
\hat{B} = \frac{1}{\sqrt{2}}\left(\hat{S}_1^{(2)} + \hat{S}_2^{(2)}\right),\,(\xi =
\frac{\pi}{8}) 
\]
или 
\[
\hat{B}' = \frac{1}{\sqrt{2}}\left(\hat{S}_1^{(2)} - \hat{S}_2^{(2)}\right),\,(\xi =
- \frac{\pi}{8}).
\]
Для ответа на вопрос о возможных показаниях $D^{(2)}$ необходимо найти
собственные числа операторов $\hat{B}$ и $\hat{B}'$. Используя
матричное представление операторов $\hat{S}_1$
\eqref{eqEntangS1Matrix} и $\hat{S}_2$ \eqref{eqEntangS2Matrix},
получим для оператора $\hat{B}$ следующее матричное представление:
\begin{equation}
\hat{B} = 
\left(
\begin{array}{cc}
\frac{1}{\sqrt{2}} & \frac{1}{\sqrt{2}} \\
\frac{1}{\sqrt{2}} & -\frac{1}{\sqrt{2}} 
\end{array}
\right).
\label{eqEntangBMatrix}
\end{equation}
Из \eqref{eqEntangBMatrix} можно получить характеристическое
уравнение:
\begin{equation}
\left(1 -\sqrt{2} b\right)\left(- 1 -\sqrt{2} b\right) -1 = 0,
\label{eqEntangBCharakterEq}
\end{equation}
откуда для собственных чисел имеем: $b = \pm 1$. В случае оператора
$\hat{B}'$  можно получить из \eqref{eqEntangS1Matrix} и
\eqref{eqEntangS2Matrix} следующее матричное преставление:
\begin{equation}
\hat{B}' = 
\left(
\begin{array}{cc}
\frac{1}{\sqrt{2}} & -\frac{1}{\sqrt{2}} \\
-\frac{1}{\sqrt{2}} & -\frac{1}{\sqrt{2}} 
\end{array}
\right).
\label{eqEntangBaddMatrix}
\end{equation}
и соответствующее ему характеристическое уравнение
\begin{equation}
\left(1 -\sqrt{2} b\right)\left(- 1 -\sqrt{2} b\right) -1 = 0.
\label{eqEntangBaddCharakterEq}
\end{equation}
Как нетрудно заметить, уравнение \eqref{eqEntangBaddCharakterEq} имеет
те же самые решения, что и \eqref{eqEntangBCharakterEq}, таким образом
показаниями $D^{(2)}$, так же как и для $D^{(1)}$, будут только два
значения $\pm 1$.

Проводятся 4 серии экспериментов по $N$ испытаний в каждой, в которых
измеряются пары операторов $\left(\hat{A},\hat{B}\right)$,
$\left(\hat{A}',\hat{B}\right)$, $\left(\hat{A},\hat{B}'\right)$ и
$\left(\hat{A}',\hat{B}'\right)$. В результате получаются следующие
наборы чисел $\left(a_i, b_i\right)$, $\left(a_i', b_i\right)$, $\left(a_i, b_i'\right)$ и
$\left(a_i', b_i'\right)$.
\footnote{
  Стоит отметить, что в классическом случае все эти величины могут
  быть измерены одновременно и понимаются здесь именно в том смысле
  как будто они получены в рамках одного эксперимента. Связано это с
  тем что измерения величин $A, A'$ производятся в одной точке
  пространства отличной от места измерения $B, B'$ и соответственно
  с классической точки зрения являются независимыми. 
  В квантовом случае это не совсем так,
  поскольку $\left[\hat{A}, \hat{A}'\right] = 2 i \hat{S_3^{(1)}}$ и
  $\left[\hat{B}', \hat{B}\right] =
  \left[\hat{S_1^{(2)}}, \hat{S_2^{(2)}}\right] =
  2 i \hat{S_3^{(2)}} \ne 0$ откуда, с учетом 
  \autoref{AddHeisenbergUncertaintyPrincipleMesuranmet}, следует что
  рассматриваемые измерения ($A, A'$ с одной стороны и $B, B'$ с
  другой) не являются независимыми.  
  Несмотря на это в дальнейшем, при рассмотрении классического случая,
  мы будем предполагать, что при каждом измерении $A$ или $A'$ с ним
  одновременно измеряются две величины $B$ и $B'$.
}
Как только что было показано, каждое
из полученных чисел может быть или $+1$ или $-1$.

В дальнейшем из этих пар вычисляется следующее значение
\begin{equation}
f_i = \frac{1}{2}\left(
a_i b_i + a_i' b_i + a_i b_i' - a_i' b_i'
\right)
\nonumber
\end{equation}
для которого вычисляется среднее
\begin{equation}
\left<F\right>_N = \frac{1}{N}\sum_i f_i.
\label{eqEntangFmain}
\end{equation}
При $N \rightarrow \infty$ можно принять
\begin{equation}
\left<F\right>_N \rightarrow \left<F\right>.
\nonumber
\end{equation}

Квантово механический подход дает следующее выражение для среднего в
рассматриваемом состоянии \eqref{eqBellInequalityState} 
\begin{eqnarray}
 \left<F\right> \sim \left<F\right>_{quant} 
=\frac{1}{2}
\left<\psi\right|
\hat{A}\hat{B} + \hat{A}'\hat{B} + \hat{A}\hat{B}' - \hat{A}'\hat{B}'
\left|\psi\right> = 
\nonumber \\
=\frac{1}{2}
\left<\psi\right|
\hat{A}\left(\hat{B} + \hat{B}'\right) + \hat{A}' \left(\hat{B}  -
\hat{B}' \right)
\left|\psi\right> = 
\nonumber \\
= \frac{1}{\sqrt{2}}
\left<\psi\right|
\hat{S}_1^{(1)}\hat{S}_1^{(2)} + \hat{S}_2^{(1)}\hat{S}_2^{(2)}
\left|\psi\right> =
\nonumber \\
= \frac{1}{\sqrt{2}}
\left(-1 - 1\right) = - \sqrt{2}.
\label{eqEntangQuant}
\end{eqnarray}

Если же принять тезис о неполноте квантовой механики, то мы
предполагаем, что поляризация фотонов, а следовательно и параметры
Стокса, определены уже сразу после выхода двух фотонов из источника
$S$.  
Таким образом существуют априорные значения $a$, $a'$, $b$ и $b'$,
а свойства источника могут быть описаны с помощью
классической теории вероятностей.
Таким образом существует 16 элементарных
вероятностей $p\left(a,b,a',b'\right)$, так что среднее значение может
быть записано следующим образом
\begin{equation}
 \left<F\right> \sim \left<F\right>_{class} 
=\sum_{a,b,a',b'=\pm 1} 
p\left(a,b,a',b'\right) f\left(a,b,a',b'\right),
\label{eqEntangClassFuncPre}
\end{equation}
где 
\begin{eqnarray}
 f\left(a,b,a',b'\right) = \frac{1}{2} 
\left(
ab + a'b + ab' - a'b'
\right) = 
\nonumber \\
=
\frac{1}{2} 
\left(
a \left(b + b'\right) + a' \left(b - b'\right)
\right).
\label{eqEntangClassFunc}
\end{eqnarray}

Функция \eqref{eqEntangClassFunc} может принимать только два значения
$f = \pm 1$. Действительно возможны два варианта: $b = b'$ или $b = - b'$. В
первом случае 
\[
f = \frac{1}{2}\left(2ab\right) = \pm 1.
\]
Во втором случае также 
\[
f = \frac{1}{2}\left(2a'b\right) = \pm 1.
\]
Таким образом функция $f$ может принимать значения в следующем
интервале $f_{min} = -1 \le f \le f_{max} = +1$ (в отличии от измеряемой величины
$f_i = 0, \pm 1, \pm 2$). Очевидно, что в классическом случае
\begin{equation}
\left|\left<F\right>_{class} \right| 
\le 1.
\label{eqEntangClass}
\end{equation}

Сравнивая выражения \eqref{eqEntangQuant} и \eqref{eqEntangClass},
которые называют неравенствами Белла,
можно увидеть, что квантовые корреляции имеют большее количественное
значение. Это количественное различие может быть проверено в
эксперименте. Первый эксперимент был проведен в 1972 году
\cite{PhysRevLett.28.938} и его результаты свидетельствуют о
полноте квантового описания. Эксперимент по проверке полноты квантовой
механики может иметь не только теоретический интерес, но и чисто
практическое применение, например в квантовой криптографии, которую мы
рассмотрим ниже (см. \ref{subsecPart3QuantInfoQuantCrypto}). 

Следует отметить, что выражение \eqref{eqEntangQuant} может быть
получено с помощью формулы  \eqref{eqEntangClassFuncPre}, при этом
очевидно необходимо принять, что некоторые из вероятностей
$p\left(a,b,a',b'\right) < 0$. Таким образом мы можем говорить о
неклассичности перепутанного состояния.  
