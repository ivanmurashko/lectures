%% -*- coding:utf-8 -*-
\section{Квантовые псевдотелепатические игры}

Неклассичность перепутанных состояний может быть ярко
продемонстрирована в так называемых квантовых псевдотелепатических играх
\cite{bPseudoTelepathy2003,bPseudoTelepathy2004}

\subsection{Описание игры}
Рассмотрим игру Мермина-Переса (Mermin-Peres) в которой два игрока,
Алиса и Боб, играют против казино. 
Алиса и Боб заполняют числами $\pm 1$ квадрат $3 \times 3$.
Казино говорит Алисе номер строки которую Алиса должна заполнить, а
Бобу номер столбца. У Алисы произведение всех цифр должно быть равным
$+1$, а у Боба $-1$. Игроки выигрывают если число на пересечении
строки и столбца совпало. Проигрывают в противном случае.

Алиса и Боб изолируются друг от друга до того момента как они получат
от казино номер строки и столбца, соответственно. Т. о. договорится о
стратегии игры они могут только до этого момента.

\begin{example}[Игра Мермина-Переса]
Допустим Алиса ($A$) получила номер строки 1, а Боб ($B$) номер
столбца 3. В этом случае выигрышная комбинация может выглядеть
следующим образом  
\begin{eqnarray}
A = \left(
\begin{array}{ccc}
+1 & -1 & -1 \\
\ast & \ast & \ast \\
\ast & \ast & \ast  
\end{array}
\right),
\nonumber \\
B = \left(
\begin{array}{ccc}
\ast & \ast & -1 \\
\ast & \ast & +1 \\
\ast & \ast & +1  
\end{array}
\right).
\nonumber 
\end{eqnarray}
Комбинация является выигрышной потому что и у Алисы и у Боба на
пересечении выбранных строки и столбца стоит одно и тоже число $-1$. 


В противном случае (строка 1, столбец 1) 
\begin{eqnarray}
A = \left(
\begin{array}{ccc}
+1 & -1 & -1 \\
\ast & \ast & \ast \\
\ast & \ast & \ast  
\end{array}
\right),
\nonumber \\
B = \left(
\begin{array}{ccc}
-1 & \ast & \ast \\
+1 & \ast & \ast \\
+1 & \ast & \ast  
\end{array}
\right),
\nonumber 
\end{eqnarray}
игроки проиграют, поскольку Алиса получает на пересечении $+1$, а Боб
$-1$. 
\end{example}


\subsection{Классическая стратегия}
В классической стратегии Алиса и Боб не смогут договориться о всех
элементах матрицы поскольку условия Боба и Алисы будут конфликтовать
друг с другом: по условиям Алисы произведение всех элементов матрицы
будет равняться $(+1)^3 = 1$, а по условиям Боба - $(-1)^3 = -1$.
Т. о. иногда Алиса и Боб выигрывают, но всегда будут игры в которых
они проиграют, поскольку вероятность выигрыша $p < 1$.

\subsection{Квантовая стратегия}
Как мы выяснили классической стратегии не существует, но с другой
стороны, если Алиса и Боб смогут заранее создать некоторое
перепутанное состояние и договориться о тех измерениях которые они
будут проводить над этим состоянием
\footnote{
естественно измерения проводятся Алисой и Бобом отдельно друг от
друга, каждым в своей лаборатории 
}, то существует способ обеспечить выигрыш с вероятностью $p = 1$.

Перепутанное состояние используемое Алисой и Бобом:
\begin{equation}
\ket{\psi} = \ket{\psi}_{A_1,B_1} \otimes \ket{\psi}_{A_2,B_2}
\nonumber
\end{equation}

Матрица используемая для измерений
\begin{equation}
X =
\left(
\begin{array}{ccc}
\hat{S}_0 \otimes \hat{S}_3 & \hat{S}_3 \otimes \hat{S}_0 & \hat{S}_3 \otimes \hat{S}_3 \\
\hat{S}_1 \otimes \hat{S}_0 & \hat{S}_0 \otimes \hat{S}_1 & \hat{S}_1 \otimes \hat{S}_1 \\
- \hat{S}_1 \otimes \hat{S}_3 & - \hat{S}_3 \otimes \hat{S}_1 & \hat{S}_2 \otimes \hat{S}_2  
\end{array}
\right)
\label{eq:quantumgames:matrix}
\end{equation}

В силу того, что 
\[
\hat{S}_1^2 = \hat{S}_2^2 = \hat{S}_3^2 = \hat{S}_0^2 = \hat{I}
\]
для каждого элемента $\hat{x}_{ij}$ матрицы
\eqref{eq:quantumgames:matrix} имеем $\hat{x}_{ij}^2 = \hat{I}$ и
следовательно при измерении мы получаем в качестве значений
собственные числа $\pm 1$.

Кроме этого Алиса и Боб могут проводить измерения, т.к. операторы в
одной строке или в одном столбце коммутируют, т.е. могут быть измерены
одновременно. При этом значения которые получает Алиса подчиняются
требованиям налагаемым на строку, а Бобом на столбец.

И наконец, измерения проводимые Алисой и Бобом для совпадающей позиции
также совпадут. Это следует из следующего соотношения
\[
\left(\hat{x}_{ij}^A \otimes \hat{x}_{ij}^B\right) \ket{\psi} = \ket{\psi},
\]
где $\hat{x}_{ij}^A$ измерение Алисы, а $\hat{x}_{ij}^B$ измерение
Боба. Таким образом произведение результатов получаных Алисой и Бобом
будет равно $1$, или же в силу того что результатом может быть только
$\pm 1$, результаты Алисы и Боба совпадут. 


TBD
