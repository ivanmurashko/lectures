%% -*- coding:utf-8 -*- 
\section{Базисные состояния Белла}

\rindex{Базисные состояния Белла}
Любое двухфотонное перепутанное состояние вида \eqref{eqEntang2Common}:
\begin{eqnarray}
\left|\psi\right> =  
c_{xx} \ket{x}_1\ket{x}_2 +
c_{xy} \ket{x}_1\ket{y}_2 +
\nonumber \\
+
c_{yx} \ket{y}_1\ket{x}_2 +
c_{yy} \ket{y}_1\ket{y}_2
\nonumber
\end{eqnarray}
может быть разложено по следующим базисным состояниям Белла:
\begin{eqnarray}
  \left|\psi^{\dag}\right>_{12} = 
  \frac{1}{\sqrt{2}}\left(
  \ket{x}_1\ket{y}_2 + 
  \ket{y}_1\ket{x}_2
  \right),
  \nonumber \\
  \left|\psi^{-}\right>_{12} = 
  \frac{1}{\sqrt{2}}\left(
  \ket{x}_1\ket{y}_2 - 
  \ket{y}_1\ket{x}_2
  \right),
  \nonumber \\
  \left|\phi^{\dag}\right>_{12} = 
  \frac{1}{\sqrt{2}}\left(
  \ket{x}_1\ket{x}_2 + 
  \ket{y}_1\ket{y}_2
  \right),
  \nonumber \\
  \left|\phi^{-}\right>_{12} = 
  \frac{1}{\sqrt{2}}\left(
  \ket{x}_1\ket{x}_2 - 
  \ket{y}_1\ket{y}_2
  \right).
  \label{eqEntangBellBase}
\end{eqnarray}
 
Для доказательства разложимости нам достаточно доказать, что волновые
функции \eqref{eqEntangBellBase} являются ортогональными. Для этого
запишем каждую из этих функций в матричном виде в базисе, образованном
$\ket{x}_1\ket{x}_2$, $\ket{x}_1\ket{y}_2$,
$\ket{y}_1\ket{x}_2$ и
$\ket{y}_1\ket{y}_2$, в результате получим 
\begin{eqnarray}
  \left|\psi^{\dag}\right>_{12} = 
  \frac{1}{\sqrt{2}}
  \left(
  \begin{array}{c}
    0 \\
    1 \\
    1 \\
    0
  \end{array}
  \right),
  \nonumber \\
  \left|\psi^{-}\right>_{12} = 
  \frac{1}{\sqrt{2}}
  \left(
  \begin{array}{c}
    0 \\
    1 \\
    -1 \\
    0
  \end{array}
  \right),
  \nonumber \\
  \left|\phi^{\dag}\right>_{12} = 
  \frac{1}{\sqrt{2}}
  \left(
  \begin{array}{c}
    1 \\
    0 \\
    0 \\
    1
  \end{array}
  \right),
  \nonumber \\
  \left|\phi^{-}\right>_{12} = 
  \frac{1}{\sqrt{2}}
  \left(
  \begin{array}{c}
    1 \\
    0 \\
    0 \\
    -1
  \end{array}
  \right).
  \label{eqEntangBellBaseMatrix}
\end{eqnarray}

В случае проверки ортогональности $\left|\psi^{\dag}\right>_{12}$ и 
$\left|\psi^{-}\right>_{12}$ имеем
\begin{eqnarray}
\left<\psi^{\dag}\right|_{12}\left.\psi^{-}\right>_{12} = 
  \left|\psi^{\dag}\right>_{12} = 
  \frac{1}{\sqrt{2}} \frac{1}{\sqrt{2}}
  \left(
  \begin{array}{cccc}
    0 & 1 & 1 & 0
  \end{array}
  \right)   
  \left(
  \begin{array}{c}
    0 \\
    1 \\
    -1 \\
    0
  \end{array}
  \right) = 
\nonumber \\
=
  \frac{1}{2}\left(1 - 1\right) = 0.
\nonumber 
\end{eqnarray}
Аналогично можно доказать ортогональность остальных состояний
\eqref{eqEntangBellBase}. 

Кроме этого, воспользовавшись \eqref{eqEntangBellBaseMatrix}, можно
показать, что состояния \eqref{eqEntangBellBase} нормированы, например
для $\left|\psi^{\dag}\right>_{12}$ получим 
\begin{eqnarray}
\left<\psi^{\dag}\right|_{12}\left.\psi^{\dag}\right>_{12} = 
  \left|\psi^{\dag}\right>_{12} = 
  \frac{1}{\sqrt{2}} \frac{1}{\sqrt{2}}
  \left(
  \begin{array}{cccc}
    0 & 1 & 1 & 0
  \end{array}
  \right)   
  \left(
  \begin{array}{c}
    0 \\
    1 \\
    1 \\
    0
  \end{array}
  \right) = 
\nonumber \\
=
  \frac{1}{2}\left(1 + 1\right) = 1.
\nonumber 
\end{eqnarray}
