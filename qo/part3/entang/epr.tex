%% -*- coding:utf-8 -*- 
\section{Парадокс ЭПР для параметров
  Стокса и перепутанные состояния}

Рассмотрим состояния с левой и правой круговыми поляризациями:
\begin{eqnarray}
  \left| + \right> = \frac{1}{\sqrt{2}}
  \left(
  \left|x\right> + i \left|y\right>
  \right),
  \nonumber \\
  \left| - \right> = \frac{1}{\sqrt{2}}
  \left(
  \left|x\right> - i \left|y\right>
  \right).
  \nonumber
\end{eqnarray}

Возьмем следующее перепутанное состояние двух фотонов
\begin{eqnarray}
  \left|\psi\right> = \frac{
    \left| + \right>_1\left| - \right>_2 -
    \left| - \right>_1\left| + \right>_2
  }{\sqrt{2}}.
  \nonumber
\end{eqnarray}

Если измерение линейной поляризации 1-ого фотона дает
$\left|x\right>_1$, т. е. при измерении параметров Стокса
$\hat{S}_1^{(1)}$ получаем значение $+1$ для первого фотона, тогда для
второго фотона имеем
\begin{eqnarray}
  P_{\left|x\right>_1}\left|\psi\right> =
  \left|x\right>_1\left<x\right|_1 \left|\psi\right> =
  \nonumber \\
  =
  \left|x\right>_1\left<x\right|_1
  \frac{
    \left( \left|x\right>_1 + i \left|y\right>_1 \right)\left| - \right>_2 -
    \left( \left|x\right>_1 - i \left|y\right>_1 \right)\left| + \right>_2
  }{2} =
  \nonumber \\
  =
  \left|x\right>_1
  \frac{\left| - \right>_2 - \left| + \right>_2}{2} =
  \frac{1}{2\sqrt{2}}\left|x\right>_1 \left(2 i\right)
  \left|y\right>_2 =
  \frac{i}{\sqrt{2}}\left|x\right>_1\left|y\right>_2,
  \nonumber
\end{eqnarray}
т.е. для второго фотона значение параметра Стокса 
$\hat{S}_1^{(2)}$ получаем значение $-1$.

Аналогично если при измерении параметра Стокса $\hat{S}_1^{(1)}$
первого фотона имеем $-1$, то для второго фотона получим
\begin{eqnarray}
  P_{\left|y\right>_1}\left|\psi\right> =
  \left|y\right>_1\left<y\right|_1 \left|\psi\right> =
  \nonumber \\
  =
  \left|y\right>_1\left<y\right|_1
  \frac{
    \left( \left|x\right>_1 + i \left|y\right>_1 \right)\left| - \right>_2 -
    \left( \left|x\right>_1 - i \left|y\right>_1 \right)\left| + \right>_2
  }{2} =
  \nonumber \\
  =
  \left|y\right>_1 i 
  \frac{\left| - \right>_2 + \left| + \right>_2}{2} =
  \frac{i}{2\sqrt{2}}\left|y\right>_1 \left(2\right)
  \left|x\right>_2 =
  \frac{i}{\sqrt{2}}\left|y\right>_1\left|x\right>_2,
  \nonumber
\end{eqnarray}
т. е. для второго фотона значение параметра Стокса 
$\hat{S}_1^{(2)}$ получаем значение $+1$.

При этом в рассматриваемом состоянии среднее значение параметра Стокса
$\left<\hat{S}_3^{(2)}\right>$ имеет вид
\begin{eqnarray}
  \left<\hat{S}_3^{(2)}\right> =
  \left<\psi\right|\hat{S}_3^{(2)}\left|\psi\right> =
  \left<\psi\right|\hat{S}_3^{(2)}\frac{
    \left| + \right>_1\left| - \right>_2 -
    \left| - \right>_1\left| + \right>_2
  }{\sqrt{2}} =
  \nonumber \\
  = \left<\psi\right|\frac{
    \left| + \right>_1\left| - \right>_2 +
    \left| - \right>_1\left| + \right>_2
  }{\sqrt{2}} =
  \nonumber \\
  = \frac{1}{2} \left(
  \left< + \right|_1\left< - \right|_2 -
  \left< - \right|_1\left< + \right|_2
  \right)
  \left(
  \left| + \right>_1\left| - \right>_2 -
  \left| - \right>_1\left| + \right>_2
  \right) = 1.
  \label{eqEntangS3Mean}
\end{eqnarray}

При выводе (\ref{eqEntangS3Mean}) использовались выражения
(\ref{eqEntangS3Eigenvec}). Таким образом из неравенства Гейзенберга
(\ref{eqAddHeisenbergUncertaintyPrinciple}) можно получить что
\[
\Delta s_1^{(2)} \Delta s_2^{(2)} \ge 1,
\]
т. е. первый и второй параметры Стокса для второй частицы не могут
быть измерены одновременно. 
