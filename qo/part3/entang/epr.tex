%% -*- coding:utf-8 -*- 
\section{Парадокс ЭПР для параметров
  Стокса и перепутанные состояния}

Рассмотрим состояния с левой и правой круговыми поляризациями:
\begin{eqnarray}
  \ket{ + } = \frac{1}{\sqrt{2}}
  \left(
  \ket{x} + i \ket{y}
  \right),
  \nonumber \\
  \ket{ - } = \frac{1}{\sqrt{2}}
  \left(
  \ket{x} - i \ket{y}
  \right).
  \nonumber
\end{eqnarray}

Возьмем следующее перепутанное состояние двух фотонов
\begin{eqnarray}
  \left|\psi\right> = \frac{
    \ket{ + }_1\ket{ - }_2 -
    \ket{ - }_1\ket{ + }_2
  }{\sqrt{2}}.
  \nonumber
\end{eqnarray}

\index{фотон}Если измерение линейной поляризации 1-ого фотона дает
$\ket{x}_1$, т. е. при измерении параметров Стокса
$\hat{S}_1^{(1)}$ получаем значение $+1$ для первого фотона, тогда для
второго фотона имеем
\begin{eqnarray}
  P_{\ket{x}_1}\left|\psi\right> =
  \ket{x}_1\bra{x}_1 \left|\psi\right> =
  \nonumber \\
  =
  \ket{x}_1\bra{x}_1
  \frac{
    \left( \ket{x}_1 + i \ket{y}_1 \right)\ket{ - }_2 -
    \left( \ket{x}_1 - i \ket{y}_1 \right)\ket{ + }_2
  }{2} =
  \nonumber \\
  =
  \ket{x}_1
  \frac{\ket{ - }_2 - \ket{ + }_2}{2} =
  \frac{1}{2\sqrt{2}}\ket{x}_1 \left(2 i\right)
  \ket{y}_2 =
  \frac{i}{\sqrt{2}}\ket{x}_1\ket{y}_2,
  \nonumber
\end{eqnarray}
т.е. для второго фотона значение параметра Стокса 
$\hat{S}_1^{(2)}$ получаем значение $-1$.

Аналогично если при измерении параметра Стокса $\hat{S}_1^{(1)}$
первого фотона имеем $-1$, то для второго фотона получим
\begin{eqnarray}
  P_{\ket{y}_1}\left|\psi\right> =
  \ket{y}_1\bra{y}_1 \left|\psi\right> =
  \nonumber \\
  =
  \ket{y}_1\bra{y}_1
  \frac{
    \left( \ket{x}_1 + i \ket{y}_1 \right)\ket{ - }_2 -
    \left( \ket{x}_1 - i \ket{y}_1 \right)\ket{ + }_2
  }{2} =
  \nonumber \\
  =
  \ket{y}_1 i 
  \frac{\ket{ - }_2 + \ket{ + }_2}{2} =
  \frac{i}{2\sqrt{2}}\ket{y}_1 \left(2\right)
  \ket{x}_2 =
  \frac{i}{\sqrt{2}}\ket{y}_1\ket{x}_2,
  \nonumber
\end{eqnarray}
т. е. для второго фотона значение параметра Стокса 
$\hat{S}_1^{(2)}$ получаем значение $+1$.

При этом параметры Стокса для второй частицы  $\hat{S}_1^{(2)}$ и
$\hat{S}_2^{(2)}$ имеют 
разные собственные вектора:
$\ket{x}_2, \ket{y}_2$ (\ref{eq:part2:pol:stocks_s1_1,
  eq:part2:pol:stocks_s1_2}) и
$\frac{1}{\sqrt{2}}\left(\ket{x} \pm
\ket{y}\right)$ (\ref{eq:part2:pol:stocks_s1_2,
  eq:part2:pol:stocks_s2_2}) соответственно.
%% При этом в рассматриваемом состоянии среднее значение параметра Стокса
%% $\left<\hat{S}_3^{(2)}\right>$ имеет вид
%% \begin{eqnarray}
%%   \left<\hat{S}_3^{(2)}\right> =
%%   \left<\psi\right|\hat{S}_3^{(2)}\left|\psi\right> =
%%   \left<\psi\right|\hat{S}_3^{(2)}\frac{
%%     \ket{ + }_1\ket{ - }_2 -
%%     \ket{ - }_1\ket{ + }_2
%%   }{\sqrt{2}} =
%%   \nonumber \\
%%   = \left<\psi\right|\frac{
%%     \ket{ + }_1\ket{ - }_2 +
%%     \ket{ - }_1\ket{ + }_2
%%   }{\sqrt{2}} =
%%   \nonumber \\
%%   = \frac{1}{2} \left(
%%   \bra{ + }_1\bra{ - }_2 -
%%   \bra{ - }_1\bra{ + }_2
%%   \right)
%%   \left(
%%   \ket{ + }_1\ket{ - }_2 -
%%   \ket{ - }_1\ket{ + }_2
%%   \right) = 1.
%%   \label{eqEntangS3Mean}
%% \end{eqnarray}

%% При выводе \eqref{eqEntangS3Mean} использовались выражения
%% \eqref{eqEntangS3Eigenvec}. Таким образом из неравенства Гейзенберга
%% \eqref{eqAddHeisenbergUncertaintyPrinciple} можно получить что
%% \[
%% \Delta s_1^{(2)} \Delta s_2^{(2)} \ge 1,
%% \]
Т. о. первый и второй параметры Стокса для второй частицы не могут
быть измерены одновременно, действительно если бы они могли бы быть
измерены одновременно, то измеренному значению соответствовал бы
некоторый вектор состояния, который был бы собственным для обеих
операторов $\hat{S}_1^{(2)}$ и
$\hat{S}_2^{(2)}$.
