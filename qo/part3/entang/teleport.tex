%% -*- coding:utf-8 -*- 
\section{Теорема о запрете клонирования и квантовая телепортация.}
\label{pPart3EntangleNoClone}
В квантовой механике существует так называемая теорема о запрете
клонирования \cite{bNoClone}. Смысл этой теоремы может быть пояснен на
нашем примере фотона с двумя 
взаимно ортогональными поляризациями, волновая функция которого
задается соотношением \eqref{eqEntangSimpleState}. Теорема о запрете
клонирования  утверждает что невозможно создать прибор, на вход
которого подается частица в состоянии \eqref{eqEntangSimpleState}, 
а на выходе получаются две частицы в этом же состоянии. 

Действительно допустим, что у нас имеется прибор, который производит
клонирование состояний. Действие этого прибора на фотон, поляризованный
по $x$ будет описываться оператором клонирования $\hat{D}$
следующим образом 
\begin{equation}
  \hat{D} \ket{D_I}\ket{x} = \ket{D_{F_x}}\ket{x}\ket{x},
  \nonumber
\end{equation}
где $\ket{D_I}$ волновая функция, описывающая начальное
состояние прибора для клонирования, а  $\ket{D_{F_x}}$ волновая
функция, описывающая состояние прибора после клонирования фотона  
в вертикальной $x$-поляризации. Для фотона, поляризованного по $y$, имеем
\begin{equation}
  \hat{D} \ket{D_I}\ket{y} = \ket{D_{F_y}}\ket{y}\ket{y},
  \nonumber
\end{equation}
где $\ket{D_{F_y}}$ волновая функция, описывающая состояние
прибора после клонирования этого фотона. 

Если подействовать оператором клонирования $\hat{D}$ на фотон в
произвольном состоянии \eqref{eqEntangSimpleState}, получим 
\begin{equation}
  \hat{D} \ket{D_I}\left|\psi\right> = 
  \hat{D} \ket{D_I} \left(\alpha \ket{x} +
  \beta \ket{y}\right) = 
  \alpha \ket{D_{F_x}}\ket{x}\ket{x} +
  \beta \ket{D_{F_y}}\ket{y}\ket{y},
  \nonumber
\end{equation}
откуда никоим образом не получить ожидаемый результат
\begin{equation}
  \hat{D} \ket{D_I}\left|\psi\right> = 
  \ket{D_{F_{xy}}}  \left(\alpha \ket{x} +
  \beta \ket{y}\right)
  \left(\alpha \ket{x} +
  \beta \ket{y}\right).
  \nonumber
\end{equation}

Но если состояние фотона нельзя клонировать, то оказывается, что его
можно передать из одной точки пространства в другую  
(естественно, с разрушением исходного состояния), что
демонстрируют эксперименты по квантовой телепортации. 

\input ./part3/entang/figteleport.tex

Схема протокола квантовой телепортации представлена на 
\autoref{figTeleport}. На этом рисунке Алиса хочет передать Бобу  
состояние фотона 1, которое описывается волновой функцией
\eqref{eqEntangSimpleState}. 
\[
\left|\psi\right>_1 = \alpha \ket{x}_1 +
\beta \ket{y}_1, 
\]
\rindex{фотон!перепутанное состояние}Имеется также источник перепутанных фотонов $S$, который излучает пары
фотонов 2 и 3 в состоянии, описываемом следующей волновой функцией 
\begin{equation}
  \left|\psi\right>_{23} = \left|\psi^{-}\right>_{23} = \frac{1}{\sqrt{2}}\left(
  \ket{x}_2\ket{y}_3 - 
  \ket{y}_2\ket{x}_3
  \right).
  \nonumber
\end{equation}

Алиса смешивает фотоны 1 и 2 на светоделителе $LS$ и в дальнейшем
регистрирует некоторое Белловское состояние с помощью детекторов
$D^{(1,2)}$. Проще всего произвести регистрацию состояния
$\left|\psi^{-}\right>_{12}$, в котором оба фотодетектора должны
сработать одновременно.

Общее состояние трех частиц записывается в виде
\begin{equation}
  \left|\psi\right>_{123} = \left|\psi\right>_1 \left|\psi\right>_{23},
  \nonumber
\end{equation}
который может быть разложен по Белловским состояниям фотонов 1 и 2:
\begin{eqnarray}
\left|\psi\right>_{123} = 
c_{\left|\psi^{\dag}\right>_{12}}\left|\psi^{\dag}\right>_{12} +
c_{\left|\psi^{-}\right>_{12}}\left|\psi^{-}\right>_{12} +
\nonumber \\
+
c_{\left|\phi^{\dag}\right>_{12}}\left|\phi^{\dag}\right>_{12} +
c_{\left|\phi^{-}\right>_{12}}\left|\phi^{-}\right>_{12}.
\label{eqPart3EntangTeleportsepar}
\end{eqnarray}
В разложении \eqref{eqPart3EntangTeleportsepar} нас будет интересовать
коэффициент 
$c_{\left|\psi^{-}\right>_{12}}$, так как именно он будет описывать
состояние третьего фотона при регистрации Алисой Белловского состояния 
$\left|\psi^{-}\right>_{12}$. Для искомого коэффициента получим:
\begin{eqnarray}
  c_{\left|\psi^{-}\right>_{12}} = 
  \left<\psi^{-}\right|_{12} \left.\psi\right>_{123} = 
  \nonumber \\
  =
  \frac{1}{\sqrt{2}}
  \left(
  \bra{x}_1\bra{y}_2 - 
  \bra{y}_1\bra{x}_2
  \right)
  \left(
  \alpha \ket{x}_1 +
  \beta \ket{y}_1
  \right)
  \frac{1}{\sqrt{2}}
  \left(
  \ket{x}_2\ket{y}_3 - 
  \ket{y}_2\ket{x}_3
  \right) = 
  \nonumber \\
  = \frac{1}{2}
  \left(
  \alpha\bra{y}_2 - 
  \beta\bra{x}_2
  \right)
  \left(
  \ket{x}_2\ket{y}_3 - 
  \ket{y}_2\ket{x}_3
  \right) = 
  - \frac{1}{2}  
  \left(
  \alpha \ket{x}_3 +
  \beta \ket{y}_3
  \right).
\nonumber
\end{eqnarray}
Это значит, что всякий раз, когда Алиса регистрирует пару фотонов 1 и 2
в состоянии $\left|\psi^{-}\right>_{12}$, т. е. когда оба детектора
$D^{(1)}$ и $D^{(2)}$ срабатывают одновременно, фотон 3 на стороне
Боба оказывается в состоянии, идентичном состоянию исходного фотона 1,
т. е. произойдет телепортация фотона 1  к Бобу.

%% Волновая функция $\left|\psi\right>_{123}$ может быть разложена по
%% базису Белла \eqref{eqEntangBellBase}.
%% Результат разложения записывается в следующем виде (FIX ME!!! check it)
%% \begin{eqnarray}
%%   \left|\psi\right>_{123} = \left|\psi\right>_1 \left|\psi\right>_{23} = 
%%   \nonumber \\
%%   = \left|\psi^{-}\right>_{12} \frac{\alpha\ket{x}_3 + \beta\ket{y}_3}{2} + 
%%   \nonumber \\
%%   + 
%%   \left|\psi^{\dag}\right>_{12} \frac{- \alpha\ket{x}_3 + \beta\ket{y}_3}{2} +
%%   \nonumber \\
%%   + 
%%   \left|\phi^{\dag}\right>_{12} \frac{- \beta\ket{x}_3 + \alpha\ket{y}_3}{2} +
%%   \nonumber \\
%%   +
%%   \left|\phi^{-}\right>_{12} \frac{\beta\ket{x}_3 + \alpha\ket{y}_3}{2}.
%%   \nonumber
%% \end{eqnarray}

%% Таким образом всякий раз когда Алиса регистрирует пару фотонов 1 и 2 в белловском состоянии
%% $\left|\psi^{-}\right>_{12}$ то фотон 3 на стороне Боба будет находится в состоянии
%% $\alpha\ket{x}_3 + \beta\ket{y}_3$, то есть произойдет как бы телепортация фотона 1 к Бобу.

%% Белловское состояние $\left|\psi^{-}\right>_{12}$, как это уже
%% отмечалось выше, может быть зарегистрировано Алисой по одновременному срабатыванию 
%% фотодетекторов $D^{(1)}$ и $D^{(2)}$

% Стоит отметить два факта
% \begin{itemize}
% \item При данной схеме телепортации передается не материя, а некоторая информация о квантовом объекте.
% \item Передача этой информации мгновенна, но вместе с тем для того чтобы Боб смог узнать о факте телепортации необходим классический канал связи 
%   между Алисой и Бобом \footnote{По этому каналу Алиса говорит Бобу когда у нее сработали оба фотодетектора сообщая тем самым о факте телепортации}.
% \end{itemize}
