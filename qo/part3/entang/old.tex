%% -*- coding:utf-8 -*- 
\section{Перепутанные состояния}
Определение. Пародокс ЭПР. Неравенства Белла. Эксперементальная
проверка. Квантовая телепортация.

\section{Пародокс ЭПР.}

Долгое время в квантовой механике существовало как бы две
части. Первая из них была чисто практической и позволяла вычислять
например спектры излучения атомов и молекул. 

Вторая часть квантовой механики была более философской чем
практической. Основным вопросом на который надо было дать ответ здесь
был следующий - "Что есть квантово-механическое описание мира?" 

Многим маститым ученым таким как Альберт Эйнштейн было сложно принять
квантовую механику с ее вероятностным характером. 

В момент становления квантовой механики (в начале XX века)
существовало две основных интерпретации квантовой механики. Первая из 
них так называемая статистическая интерпретация основным стороником
которой был Альберт Эйнштейн. Согласно статистической интерпретации
состояние отдельной квантовой частицы не имеет смысла (или другими
словами пока нет теории которая бы полностью объясняла это
состояние). Смысл имеет имеет некоторый статистический ансамбль
квантовых частиц для которого можно вычислять вероятности почти также
как в статистической физике. 

Вторая интерпретация квантовой механики носит название копенгагенской
интерпретации. Автором этой трактовки квантовой механики
был Нильс Бор. Согласно этой интерпретации квантово механическое
описание является полным и оно может быть использовано для описания
отдельных частиц.

Очевидно что вторая интерпретация является более общей и включает в
себя первую (обратное не верно). Одной из наиболее известных попыток
Альберта Эйнштейна защитить свою точку зрения и показать неполность
квантовой механики была статья \cite{bEPR}. В данной статье описывается так
называемый парадокс Эйнштейна-Подольского-Розена (ЭПР). 

В дальнейшем мы будем рассматривать ЭПР на примере пары фотонов в перепутанном 
по поляризации состоянии. Волновая функция отдельного фотона может быть 
представлена следующим образом
\begin{equation}
\left|\psi\right> = \alpha \left|V\right> +
\beta \left|H\right>, 
\label{eqEntaglementPSI}
\end{equation}
где $\left|V\right>$ обозначает вертикальную поляризацию а 
$\left|H\right>$ горизонтальную. Когда производится одиночный
акт измерения то измерительный прибор может показать только один из двух 
результатов: фотон в вертикальной или горизонтальной поляризации. При этом происходит 
так называемая редукция волновой функции:
\begin{eqnarray}
\left|\psi\right> \rightarrow \left|V\right> \mbox{ при регистрации вертикальной поляризации},
\nonumber \\
\left|\psi\right> \rightarrow \left|H\right> \mbox{ при регистрации горизонтальной поляризации}.
\nonumber
\end{eqnarray}
Математически процесс редукции выражается воздействием на волновую функцию следующих 
операторов проектирования.\footnote{Более подробно об операторах
  проектирования см. \autoref{AddDiracProjector}}
\begin{eqnarray}
\left|V\right>\left<V\right| \mbox{ для вертикальной поляризации},
\nonumber \\
 \left|H\right>\left<H\right| \mbox{ для горизонтальной поляризации}.
\label{eqEntaglementProjector}
\end{eqnarray}
Это нам дает $\left|V\right>\left<V\right|\left|\psi\right> = \alpha \left|V\right>$ 
при регистрации вертикальной поляризации и
$\left|H\right>\left<H\right|\left|\psi\right> = \beta \left|H\right>$ 
при регистрации горизонтальной поляризации.

Теперь допустим что у нас имеется источник частиц который испускает два фотона поляризация 
у которых взаимно ортогональна. Такая система может быть основана на эффекте спонтанного 
параметрического рассеяния света \cite{bKlishko}. Волновая функция такой составной системы 
может быть представлена в следующем виде
\begin{equation}
\left|\psi\right> = \frac{1}{\sqrt{2}}\left(
\left|V\right>_1\left|H\right>_2 - \left|H\right>_1\left|V\right>_2
\right).
\label{eqEntaglementEntaglement}
\end{equation}
Образовавшись в одном источнике эти два два фотона разлетаются в разных направлениях к двум разным
наблюдателям которых мы следуя традиции назовем Алисой и Бобом. Допустим Алиса регистрирует фотон 1, а 
Боб фотон 2. С стороны Алисы влияние измерительного прибора в единичном акте регистрации поляризации
фотона сводится к воздействию операторов проектирования \eqref{eqEntaglementProjector} на 
волновую функцию \eqref{eqEntaglementEntaglement}. Если на стороне Алисы фотон 1 был зарегистрирован в
состоянии с вертикальной поляризацией $\left|V\right>_1$, то волновая функция \eqref{eqEntaglementEntaglement}
редуцируется к состоянию
\begin{equation}
\left|V\right>_1\left<V\right|_1 
\frac{1}{\sqrt{2}} \left(
\left|V\right>_1\left|H\right>_2 - \left|H\right>_1\left|V\right>_2
\right) = \frac{1}{\sqrt{2}}
\left|V\right>_1\left|H\right>_2.
\nonumber
\end{equation}
Таким образом состояние фотона 2 который регистрируется Бобом становится определенным и равным $\left|H\right>_2$.
Если бы Алиса зарегистрировала фотон 1 в состоянии с горизонтальной поляризацией, то Боб со 100\% вероятностью 
получил бы фотон 2 с вертикальной поляризацией. Таким образом эти два фотона даже разлетевшись на большое 
растояние остаются как бы связанными друг с другом и результаты экспериментов
над одним из них способны влиять на результаты экспериментов над другим. 

Казалось бы, такое предположение противоречит теории относительности, запрещающей распространение сигналов быстрее скорости света. 
В данном же случае возмущение должно распространяться мгновенно, ибо фотоны могут находиться на любом расстоянии друг 
от друга к моменту проведения измерения.

И все-таки противоречия нет. По законам квантовой механики, возмущение, вносимое при измерении, случайно. В этом случае, мгновенная передача возмущения не есть передача сигнала, ибо не может нести информацию.

В самом деле, представим себе, что на двух планетах в разных концах Галактики есть две монетки, выпадающие всегда одинаково. Если запротоколировать результаты всех подбрасываний, а потом сравнить их, то они совпадут. Сами же выпадания случайны, на них никак нельзя повлиять. Нельзя, например, договориться, что орел - это единица, а решка - это ноль, и передавать таким образом двоичный код. Ведь последовательность нулей и единиц будет случайной и на том и на другом "конце провода" и не будет нести никакого смысла.

Получается, что парадоксу есть объяснение, логически совместимое и с теорией относительности, и с квантовой механикой.

\section{Теории скрытых параметров. Неравенства Белла}
Существует несколько альтернативных способов разрешения парадокса ЭПР. Первый из них предложенный коллективом авторов в 
\cite{bEPR} заключается в том что квантово-механическое описание действительности не полное и существуют 
какие-то скрытые параметры учет которых привел бы к устранению парадокса ЭПР. Согласно этой точке зрения 
квантовая механика может быть применена к описанию ансамблей частиц (статистическая интерпретация квантовой механики)
описание же поведения отдельных частиц с помощью квантово-механического аппарата является не точным. 

Влияние скрытых параметров может быть показано на примере игральной кости. Если про игральную кость нам известно только 
то что она приготвлена из однородного материала (без смещенного центра тяжести) то все что мы можем сказать о конкретном 
результате броска - это вероятноть выпадения каждой из граней - $\frac{1}{6}$. Однако если мы примем во внимание 
скрытые параметры этой задачи такие как сила и направление броска, начальное положение кости и другие, то задача становится
обычной динамической задачей решение которой даст вполне определенный ответ на вопрос о том какая из граней кости выпадет.

В 1964 году Дж. Белл показал \cite{bBell} что предсказания квантовой механики для парадокса ЭПР немного отличаются от 
предсказаний весьма широкого класса теорий которые оперируют со скрытыми параметрами. Грубо говоря квантовая механика 
пресказывает более строгую статистическую корреляцию \footnote{Порядка 70\% против 50\% в случае теории скрытых параметров}
между результатами выполнеными на стороне Алисы и Боба чем 
теории скрытых пераметров. Эти отличия имеющие форму неравенств и называемые неравенствами Белла могут быть проверены 
в эксперименте. Такие проверки были выполнены \cite{bBellTest} и показали что верны предсказания квантовой механики. 

FIX ME!!! вывод неравенств белла и схема эксперимента по проверке


%% \section{Квантовая телепортация.}

%% Квантовая механика имеет так называемую теорему о запрете клонирования [FIX ME!!! add citation]. 
%% Смысл этой теоремы может быть пояснен на нашем примере фотона с двумя взаимно ортогональными поляризациями
%% волновая функция которого задается соотношением \eqref{eqEntaglementPSI}. Теорема о запрете клонирования 
%% утверждает что невозможно создать прибор на вход которого подается частица в состоянии \eqref{eqEntaglementPSI}
%% а на выходе получаются две частицы в этом же состоянии. 

%% Действительно допустим у нас имеется прибор который производит клонирование состояний. Действие этого прибора
%% на фотон в вертикальной поляризации будет описываться оператором клонирования $\hat{D}$ следующим образом
%% \begin{equation}
%% \hat{D} \left|D_I\right>\left|V\right> = \left|D_{FV}\right>\left|V\right>\left|V\right>,
%% \nonumber
%% \end{equation}
%% где $\left|D_I\right>$ волновая функция описывающая начальное состояние прибора для клонирования, а 
%% $\left|D_{FV}\right>$ волновая функция описывающая состояние прибора после клонирования фотона 
%% в вертикальной поляризации. Для фотона в горизонтальной поляризации имеем
%% \begin{equation}
%% \hat{D} \left|D_I\right>\left|H\right> = \left|D_{FH}\right>\left|H\right>\left|H\right>,
%% \nonumber
%% \end{equation}
%% где $\left|D_{FH}\right>$ волновая функция описывающая состояние прибора после клонирования фотона 
%% с горизонтальной поляризацией.

%% Если подействовать оператором клонирования $\hat{D}$ на фотон в произвольном состоянии \eqref{eqEntaglementPSI} получим
%% \begin{equation}
%% \hat{D} \left|D_I\right>\left|\psi\right> = 
%% \hat{D} \left|D_I\right> \left(\alpha \left|V\right> +
%% \beta \left|H\right>\right) = 
%% \alpha \left|D_{FV}\right>\left|V\right>\left|V\right> +
%% \beta \left|D_{FH}\right>\left|H\right>\left|H\right>,
%% \nonumber
%% \end{equation}
%% откуда никоим образом не получить ожидаемый результат
%% \begin{equation}
%% \hat{D} \left|D_I\right>\left|\psi\right> = 
%% \left|D_{FVH}\right>  \left(\alpha \left|V\right> +
%% \beta \left|H\right>\right)
%% \left(\alpha \left|V\right> +
%% \beta \left|H\right>\right).
%% \nonumber
%% \end{equation}

%% Но если состояние фотона нельзя клонировать, то оказывается что его можно передать из одной точки простраства в другую 
%% \footnote{естественно с разрушением исходного состояния}, что демонстрируют эксперименты по квантовой телепортации.

%% \input ./part3/figteleport.tex

%% Схема протокола квантовой телепортации представлена на рис \ref{figTeleport}. На этом рисунке Алиса хочет передать Бобу 
%% состояние фотона 1 которое описывается волновой функцией \eqref{eqEntaglementPSI}.
%% \[
%% \left|\psi\right>_1 = \alpha \left|V\right>_1 +
%% \beta \left|H\right>_1, 
%% \]
%% Имеется также источник перепутанных фотонов EPR который излучает пары фотов 2 и 3 в состоянии описаваемом следующей волновой функцией
%% \begin{equation}
%% \left|\psi\right>_{23} = \frac{1}{\sqrt{2}}\left(
%% \left|V\right>_2\left|H\right>_3 - 
%% \left|H\right>_2\left|V\right>_3
%% \right).
%% \nonumber
%% \end{equation}

%% Алиса смешивает фотоны 1 и 2 на светоделителе LD. Общее состояние 3-х частиц описывается с помощью
%% \begin{equation}
%% \left|\psi\right>_{123} = \left|\psi\right>_1 \left|\psi\right>_{23}.
%% \nonumber
%% \end{equation}
%% Волновая функция $\left|\psi\right>_{123}$ может быть разложена по следующему базису который называется состояниями
%% Белла:
%% \begin{eqnarray}
%% \left|\psi^{\dag}\right>_{12} = 
%% \frac{1}{\sqrt{2}}\left(
%% \left|V\right>_1\left|H\right>_2 + 
%% \left|H\right>_1\left|V\right>_2
%% \right),
%% \nonumber \\
%% \left|\psi^{-}\right>_{12} = 
%% \frac{1}{\sqrt{2}}\left(
%% \left|V\right>_1\left|H\right>_2 - 
%% \left|H\right>_1\left|V\right>_2
%% \right),
%% \nonumber \\
%% \left|\phi^{\dag}\right>_{12} = 
%% \frac{1}{\sqrt{2}}\left(
%% \left|V\right>_1\left|V\right>_2 + 
%% \left|H\right>_1\left|H\right>_2
%% \right),
%% \nonumber \\
%% \left|\phi^{-}\right>_{12} = 
%% \frac{1}{\sqrt{2}}\left(
%% \left|V\right>_1\left|V\right>_2 - 
%% \left|H\right>_1\left|H\right>_2
%% \right).
%% \nonumber
%% \end{eqnarray}
%% Результат разложения записывается в следующем виде (FIX ME!!! check it)
%% \begin{eqnarray}
%% \left|\psi\right>_{123} = \left|\psi\right>_1 \left|\psi\right>_{23} = 
%% \nonumber \\
%% = \left|\psi^{-}\right>_{12} \frac{\alpha\left|V\right>_3 + \beta\left|H\right>_3}{2} + 
%% \nonumber \\
%% + 
%% \left|\psi^{\dag}\right>_{12} \frac{- \alpha\left|V\right>_3 + \beta\left|H\right>_3}{2} +
%% \nonumber \\
%% + 
%% \left|\phi^{\dag}\right>_{12} \frac{- \beta\left|V\right>_3 + \alpha\left|H\right>_3}{2} +
%% \nonumber \\
%% +
%% \left|\phi^{-}\right>_{12} \frac{\beta\left|V\right>_3 + \alpha\left|H\right>_3}{2}.
%% \nonumber
%% \end{eqnarray}

%% Таким образом всякий раз когда Алиса регистрирует пару фотонов 1 и 2 в белловском сотоянии
%% $\left|\psi^{-}\right>_{12}$ то фотон 3 на стороне Боба будет находится в состоянии
%% $\alpha\left|V\right>_3 + \beta\left|H\right>_3$, то есть произойдет как бы телепортация фотона 1 к Бобу.

%% Белловское состояние $\left|\psi^{-}\right>_{12}$ может быть зарегистрировано Алисой по одновременному срабатыванию 
%% фотодетекторов PH1 и PH2 (FIX ME!!! add calculations). В случае других белловских состояний оба фотона уйдут
%% или на детектор PH1 или на PH2.

%% Стоит отметить два факта
%% \begin{itemize}
%% \item При данной схеме телепортации передается не материя, а некоторая информация о квантовом объекте.
%% \item Передача этой информации мгновенна, но вместе с тем для того чтобы Боб смог узнать о факте телепортации необходим классический канал связи 
%% между Алисой и Бобом \footnote{По этому каналу Алиса говорит Бобу когда у нее сработали оба фотодетектора сообщая тем самым о факте телепортации}.
%% \end{itemize}
