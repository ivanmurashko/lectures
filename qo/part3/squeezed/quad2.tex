%% -*- coding:utf-8 -*- 
\section{Сжатие квадратурного состояния}
Сначала рассмотрим теоретическую возможность создания сжатых
квадратурных состояний. Рассмотрим построение сжатого состояния из
когерентного состояния, в котором будет сжата одна из квадратурных
составляющих. Преобразование состояний производится при помощи
некоторого унитарного преобразования.

Рассмотрим унитарный оператор, который называют оператором сжатия
(причина такого названия станет ясна из дальнейшего)
\begin{equation}
\hat{S}\left(z\right) = e^{\frac{1}{2}z^{*}\hat{a}^2 -
\frac{1}{2}z\hat{a}^{+2}},
\nonumber
\end{equation}
где $z = r e^{i\theta}$ - произвольное комплексное число.
Сопряженный оператор имеет вид
\begin{equation}
\hat{S}^{\dag}\left(z\right) = e^{\frac{1}{2}z\hat{a}^{+2} -
\frac{1}{2}z^{*}\hat{a}^{2}}, 
\nonumber
\end{equation}
откуда видно, что
\begin{equation}
\hat{S} \hat{S}^{\dag} = \hat{S}^{+} \hat{S} = \hat{I},
\nonumber
\end{equation}
где $\hat{I}$ является единичным оператором. Таким образом, $\hat{S}^{\dag} =
\hat{S}^{-1}$ действительно является унитарным оператором.

Рассмотрим действие оператора $\hat{S}\left(z\right)$ на операторы
$\hat{a}$ и $\hat{a}^{\dag}$ и на когерентное состояние. Для этой цели
применим теорему об операторном разложении (подробный вывод
см. \myref{thm:addoperatorequality}{теорему об операторном тождестве}):  
\begin{equation}
e^{\hat{A}}\hat{B}e^{-\hat{A}} = 
\hat{B} + \left[\hat{A},\hat{B}\right] + 
\frac{1}{2!} \left[\hat{A},\left[\hat{A},\hat{B}\right]\right] + \dots
\label{eqPart3Squeezed15}
\end{equation}
При помощи \eqref{eqPart3Squeezed15} легко показать, если принять
\[
e^{\hat{A}} = \hat{S}\left(z\right), \, \hat{B} = \hat{a},
\]
что
\begin{eqnarray}
\hat{A} = \hat{S}\left(z\right)\hat{a}\hat{S}^{\dag}\left(z\right) =
\hat{a} + z \hat{a}^{\dag} + \frac{\left|z\right|^2\hat{a}}{2!} +
\frac{z\left|z\right|^2\hat{a}^{\dag}}{3!} + \dots = 
\nonumber \\
=\hat{a} ch\,r + \hat{a}^{\dag} e^{i\theta} sh \, r = 
\mu \hat{a} + \nu \hat{a}^{\dag},
\label{eqPart3Squeezed16}
\end{eqnarray}
где $\mu = ch\,r$, $\nu = e^{i\theta} sh\,r$, $\left|\mu\right|^2 -
\left|\nu\right|^2  = 1$.
Таким же образом можно показать, что при обратном порядке действия
операторов $\hat{S}^{\dag}\dots\hat{S}$ получается формула, которая
отличается от \eqref{eqPart3Squeezed16} только знаком между
слагаемыми:
\begin{eqnarray}
\hat{S}^{\dag}\left(z\right)\hat{a}\hat{S}\left(z\right) 
=\hat{a} ch\,r - \hat{a}^{\dag} e^{i\theta} sh \, r,
\nonumber \\
\hat{S}^{\dag}\left(z\right)\hat{a}^{+}\hat{S}\left(z\right) 
=\hat{a}^{\dag} ch\,r - \hat{a} e^{-i\theta} sh \, r.
\label{eqPart3Squeezed16a}
\end{eqnarray}

Подействуем оператором $\hat{S}$ на вектор когерентного состояния. При
этом мы получим новое состояние:
\begin{equation}
\left|\alpha, z\right> = \hat{S}\left(z\right)\left|\alpha\right>.
\label{eqPart3Squeezed17}
\end{equation}
Покажем теперь, что состояние $\left|\alpha, z\right>$ является
собственным состоянием оператора $\hat{A}$:
\begin{eqnarray}
\hat{A}\left|\alpha, z\right> = 
\hat{S}\left(z\right)\hat{a}\hat{S}^{\dag}\left(z\right)\hat{S}\left(z\right)\left|\alpha\right>
= 
\hat{S}\left(z\right)\hat{a}\left|\alpha\right> = 
\nonumber \\
= \alpha \hat{S}\left(z\right)\left|\alpha\right> = 
\alpha \left|\alpha, z\right>.
\label{eqPart3Squeezed18}
\end{eqnarray}
Здесь использовано определение $\hat{A}$ и $\hat{S}\hat{S}^{\dag} = \hat{I}$.
Из \eqref{eqPart3Squeezed18} следует, что состояние $\left|\alpha, z\right>$ является
собственным состоянием оператора $\hat{A}$, а собственное число совпадает с
собственным числом когерентного состояния, из которого получено
состояние $\left|\alpha, z\right>$.
Сопряженное равенство имеет вид
\begin{equation}
\left<\alpha, z\right|\hat{A}^{\dag} = 
\alpha^{*}\left<\alpha, z\right|
\label{eqPart3Squeezed19}
\end{equation}
Исходя из соотношений \eqref{eqPart3Squeezed18} и
\eqref{eqPart3Squeezed19}, по аналогии с операторами $\hat{a}$ и
$\hat{a}^{\dag}$, $\hat{A}$ и $\hat{A}^{+}$ называют операторами
квазирождения и квазиуничтожения.

Остается выяснить, является ли состояние $\left|\alpha, z\right>$
сжатым состоянием? Для этого надо написать соотношение
неопределенности для $\Delta X_1$  и $\Delta X_2$. Имеем
\begin{equation}
\left(\Delta X_{1,2}\right)^2 = 
\left<\alpha, z\right| \hat{X}_{1,2}^2\left|\alpha, z\right> -
\left<\alpha, z\right| \hat{X}_{1,2}\left|\alpha, z\right>^2.
\nonumber
\end{equation}
Это выражение, используя \eqref{eqPart3Squeezed17}, можно представить в
виде 
\begin{equation}
\left(\Delta X_{1,2}\right)^2 = 
\left<\alpha\right|\hat{S}^{\dag}\left(z\right) \hat{X}_{1,2}^2\hat{S}\left(z\right)\left|\alpha\right> -
\left<\alpha\right|\hat{S}^{\dag}\left(z\right) \hat{X}_{1,2}\hat{S}\left(z\right)\left|\alpha\right>^2.
\label{eqPart3Squeezed20}
\end{equation}
Операторы $\hat{X}_{1,2}$ выражаются через $\hat{a}$ и $\hat{a}^{\dag}$:
\begin{eqnarray}
\hat{X}_1 = \frac{1}{2}\left(\hat{a} + \hat{a}^{\dag}\right), 
\nonumber \\
\hat{X}_2 = \frac{1}{2 i}\left(\hat{a} - \hat{a}^{\dag}\right).
\nonumber
\end{eqnarray}
Подставляя это в \eqref{eqPart3Squeezed20}, преобразуя операторы
$\hat{a}$ и $\hat{a}^{\dag}$ при помощи \eqref{eqPart3Squeezed16a},
получим окончательное выражение для $\Delta X_1$  и $\Delta X_2$:
(при условии $z = r$, т.е. $\theta = 0$)
%(подробный вывод смотри в приложении FIXME!!! add it):
\begin{eqnarray}
\left(\Delta X_1\right)^2 = \frac{1}{4}e^{-2 r},
\nonumber \\
\left(\Delta X_2\right)^2 = \frac{1}{4}e^{2 r},
\nonumber \\
\left(\Delta X_1 \Delta X_2\right) = \frac{1}{4},
\label{eqPart3Squeezed21}
\end{eqnarray}
где $r = \left|z\right|$ является параметром сжатия. Из
\eqref{eqPart3Squeezed21} следует, что состояние $\left|\alpha,
z\right>$ действительно является сжатым состоянием для одной из
квадратурных компонент. Поскольку состояние неопределенности имеет
минимальное значение, состояние является идеально сжатым
состоянием. Мы будем называть это состояние идеально сжатым
квадратурным состоянием.

Если воздействовать оператором $\hat{S}\left(z\right)$ на вакуумное
состояние ($\alpha = 0$), получим сжатое вакуумное состояние (сжатый
вакуум)
\begin{equation}
\hat{S}\left(z\right)\ket{0} = \ket{z, 0}.
\nonumber
\end{equation}

Среднее число фотонов в квадратурно сжатом когерентном состоянии
определяется выражением:
\begin{eqnarray}
\left<\alpha, z\right|\hat{a}^{\dag}\hat{a}\left|\alpha, z\right> =
\left<\alpha\right|\hat{S}^{\dag}\hat{a}^{+}\hat{a}\hat{S}\left|\alpha\right>
=
\nonumber \\
=
\left<\alpha\right|\hat{S}^{\dag}\hat{a}^{+}\hat{S}\hat{S}^{+}\hat{a}\hat{S}\left|\alpha\right>
= 
\nonumber \\
=
\left<\alpha\right|
\left(\hat{a}^{\dag} ch\,r - \hat{a}e^{-i\theta}sh\,r\right)
\left(\hat{a} ch\,r - \hat{a}^{\dag} e^{i\theta}sh\,r\right)
\left|\alpha\right> = 
\nonumber \\
=
\left(
\left|\alpha\right|^2\left(ch^2 r + sh^2 r\right) -
\left(\alpha^{*}\right)^2
e^{i\theta} sh\,r ch\,r - 
\alpha^2 sh\,r ch\,r e^{- i\theta} + sh^2 r
\right).
\label{eqPart3Squeezed23}
\end{eqnarray}
При выводе \eqref{eqPart3Squeezed23} мы использовали соотношение
\eqref{eqPart3Squeezed16a}.

\index{Сжатое вакуумное состояние}
\index{сжатый вакуум}
Сжатому вакуумному состоянию соответствует $\alpha = 0$. Тогда среднее
число фотонов в этом состоянии
\begin{equation}
\bra{0, z}\hat{a}^{\dag}\hat{a}\ket{0, z} =
sh^2 r \ne 0,
\label{eqPart3Squeezed24}
\end{equation}
где $r$ - параметр сжатия.

Очевидно, что состояние сжатого вакуума не является настоящим
вакуумным состоянием, т. к. может быть обнаружено фотоприемником,
использующим фотоэффект, что для истинного вакуумного состояния
невозможно.

\input ./part3/squeezed/fig1.tex
\input ./part3/squeezed/fig2.tex
\input ./part3/squeezed/fig3.tex
%\input ./part3/squeezed/fig4.tex
%\input ./part3/squeezed/fig5.tex

Наглядно квадратурно сжатые  состояния представлены на рисунках
\ref{figPart3Squeezed_1}-\ref{figPart3Squeezed_3}, где представлены
начальные состояния и развитие колебаний во 
времени для различных сжатых состояний. На правых графиках изображены
области неопределенности, которые имеют форму элипса. Такую картину
можно было бы увидеть, если бы мы имели что-то вроде оптического
стробоскопического осциллографа, которого в действительности нет.

Рассмотрим полученные состояния более подробно. Формулу
\eqref{eqPart3Squeezed17} можно написать несколько иначе, если
воспользоваться представлением когерентного состояния через вакуумное
в форме \eqref{eqCh1_astate4squeezed}:
\begin{equation}
\left|\alpha\right> =  
e^{\alpha \hat{a}^{\dag} - \alpha^{*} \hat{a}}\ket{0} = 
\hat{D}\ket{0},
\nonumber
\end{equation}
где $\hat{D} = e^{\alpha \hat{a}^{\dag} - \alpha^{*} \hat{a}}$ называют
оператором смещения. $\hat{D}$ является унитарным оператором, т. к. 
\[
\hat{D}^{\dag} = e^{-\left(\alpha \hat{a}^{+} - \alpha^{*} \hat{a}\right)}
\]
откуда
\[
\hat{D}^{\dag} \hat{D} = \hat{D} \hat{D}^{+} =\hat{I},
\]
т. е. $\hat{D}^{\dag} = \hat{D}^{-1}$.

Название оператора связано с тем, что действие
$\hat{D}\left(\alpha\right)$ на операторы $\hat{a}$ и $\hat{a}^{\dag}$
приводит к их смещению на величину $\alpha$ ($\alpha^{*}$):
\begin{eqnarray}
\hat{D}^{\dag}\left(\alpha\right)\hat{a}\hat{D}\left(\alpha\right) =
\hat{a} + \alpha,
\nonumber \\
\hat{D}^{\dag}\left(\alpha\right)\hat{a}^{+}\hat{D}\left(\alpha\right) =
\hat{a}^{\dag} + \alpha^{*},
\label{eqPart3SqueezedTaskOffset}
\end{eqnarray}
Этот результат следует из теоремы об операторном разложении
\eqref{eqPart3Squeezed15}, если принять $\hat{A} = - \alpha \hat{a}^{\dag}
+ \alpha^{*} \hat{a}$, $B=\hat{a}, \hat{a}^{\dag}$.

\input ./part3/squeezed/fig6.tex

Таким образом, когерентное состояние $\left|\alpha\right>$ является
смещенным на $\alpha$ вакуумным состоянием
(см. \autoref{figPart3Squeezed_6}).
%где штриховкой показана область когерентности). 
Исходя из сказанного, можно представить сжатое
состояние как результат двух действий: смещение вакуумного состояния и
последующего его сжатия, как это показано на
\autoref{figPart3Squeezed_7}. 

\input ./part3/squeezed/fig7.tex
\input ./part3/squeezed/fig8.tex

Рассмотренная нами процедура сжатия не является единственной. Можно
применить другую последовательность действий - сперва сжать вакуумное
состояние, а затем сместить его, получив новое сжатое состояние
(см. \autoref{figPart3Squeezed_8}). В операторном виде эту операцию
можно представить так:
\begin{equation}
\left|\alpha, z\right> =
\hat{D}\left(\alpha\right)\hat{S}\left(z\right) \ket{0}.
\label{eqPart3Squeezed28}
\end{equation}
Для различия состояний последовательность параметров здесь описана
обратная по сравнению с 
\[
\left|z,\alpha\right> =
\hat{S}\left(z\right) \hat{D}\left(\alpha\right)\ket{0}.
\]
Состояние \eqref{eqPart3Squeezed28} также относится к квадратурно
сжатому состоянию, но между собой они не вполне совпадают. Подробно
рассматривать этот случай мы здесь не будем.

\input ./part3/squeezed/fig9.tex

Приведенные выше результаты и иллюстрации, изображенные на
\autoref{figPart3Squeezed_1}-\ref{figPart3Squeezed_3} получены для
случая, когда параметры $z = r e^{i\theta}$ и $\alpha =
\left|\alpha\right|e^{i \varphi}$ вещественны ($\varphi = \theta =
0$). Если параметры комплексные, то картинка несколько изменится
(см. \autoref{figPart3Squeezed_9}): ось сжатия повернута на угол $\frac{\theta}{2}$
относительно осей $X_1$, $X_2$, а положение центра области
неопределенности повернуто на угол $\varphi$ относительно
$X_1$. Изменяя в процессе генерации и регистрации фазы действующих там
световых пучков, мы можем перейти к новым квадратурам $Y_1$, $Y_2$, как
это изображено на \autoref{figPart3Squeezed_9}. Если еще компенсировать сдвиг фазы
$\varphi$, мы перейдем к ситуации, которая изображена на рисунках 
\ref{figPart3Squeezed_1}-\ref{figPart3Squeezed_3}. Из  
\autoref{figPart3Squeezed_9} видно, что измерением углов $\theta$ и
$\varphi$ можно, например, состояние, сжатое по $X_1$, превратить в
состояние, сжатое по $X_2$, и наоборот.
