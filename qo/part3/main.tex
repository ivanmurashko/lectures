%% -*- coding:utf-8 -*- 
\part{Неклассический свет}

\input ./part3/nonclass/nonclass.tex
\input ./part3/squeezed/main.tex
\input ./part3/entang/main.tex

%% \begin{thebibliography}{99}
%% \bibitem{bNonclassSmirnovTroshin} Д. Ф. Смирнов, А. С. Трошин ``Новые
%%   явления в квантовой оптике: антигруппировка и субпуассоновская
%%   статистика фотонов, сжатые состояния'', УФН 153 , 2 (1987).
%% \bibitem{bNonclassSqueezedStateDetection}
%%  Y. Yamamoto, H. A. Haus,
%% ``Preparation, measurement and information capacity of optical quantum
%%   states'',
%% Rev. Mod. Phys. 58, 1001 - 1020 (1986).
%% \bibitem{bPhisQuantInfoOptics} Физика квантовой информации, под
%%   ред. А. Цайлингера. М. Постмаркет 2002
%% \bibitem{bEntangKlyshko} Д. Н. Клышко Основные понятия квантовой
%%   физики с операциональной точки зрения. УФН 168, 9
%% \bibitem{bEPR}  A. Einstein, B. Podolsky, and N. Rosen, Can
%%   quantum-mechanical description of physical reality be considered
%%   complete. Phys. Rev. 47 777 (1935).  
%% \bibitem{bKlishko} Д. Н. Клышко. Фотоны и нелинейная оптика. 
%% М. Наука 1980г. 256 с.
%% \bibitem{bBell} J. S. Bell, On the Einstein Podolsky Rosen Paradox, Physics 1, 195 (1964)
%% \bibitem{bBellTest} Weihs, 1998: G. Weihs, et al., Violation of Bell's
%%   inequality under strict Einstein locality conditions,
%%   Phys. Rev. Lett. 81, 5039 (1998) 
%% \bibitem{bBelokTimHrus} В.В.Белокуров, О.Д.Тимофеевская,
%%   О.А.Хрусталев. Квантовая телепортация - обыкновенное чудо. - Ижевск:
%%   НИЦ ``Регулярная и хаотическая динамика''. 2000. - 256с. 
%% \bibitem{bFeinman} Р. Фейнман, Р. Лейтон, М. Сэндс. Феймановские
%%   лекции по физике (вып. 8, 9). Квантовая механика. - М. Мир. 1978. -
%%   525с.  
%% \bibitem{bKulik} Y. H. Kim, S. P. Kulik, Y. Shih, Quantum
%%   teleportation with a complete Bell state
%%   measurement. Phys. Rev. Lett. 86 N 7, pp. 1370-1373
%% \bibitem{bNoClone} W.K. Wootters and W.H. Zurek, A Single Quantum
%%   Cannot be Cloned, Nature 299 (1982), pp. 802-803. 
%% \end{thebibliography} 


%% \chapter{Квантовые вопросы нелинейной оптики}

%% Генерация второй гармоники. Параметрическая генерация квантового
%% электромагнитного поля.

%% TBD

%% \chapter{Квантовые проблемы оптической связи}

%% Прием слабых оптических полей. Режим счета фотонов. Гетеродинный
%% прием. Элементы кватновой теории связи. Квантовое выражение для
%% информации. Вероятность плотного приема и т. п. Квантовая
%% криптография. 

%% TBD



% \chapter{Некоторые ``философские'' вопросы квантовой оптики}

% ``Что такое фотон?'' Статьи Лэмб ``Антифотон'' Клышко и т. п.
